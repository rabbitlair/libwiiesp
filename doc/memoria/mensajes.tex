% Este archivo es parte de la memoria de libWiiEsp, protegida bajo la 
% licencia GFDL. Puede encontrar una copia de la licencia en el archivo fdl-1.3.tex

% Fuente tomada de la plantilla LaTeX para la realización de Proyectos Final 
% de Carrera de Pablo Recio Quijano.

% Copyright (C) 2009 Pablo Recio Quijano
% Copyright (C) 2011 Ezequiel Vázquez de la Calle

% -*-mensajes.tex-*-

Debido a la complejidad de la depuración de código en la plataforma Nintendo Wii (especialmente en lo referente a depuración en tiempo de ejecución), se hace patente la necesidad de un módulo que permita al desarrollador conocer el estado de las variables y expresiones cuando se está ejecutando un programa en el que se detecta un error. Para cubrir estas situaciones, se crea un sistema de gestión de errores basado en excepciones, que se aplica a todas las clases implementadas hasta el momento, y un módulo que permita generar un registro de mensajes, o \emph{logs}, a partir de la información que el programador estime oportuna.

\subsection{Excepciones}

\programa{LibWiiEsp} proporciona una jerarquía de excepciones que permite controlar los distintos comportamientos erróneos que sucedan durante la ejecución de un programa. Todas las excepciones derivan de la clase \emph{Excepcion}, que a su vez hereda de la clase \emph{std::exception}, de tal manera que es compatible con una gestión de excepciones estándar.\\

Para conocer el mensaje de error que contienen todas las excepciones, puede utilizarse el método \emph{what}, que devuelve una cadena de caracteres de tipo \emph{string} con la información que se haya indicado al lanzar la excepción.\\

Las excepciones que se contemplan son las siguientes:

\begin{itemize}
\item \textbf{Excepcion}: Clase de excepción que sirve como base a todas las demás.
\item \textbf{ArchivoEx}: Excepción para controlar errores relacionados con la apertura de archivos.
\item \textbf{CodigoEx}: Sirve para gestionar los fallos que puedan producirse al trabajar con códigos identificativos.
\item \textbf{ImagenEx}: Para tratar los errores concretos de la clase Imagen.
\item \textbf{NunchuckEx}: Esta clase de excepción se utiliza para informar de situaciones relacionadas con el \emph{Nunchuck}.
\item \textbf{TarjetaEx}: Generalmente sólo se utiliza para indicar que la tarjeta SD no está disponible.
\item \textbf{XmlEx}: Excepción que permite identificar cuándo sucede un error al trabajar con un árbol XML.
\end{itemize}

\subsection{Identificación de funcionalidad necesaria}

La gestión de mensajes del sistema debe estar centralizada en un punto del sistema, que además resulte accesible desde cualquier punto de un programa. Lo que se pretende con la gestión de errores y el registro de mensajes consiste en almacenar textos con información relevante sobre el estado del sistema en un momento determinado. Dicha información se guardará temporalmente en un \emph{búffer} en memoria y será posteriormente volcada a un archivo de texto plano en la tarjeta SD.\\

Se permitirán tres niveles de prioridad para los mensajes: errores (sólo se muestran los mensajes de situaciones que generen una interrupción en la ejecución del programa), avisos (se muestran los anteriores y también aquellos derivados de los casos en que, sin llegar a interrumpir la ejecución, provocan un comportamiento anómalo en el sistema), e información, que muestra todos los mensajes definidos.\\

También debe poderse apagar el sistema de errores, de tal manera que no genere un archivo de \emph{logs} o registro cuando un programa ya esté en producción.

\subsection{Diseño e implementación}

Para el registro de mensajes del sistema, se crea una clase que implementa el patrón \emph{Singleton}, ya que queremos centralizar toda la gestión de mensajes en un único punto del sistema, y tenerlo siempre accesible. Esta clase tendrá como atributos privados el nivel de los mensajes que quieren observar, un \emph{búffer} para almacenar todos los textos que se escribirán en el archivo de \emph{logs} (que es una cadena de caracteres \emph{string}) y un flujo de salida de bytes a fichero, en el que se realizará el volcado de información desde el \emph{búffer} hacia el archivo.\\

Los niveles de registro están agrupados en una enumeración (con los valores OFF, ERROR, AVISO, INFO) que, a su vez, se define como tipo público de la clase para facilitar su utilización. El orden es desde el nivel más restrictivo (OFF, no se registra nada) a más amplio (INFO, se registra todo). Un nivel más amplio incluye a los más restrictivos que él, lo cual quiere decir que si, por ejemplo, se activara el nivel AVISO, se registrarían tanto mensajes de aviso como de error.\\

Cuando se inicializa el sistema de registro de mensajes, se guarda el nivel de \emph{log} introducido como parámetro, y si éste no es OFF (no registrar nada), se sigue adelante. Se comprueba que la tarjeta SD esté disponible para realizar operaciones de entrada/salida, se abre el archivo de \emph{log} indicado en el primer parámetro en modo escritura (si no existiera, se crea), y se establece el \emph{búffer} para los mensajes. La instancia irá registrando mensajes en el búfalo, cada uno en una línea, y comenzando por las cadenas [INFO] para el nivel de información, [AVISO] para el nivel de aviso, y [ERROR] para una cadena que contenga un mensaje con nivel de registro error.\\

El guardado de mensajes se realiza por orden de ejecución, es decir, si un mensaje se encuentra por debajo de otro en el archivo de texto plano, significa que se ejecutó posteriormente al segundo. Por último, hay que tener en cuenta que el volcado del \emph{búffer} en el archivo se realiza cuando se destruye la instancia de la clase. La ventaja de esta decisión es que, al realizarse un único guardado, el rendimiento del programa no se ve afectado con las muchas operaciones de escritura que pueden darse durante la ejecución. Sin embargo, existe un inconveniente, ya que si ocurriera un error que no estuviera controlado en el código, se generaría un \emph{pantallazo} y se finalizaría la ejecución sin guardar el registro de mensajes en el archivo. En versiones posteriores se mejorará este aspecto, haciendo posible que se puedan guardar la información generada hasta un cierto momento, a voluntad del programador.\\

En la figura \ref{logger} se muestra la interfaz pública de la clase Logger.\\

\figura{logger.png}{scale=0.6}{Interfaz pública de la clase Logger}{logger}{h}

\subsection{Pruebas}

Para comprobar el buen funcionamiento de esta clase, se creó un pequeño programa de prueba que escribía mensajes de sistema en distintas prioridades, y se probó este programa con los distintos niveles posibles para la clase (OFF, ERROR, AVISO, INFO).

