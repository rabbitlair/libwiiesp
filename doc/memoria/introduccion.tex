% Este archivo es parte de la memoria de libWiiEsp, protegida bajo la 
% licencia GFDL. Puede encontrar una copia de la licencia en el archivo fdl-1.3.tex

% Fuente tomada de la plantilla LaTeX para la realización de Proyectos Final 
% de Carrera de Pablo Recio Quijano.

% Copyright (C) 2009 Pablo Recio Quijano
% Copyright (C) 2011 Ezequiel Vázquez de la Calle

% -*-introduccion.tex-*-

Actualmente, el sector de los videojuegos está en alza, llegando en algunos países europeos a facturar más que la industria cinematográfica. A lo largo de los años, han visto la luz multitud de sistemas especializados en la ejecución de estas aplicaciones software de entretenimiento, las videoconsolas. Sin embargo, es común en este mundillo que los fabricantes no permitan la ejecución de código no firmado por ellos en dichas plataformas. Gracias a la labor de muchas personas a lo largo y ancho del planeta, se ha hecho posible el desarrollo de software casero en algunas videoconsolas (por ejemplo, en la \emph{Sony PlayStation 3}, o en la \emph{Nintendo Wii}). Generalmente, las herramientas para construir y ejecutar este software suele ser de muy bajo nivel, y, en muchas ocasiones, de código cerrado y sin documentación técnica alguna disponible, por lo que se hace patente la necesidad de facilitar al gran público herramientas libres que cubran las posibilidades de desarrollo en estas videoconsolas.\\

A continuación, se describe de forma general el contenido del proyecto y de esta memoria.

\section{Objetivos}

Los objetivos de este proyecto son varios, entre los cuales destacan especialmente el afán de conocimiento sobre el funcionamiento de una videoconsola (en este caso concreto, la Nintendo Wii), la puesta en práctica de los conocimientos adquiridos durante las asignaturas de la titulación de \emph{Ingeniería Técnica en Informática de Gestión}, la profundización en los métodos de programación de videojuegos de dos dimensiones (ampliando lo aprendido en la asignatura \emph{Diseño de Videojuegos}), y el deseo de aportar una herramienta completa, libre, de alto nivel y documentada en español para el desarrollo de videojuegos en la plataforma Nintendo Wii.\\

La lista completa de objetivos que se persiguen con la realización del proyecto son los siguientes:

\begin{itemize}
\item \textbf{Aportar una herramienta completa de desarrollo de videojuegos 2D en Wii}: se pretende ofrecer al mundo del \emph{software libre} una forma de desarrollar videojuegos en dos dimensiones para la consola Nintendo Wii, que resulte sencilla de utilizar, pero que a su vez permita construir juegos completos.
\item \textbf{Proporcionar una visión general sobre el funcionamiento de la consola}: una videoconsola no es más que un ordenador dedicado exclusivamente a la ejecución de videojuegos; sin embargo, existen numerosas diferencias en el desarrollo de un programa para una consola respecto a hacerlo para un ordenador personal. Si bien no se desea profundizar al máximo en este asunto, sí se quiere aportar una idea más o menos completa del cambio que supone elegir una plataforma de desarrollo u otra.
\item \textbf{Ofrecer una documentación completa en español}: actualmente, existen pocas herramientas de desarrollo libres para Nintendo Wii, y las que hay son de muy bajo nivel, además de estar poco o nada documentadas. Con este proyecto se busca proporcionar una documentación completa, detallada y en español, de tal manera que cualquier persona con ciertos conocimientos sobre programación de videojuegos y orientación a objetos pueda desarrollar fácilmente un videojuego para la videoconsola.
\item \textbf{Desarrollar tres juegos de ejemplo}: la mejor manera de dominar una herramienta software es practicar con ella, pero siempre es conveniente tener un producto acabado que sirva de referencia. En este proyecto se quiere aportar tres juegos, totalmente funcionales y relativamente completos, que ilustren los resultados hasta los que se puede llegar con la biblioteca \programa{LibWiiEsp}.
\subitem \textbf{Arkanoid}: clon del clásico de \emph{Taito}, en el que el jugador debe destruir ladrillos, golpeándolos con una pelota que rebota en las paredes del escenario, y debe evitar también que la pelota caiga hacia la zona inferior de la pantalla.
\subitem \textbf{Duck Hunt}: basado en el clásico de \emph{Nintendo}. En lugar del comportamiento del videojuego de 1984, en éste participan dos jugadores a la vez. El objetivo de una partida consistirá en ser el primero de los dos en abatir un número concreto de patos.
\subitem \textbf{Wii Pang}: clon del clásico \programa{Pang} de \emph{Mitchel Co.} de 1989. El jugador controla a un personaje que debe evitar ser aplastado por unas pompas de colores que rebotan en el escenario del juego. Este personaje lanza ganchos verticales para romper las pompas en otras dos más pequeñas, que se vuelven a dividir en dos cuando reciben otro impacto sucesivamente hasta desaparecer al llegar a su tamaño más pequeño.
\end{itemize}

\section{Alcance}

Este proyecto pretende cubrir la escasez de herramientas libres que permiten desarrollar videojuegos en dos dimensiones para la consola Nintendo Wii, aportando una biblioteca con la que resulte fácil, pero a la vez eficiente, la construcción de juegos para esta plataforma. También proporciona una documentación útil y completa, enteramente en español, en contraposición a la poca información disponible sobre este tema, y que en general sólo puede encontrarse en inglés.

\subsection{Identificación del producto}

El producto resultante de este proyecto es \programa{LibWiiEsp}, una biblioteca libre y completa, pensada para hacer posible, de una forma sencilla y eficaz, el desarrollo de videojuegos libres en dos dimensiones para Nintendo Wii.

\subsection{Funcionalidades}

Esta biblioteca, escrita en C++ \cite{abur01} y publicada bajo licencia GPLv3, proporciona una interfaz que permite interactuar con los mandos, el sistema gráfico, el sistema de sonido y el lector de tarjetas SD de la consola. Además, incluye un \programa{parser} de XML sencillo pero efectivo, un sistema de soporte de internacionalización basado en ficheros XML, un sistema de gestión de contenido multimedia (imágenes, efectos de sonido, pistas de música y fuentes de texto), registro de eventos del sistema (\emph{logging}), creación de animaciones a partir de una imagen organizada en rejilla (\emph{spritesheet}) y un módulo de detección de colisiones basado en figuras planas fácilmente ampliable.\\

Como último punto a destacar, \programa{LibWiiEsp} incluye tres clases abstractas pensadas para ser utilizadas como plantillas para la creación de actores, niveles y la clase principal del videojuego (en la que se controla el bucle principal). La plantilla para niveles permite crear éstos con la herramienta libre Tiled \cite{website:tiled}, de tal manera que se facilita la creación de escenarios nuevos una vez terminada la programación del juego.\\

Parte importante del proyecto es también la documentación, que incluye un manual de instalación del entorno y de uso de las plantillas, y un manual de referencia completo.\\

A pesar de que es posible utilizar \programa{LibWiiEsp} en sistemas \programa{Windows}, \programa{GNU/Linux} y \programa{Mac}, la documentación sólo contempla los sistemas \programa{GNU/Linux}. Además, cabe destacar que el proceso de aprendizaje a la hora de utilizar la biblioteca requiere un esfuerzo moderado en un principio, debido al deseo de cubrir todos los aspectos del desarrollo de un videojuego en dos dimensiones.

\subsection{Aplicaciones del software}

El principal beneficio que aporta \programa{LibWiiEsp} es el de proporcionar, a cualquier persona con conocimientos de C++ y orientación a objetos, la posibilidad de desarrollar videojuegos en dos dimensiones en la plataforma Nintendo Wii.\\

El aporte de documentación completamente en español y el empleo de técnicas de programación adquiridas durante el transcurso de la titulación de \emph{Ingeniería Técnica en Informática de Gestión}, unido a la sencillez que aporta a la hora del desarrollo, hacen que el producto sea ideal para aprender y poner en práctica los procesos implicados en el diseño y creación de un videojuego en dos dimensiones.\\

Por otra parte, la evolución natural de una herramienta como es \programa{LibWiiEsp} podría producir la creación de una comunidad hispana de desarrolladores de videojuegos libres para la consola de Nintendo; si bien este hecho puede considerarse algo ambicioso, es una posibilidad a contemplar.

\section{Definiciones, abreviaturas y acrónimos}

\begin{itemize}
\item \textbf{Scener}: persona que ha colaborado en la obtención de información sobre cómo ejecutar código casero en un sistema cerrado, como una videoconsola o un \emph{smartphone}. Un \emph{scener} suele desarrollar aplicaciones libres basadas en la posibilidad de ejecución de software no firmado digitalmente por el fabricante (por ejemplo, un equipo de cuatro personas ha desarrollado un reproductor multimedia para la Nintendo Wii).
\item \textbf{Libogc}: biblioteca de muy bajo nivel, escrita en C, y desarrollada por varios \emph{sceners}. Brinda acceso a todo el hardware de la consola, pero es bastante compleja de utilizar.
\item \textbf{Spritesheet}: imagen o textura, organizada en rejilla, en la que cada recuadro de la rejilla contiene un fotograma de una animación.
\item \textbf{Textura organizada en \emph{tiles}}: una textura se representa en memoria como un flujo de datos que contiene la información de los píxeles de una imagen. La organización lineal de una textura consiste en que, en el flujo de datos, se recibe la información de los píxeles en un orden de izquierda a derecha y de arriba hacia abajo (es decir, en un recorrido de los píxeles por filas y columnas). Sin embargo, cuando Nintendo Wii trabaja con una textura, requiere que la información de los píxeles se distribuya organizada en \emph{tiles} o grupos de píxeles, de forma que, si un par de puntos son adyacentes en la imagen, también lo sean en su representación en memoria. Como consecuencia de esto, la información de los píxeles de una textura se reciben, en el flujo de datos, en grupos de 4x4 píxeles que se organizan como puede apreciarse en la figura \ref{texturatiles}.\\

\figura{texturatiles.png}{scale=0.5}{Organización de una textura en \emph{tiles}, y orden de sus píxeles en el flujo de datos}{texturatiles}{h}

\item \textbf{GPU}: \emph{Graphics Processing Unit} o Unidad de Procesamiento de Gráficos, hace referencia al chip dedicado al procesamiento de gráficos u operaciones de coma flotante, para aligerar la carga de trabajo del procesador central.
\item \textbf{EFB}: \emph{Embedded Frame Buffer}. Es el \emph{búffer} interno con el que trabaja el procesador gráfico (GPU) de Nintendo Wii, el cual recibe un flujo de datos y se encarga de dibujar la información en la pantalla a cada fotograma.
\item \textbf{Proyección ortográfica}: la proyección ortográfica es un sistema de representación gráfica, consistente en representar elementos geométricos o volúmenes en un plano (en nuestro caso, en la pantalla), mediante proyección ortogonal; se obtiene de modo similar a la sombra generada por un foco de luz procedente de una fuente muy lejana (en el infinito). En una proyección de este tipo, parece haber únicamente dos dimensiones, y la profundidad no se tiene en cuenta: un objeto situado en primer plano tiene la misma proporción que otro que se encuentre en un punto alejado del plano donde se proyecta.
\end{itemize}

\section{Visión general de la memoria}

Este documento sigue, de una manera más o menos fiel, las pautas recogidas por varios profesores del Departamento de Lenguajes y Sistemas Informáticos en el documento \emph{Recomendaciones para la realización de la Documentación del Proyecto de Fin de Carrera}. A continuación se describen de forma general los capítulos que componen la memoria:

\begin{itemize}
\item \textbf{Introducción}: Identificación de objetivos, alcance y aplicaciones del proyecto. Vistazo general de la memoria.
\item \textbf{Planificación temporal}: Desarrollo del calendario que se ha seguido a la hora de realizar el proyecto.
\item \textbf{Descripción general del proyecto}: Visión general del proyecto, ampliando la información aportada en la sección introductoria.
\item \textbf{Desarrollo del Proyecto}: Descripción en profundidad de todos los aspectos relativos al desarrollo del proyecto.
\item \textbf{Pruebas}: Descripción de las distintas pruebas que se han realizado para comprobar y validar los componentes de la biblioteca.
\item \textbf{Conclusiones}: Valoración global del trabajo realizado en el proyecto, posibles mejoras y ampliaciones.
\item \textbf{Bibliografía y referencias}: En este capítulo se indican todas las fuentes de información consultadas para realizar el proyecto.
\item \textbf{Apéndices}: Se incluyen como apéndices un listado de software utilizado en la elaboración del proyecto, el manual de instalación y uso de la biblioteca, el manual de referencia completa y el texto de la licencia bajo el que se libera este documento, que es la GFDLv1.3.
\end{itemize}

