% Este archivo es parte de la memoria de libWiiEsp, protegida bajo la 
% licencia GFDL. Puede encontrar una copia de la licencia en el archivo fdl-1.3.tex

% Fuente tomada de la plantilla LaTeX para la realización de Proyectos Final 
% de Carrera de Pablo Recio Quijano.

% Copyright (C) 2009 Pablo Recio Quijano
% Copyright (C) 2011 Ezequiel Vázquez de la Calle

% -*-video.tex-*-

El primer paso para abordar el desarrollo de \programa{LibWiiEsp} fue investigar sobre la forma de trabajar del sistema gráfico de la videoconsola. Como ya se ha comentado, la base del proyecto será \programa{libogc} \cite{website:libogc}, una biblioteca de bajo nivel que permite acceder a prácticamente todo el hardware de Nintendo Wii.

\subsection{Módulos \emph{video} y \emph{GX}}

\programa{Libogc} trabaja gráficamente con dos módulos, llamados \programa{video} y \programa{GX}. El módulo \programa{video} es el que controla las funciones básicas del chip gráfico de la consola, encargándose de detectar el modo de vídeo, enviar los datos que se quieren dibujar al bus de la \emph{GPU} y esperar la sincronización vertical. Este componente requiere para trabajar un \emph{búffer} de información situado en la memoria principal, y que debe almacenar, en cada fotograma de un programa, los datos que se quieren dibujar en la pantalla.\\

Sobre el módulo \programa{video} que controla directamente el hardware, opera el otro componente, la librería \programa{GX}, la cual ofrece muchas más posibilidades de trabajo. En concreto, proporciona una serie de estructuras y funciones que facilitan enormemente el trabajo con el sistema de video, siendo las más interesantes para nuestro objetivo las siguientes:

\begin{itemize}
\item \textbf{Dibujo de primitivas}: la \programa{GX} trabaja con una serie de primitivas de dibujo, que son polígonos que se pueden dibujar en el \emph{búffer} del sistema básico de vídeo. Cada uno de estos polígonos se dibuja a partir de un número concreto de vértices; por ejemplo, para dibujar un triángulo son necesarios tres vértices, para una línea dos, y así. Las primitivas más útiles para nosotros serán el punto (\emph{GX\_POINTS}), la línea recta (\emph{GX\_LINES}), el triángulo (\emph{GX\_TRIANGLES}) y el rectángulo (\emph{GX\_QUADS}).
\item \textbf{Trabajo con texturas en \emph{crudo} y paletas}: una textura no es más que una imagen preparada para ser dibujada dentro de un polígono que tenga área (en el caso de las primitivas antes mencionadas, triángulos y rectángulos). Esta imagen puede tener su información de forma explícita (color directo) o implícita (cada píxel es una referencia a un color de la paleta de colores del sistema). \programa{GX} proporciona las estructuras de datos necesarias para almacenar la información de los colores, texturas y paletas, aunque el trabajo directo con ellas es un tanto engorroso.
\item \textbf{Formatos de texturas}: se soportan varios formatos de textura, cada uno de los cuales tiene sus ventajas e inconvenientes, pero hay uno de ellos con el que el procesador de Wii trabaja de forma nativa: este formato es RGB5A3, que utiliza 16 bits para cada píxel y da la posibilidad de trabajar con el canal \emph{alpha} para las transparencias. Concretamente, RGB5A3 representa un píxel con 5 bits para cada canal de color y uno para \emph{alpha}, siendo determinada la componente del rojo por los 5 bits de mayor peso, el verde por los 5 siguientes y el azul por los 5 bits, dejando el de menor peso para indicar el canal \emph{alpha}. Sin embargo, cuando el bit que identifica al canal \emph{alpha} tiene el valor 1 (\emph{alpha} activado), se utilizan 4 bits para cada canal, siendo los cuatro de menor peso los que determinan el nivel de transparencia (es decir, la distribución de los bits con el canal \emph{alpha} activado sería RRRRGGGGBBBBAAAA (\emph{red} o rojo, \emph{green} o verde, \emph{blue} o azul, \emph{alpha}), con el rojo situado en los cuatros bits de mayor peso). Este sistema, además de ser el más eficiente para trabajar con la \emph{GPU} de Nintendo Wii, proporciona una flexibilidad enorme respecto al trabajo con transparencias. Por otra parte, se observa que la construcción de texturas se realiza a partir de una imagen organizadas en \emph{tiles} de 4x4 píxeles.
\item \textbf{Sistema de descriptores}: antes de realizar cualquier operación de dibujo, \programa{GX} espera recibir una serie de parámetros en los que se indiquen qué se va a dibujar; estos parámetros reciben el nombre de descriptores. Entre otras muchas funciones, los descriptores sirven para indicar si se va a dibujar una primitiva con color directo o rellena con una textura, la proyección que se va a utilizar a la hora de visualizar la imagen en la pantalla y el orden y tipo de los parámetros de los vértices.
\item \textbf{Operaciones directas sobre la \emph{GPU}}: el resto de operaciones de interés son relativas a la finalización de escritura en el \emph{búffer} gráfico, el volcado de datos desde éste hacia el \emph{EFB} del chip gráfico de la videoconsola y la fijación de estos datos en este \emph{búffer} interno de la \emph{GPU}, operación que evita el posible \emph{machacamiento} de información en caso de sobrecarga del sistema gráfico.
\end{itemize}

Un detalle más, \emph{GX} utiliza como formato para el color una variable entera de 32 bits sin signos, donde cada 8 bits representan la componente de un canal de la imagen. La distribución hexadecimal es 0xRRGGBBAA, siendo RR el byte que corresponde al color rojo, GG los 8 bits para el verde, BB la componente azul y AA el canal \emph{alpha}.

\subsection{Identificación de funcionalidad necesaria}

Una vez reunida toda esta información, y teniendo claro qué estructuras y funciones de \emph{GX} y \emph{video} nos serán útiles, procedemos a identificar qué queremos conseguir para trabajar de una forma cómoda con el sistema gráfico de la consola.\\

Lo primero que tenemos claro es que se necesita controlar la inicialización del sistema gráfico de Wii, preparando ambos módulos (\programa{video} y \programa{GX}) para que trabajen de acuerdo a nuestras necesidades, pero abstrayéndonos de todos los mecanismos implicados en este proceso. También es necesario un mecanismo que permita dar por finalizado un fotograma y enviarlo a la \emph{GPU} para que procese la información de éste y la dibuje en la pantalla, y que además enviara nueva información al \emph{EFB} únicamente si existieran cambios en los gráficos a dibujar (de este modo, se consigue evitar la sobrecarga innecesaria del chip gráfico). Sería conveniente utilizar un sistema de doble \emph{búffer}, es decir, alternar entre dos flujos de datos la hora de dibujar los gráficos y pasarlos al \emph{EFB}, de tal manera que evitaríamos posibles problemas de parpadeos al dibujar en la pantalla.\\

Por otro lado, nos interesa realizar el procesamiento necesario para crear una textura a partir de la información de una imagen que se encuentre en memoria, organizando los bits que componen dicha imagen en \emph{tiles} de 4x4, y también sería adecuado abstraer al usuario de los mecanismos implicados en el trabajo con los descriptores de \programa{GX}.\\

Como tercer y último bloque de funcionalidad, se encuentra el objetivo de dibujar en la pantalla texturas previamente procesadas, tanto de forma completa como únicamente una parte de ellas. Además, se dará la oportunidad de dibujar las siguientes formas geométricas rellenas de un color plano: un punto (o píxel), una recta con anchura, un rectángulo determinado por cuatro puntos y un círculo.\\

Por supuesto, todas las operaciones aquí descritas deberán ser eficientes, ya que no interesa sobrecargar el procesador gráfico de la videoconsola.

\subsection{Diseño e implementación}

Como resultado de toda la información concretada en los dos puntos anteriores, se llega a la conclusión de que lo más apropiado para satisfacer el requisito \emph{Controlar el sistema de vídeo} es construir una clase que implemente el patrón \emph{Singleton}, es decir, que únicamente exista una instancia de esta clase en el sistema, y que sea accesible desde cualquier punto de la aplicación. La decisión sobre la utilización de este patrón de diseño se basó, principalmente, en que la Nintendo Wii trabaja con sólo una pantalla, y por tanto no tendría sentido que coexistieran en memoria varios objetos de la clase.\\

En esta clase, debería existir un método que se encargue de inicializar los módulos de \programa{libogc} descritos con anterioridad. El proceso de inicialización debe establecer el modo de vídeo, crear los dos flujos de datos que conformarían el sistema de doble \emph{búffer}, configurarlos para que trabajen con el modo de vídeo detectado, e indicar a la \emph{GPU} dicho modo de vídeo. A continuación, debe reservarse una zona de memoria, alineada a 32 bytes, que se utilizará para copiar el contenido del búffer activo en cada fotograma al \emph{EFB}. En la figura \ref{procesargraficos} puede observarse un esquema de cómo queda la organización de los flujos de datos del sistema de doble \emph{búffer}.\\

\figura{procesargraficos.png}{scale=0.5}{Esquema que ilustra el funcionamiento del sistema gráfico de Wii}{procesargraficos}{h}

Justo después se inicializará la librería \programa{GX}, aportándole los parámetros de configuración necesarios para establecer una proyección ortográfica, calcular la resolución de pantalla, el tratamiento de color y el canal \emph{alpha}, y otros detalles más.\\

Para evitar que se envíe al chip gráfico información que no cambia de un fotograma al siguiente (y por tanto, conseguir una mayor eficiencia al procesar los gráficos únicamente cuando hay cambios en el \emph{EFB}) nuestra clase necesita una bandera interna, representada como una variable de tipo booleano, que indique si hay cambios respecto al estado anterior del \emph{EFB}. Esta bandera se activará cuando se utilice cualquier método de dibujo, y se desactivará en el método que se encarga de enviar los datos a la \emph{GPU}.\\

Una vez implementados los métodos de inicialización del sistema gráfico y de finalización de fotograma, pasamos al siguiente punto, el trabajo con texturas. Se necesita un método que, a partir de una zona de memoria que contenga información de una imagen en formato RGB5A3, se encargue de crear una textura organizada en \emph{tiles} de 4x4 e inicialice una variable que contenga la estructura que necesita \emph{GX} para trabajar con ella.\\

A continuación, se escriben los métodos privados de la clase que abstraen el trabajo de los descriptores, indicando uno de ellos los descriptores necesarios para utilizar color directo, y el otro para utilizar una textura previamente procesada.\\

Por último, se crean los métodos de dibujo que necesitamos para cubrir la funcionalidad identificada. El dibujo de una textura es trivial, ya que consiste en dibujar un rectángulo (a partir de cuatro puntos) relleno con la textura, lo cual se consigue ajustando los descriptores. El punto se consigue dibujando un píxel de color, la recta con anchura, con dos triángulos rectángulos que tengan la hipotenusa común, y el rectángulo relleno de color es parecido a dibujar una textura, pero ajustando los descriptores de \programa{GX} para que en lugar de utilizar ésta, se rellene con un color liso. Para dibujar parte de una textura, se delimita dicha parte con un rectángulo indicado por un punto sobre la textura, y un ancho y un alto en píxeles, siendo las coordenadas del punto relativas a la esquina superior izquierda de la textura original. Por último, para dibujar un círculo relleno de color, se utilizan 32 triángulos que comparten un vértice (que coincide con el centro del círculo).\\

La interfaz pública de la clase puede observarse en la figura \ref{screen}.\\

\figura{screen.png}{scale=0.6}{Interfaz pública de la clase Screen}{screen}{h}

\subsection{Pruebas}

La batería de pruebas diseñada para esta clase en particular consta de dos grupos de comprobaciones: por un lado, había que probar que el sistema gráfico se inicializa, trabaja correctamente con el sistema de doble \emph{búffer}, se crean bien las texturas a partir de una imagen cargada en memoria con formato RGB5A3 y que se finalizan adecuadamente los fotogramas. El otro conjunto de pruebas iba dirigido a confirmar que los métodos de dibujo realizan bien su trabajo.\\

La inicialización del sistema gráfico, el trabajo con el doble \emph{búffer} y la finalización de fotogramas se comprobaron utilizando el subsistema de consola que incorpora \programa{libogc}, imprimiendo mensajes en la pantalla a medida que se completaban las operaciones.\\

Por otro lado, los métodos de dibujo se probaron directamente, una vez validada la funcionalidad descrita en el párrafo anterior. Sin embargo, cabe destacar que, al no disponer aún de acceso a la tarjeta SD para cargar imágenes, las pruebas de creación de texturas y dibujo de las mismas tuvieron que realizarse con la utilidad \programa{raw2c} (incluida en \programa{DevKitPPC} \cite{website:devkitpro}), un pequeño software que transforma cualquier archivo binario (imágenes, efectos de sonido, etc.) en un archivo de cabecera de C con toda su información en forma de \emph{array} de datos. De esta manera, y a partir de una imagen BMP sin compresión, se diseñó un pequeño programa que creaba la textura a partir del flujo de datos generado por \programa{raw2c} y la dibujaba en la pantalla, tanto completa como una parte de ella.

