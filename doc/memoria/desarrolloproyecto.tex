% Este archivo es parte de la memoria de libWiiEsp, protegida bajo la 
% licencia GFDL. Puede encontrar una copia de la licencia en el archivo fdl-1.3.tex

% Fuente tomada de la plantilla LaTeX para la realización de Proyectos Final 
% de Carrera de Pablo Recio Quijano.

% Copyright (C) 2009 Pablo Recio Quijano
% Copyright (C) 2011 Ezequiel Vázquez de la Calle

% -*-descripcionproyecto.tex-*-

\programa{LibWiiEsp} ha sido construida siguiendo una metodología iterativa basada en análisis, diseño, implementación y pruebas para cada requisito, consiguiéndose tras cada iteración un módulo totalmente funcional. En este capítulo se desarrolla todo el proceso separado en fases, abordando en cada una de ellas el proceso completo realizado para cubrir cada uno de los requisitos indicados en la descripción general del proyecto.\\

Cada iteración, o fase, consta de los cuatro puntos principales ya mencionados, donde en el análisis se describe cómo trabaja \programa{libogc} a bajo nivel y qué es lo que se pretende conseguir, el diseño profundiza en el cómo se ha conseguido cubrir el requisito, en el apartado de implementación se indican los detalles más relevantes sobre la construcción del módulo, y en el epígrafe de pruebas se recopilan las diversas pruebas realizadas a cada módulo, tanto individuales, como de cohesión con los demás módulos anteriores.\\

A pesar de que en este capítulo se mencionan las pruebas individuales aplicadas a cada uno de los módulos de \programa{LibWiiEsp}, en el siguiente se recoge la documentación completa sobre el plan de pruebas de la biblioteca. Para ampliar información sobre el funcionamiento y los parámetros de las clases que componen \programa{LibWiiEsp}, consultar los manuales de la biblioteca.\\

Debido a la falta de material de consulta en forma de bibliografía, toda la información sobre el funcionamiento de Nintendo Wii ha sido recopilada a partir de un buen número de fuentes localizadas a través de la red \cite{website:tutorialhermes} \cite{website:wiibrew} \cite{website:scenebeta} \cite{website:gamedev}.

\section{Diagramas del sistema}

En la figura \ref{clases} se puede observar el diagrama de clases completo del sistema y en la figura \ref{componentes} el diagrama de componentes del sistema. Ambos diagramas representan el estado del sistema (a nivel lógico interno y a nivel físico de los archivos de cabeceras) tras finalizar el desarrollo.\\

\figura{clases.png}{scale=0.6,angle=90}{Diagrama de clases del sistema}{clases}{p}

\figura{componentes.png}{scale=0.6,angle=90}{Diagrama de componentes del sistema}{componentes}{p}

\section{Sistema de vídeo}
\label{labelvideo}
% Este archivo es parte de la memoria de libWiiEsp, protegida bajo la 
% licencia GFDL. Puede encontrar una copia de la licencia en el archivo fdl-1.3.tex

% Fuente tomada de la plantilla LaTeX para la realización de Proyectos Final 
% de Carrera de Pablo Recio Quijano.

% Copyright (C) 2009 Pablo Recio Quijano
% Copyright (C) 2011 Ezequiel Vázquez de la Calle

% -*-video.tex-*-

El primer paso para abordar el desarrollo de \programa{LibWiiEsp} fue investigar sobre la forma de trabajar del sistema gráfico de la videoconsola. Como ya se ha comentado, la base del proyecto será \programa{libogc} \cite{website:libogc}, una biblioteca de bajo nivel que permite acceder a prácticamente todo el hardware de Nintendo Wii.

\subsection{Módulos \emph{video} y \emph{GX}}

\programa{Libogc} trabaja gráficamente con dos módulos, llamados \programa{video} y \programa{GX}. El módulo \programa{video} es el que controla las funciones básicas del chip gráfico de la consola, encargándose de detectar el modo de vídeo, enviar los datos que se quieren dibujar al bus de la \emph{GPU} y esperar la sincronización vertical. Este componente requiere para trabajar un \emph{búffer} de información situado en la memoria principal, y que debe almacenar, en cada fotograma de un programa, los datos que se quieren dibujar en la pantalla.\\

Sobre el módulo \programa{video} que controla directamente el hardware, opera el otro componente, la librería \programa{GX}, la cual ofrece muchas más posibilidades de trabajo. En concreto, proporciona una serie de estructuras y funciones que facilitan enormemente el trabajo con el sistema de video, siendo las más interesantes para nuestro objetivo las siguientes:

\begin{itemize}
\item \textbf{Dibujo de primitivas}: la \programa{GX} trabaja con una serie de primitivas de dibujo, que son polígonos que se pueden dibujar en el \emph{búffer} del sistema básico de vídeo. Cada uno de estos polígonos se dibuja a partir de un número concreto de vértices; por ejemplo, para dibujar un triángulo son necesarios tres vértices, para una línea dos, y así. Las primitivas más útiles para nosotros serán el punto (\emph{GX\_POINTS}), la línea recta (\emph{GX\_LINES}), el triángulo (\emph{GX\_TRIANGLES}) y el rectángulo (\emph{GX\_QUADS}).
\item \textbf{Trabajo con texturas en \emph{crudo} y paletas}: una textura no es más que una imagen preparada para ser dibujada dentro de un polígono que tenga área (en el caso de las primitivas antes mencionadas, triángulos y rectángulos). Esta imagen puede tener su información de forma explícita (color directo) o implícita (cada píxel es una referencia a un color de la paleta de colores del sistema). \programa{GX} proporciona las estructuras de datos necesarias para almacenar la información de los colores, texturas y paletas, aunque el trabajo directo con ellas es un tanto engorroso.
\item \textbf{Formatos de texturas}: se soportan varios formatos de textura, cada uno de los cuales tiene sus ventajas e inconvenientes, pero hay uno de ellos con el que el procesador de Wii trabaja de forma nativa: este formato es RGB5A3, que utiliza 16 bits para cada píxel y da la posibilidad de trabajar con el canal \emph{alpha} para las transparencias. Concretamente, RGB5A3 representa un píxel con 5 bits para cada canal de color y uno para \emph{alpha}, siendo determinada la componente del rojo por los 5 bits de mayor peso, el verde por los 5 siguientes y el azul por los 5 bits, dejando el de menor peso para indicar el canal \emph{alpha}. Sin embargo, cuando el bit que identifica al canal \emph{alpha} tiene el valor 1 (\emph{alpha} activado), se utilizan 4 bits para cada canal, siendo los cuatro de menor peso los que determinan el nivel de transparencia (es decir, la distribución de los bits con el canal \emph{alpha} activado sería RRRRGGGGBBBBAAAA (\emph{red} o rojo, \emph{green} o verde, \emph{blue} o azul, \emph{alpha}), con el rojo situado en los cuatros bits de mayor peso). Este sistema, además de ser el más eficiente para trabajar con la \emph{GPU} de Nintendo Wii, proporciona una flexibilidad enorme respecto al trabajo con transparencias. Por otra parte, se observa que la construcción de texturas se realiza a partir de una imagen organizadas en \emph{tiles} de 4x4 píxeles.
\item \textbf{Sistema de descriptores}: antes de realizar cualquier operación de dibujo, \programa{GX} espera recibir una serie de parámetros en los que se indiquen qué se va a dibujar; estos parámetros reciben el nombre de descriptores. Entre otras muchas funciones, los descriptores sirven para indicar si se va a dibujar una primitiva con color directo o rellena con una textura, la proyección que se va a utilizar a la hora de visualizar la imagen en la pantalla y el orden y tipo de los parámetros de los vértices.
\item \textbf{Operaciones directas sobre la \emph{GPU}}: el resto de operaciones de interés son relativas a la finalización de escritura en el \emph{búffer} gráfico, el volcado de datos desde éste hacia el \emph{EFB} del chip gráfico de la videoconsola y la fijación de estos datos en este \emph{búffer} interno de la \emph{GPU}, operación que evita el posible \emph{machacamiento} de información en caso de sobrecarga del sistema gráfico.
\end{itemize}

Un detalle más, \emph{GX} utiliza como formato para el color una variable entera de 32 bits sin signos, donde cada 8 bits representan la componente de un canal de la imagen. La distribución hexadecimal es 0xRRGGBBAA, siendo RR el byte que corresponde al color rojo, GG los 8 bits para el verde, BB la componente azul y AA el canal \emph{alpha}.

\subsection{Identificación de funcionalidad necesaria}

Una vez reunida toda esta información, y teniendo claro qué estructuras y funciones de \emph{GX} y \emph{video} nos serán útiles, procedemos a identificar qué queremos conseguir para trabajar de una forma cómoda con el sistema gráfico de la consola.\\

Lo primero que tenemos claro es que se necesita controlar la inicialización del sistema gráfico de Wii, preparando ambos módulos (\programa{video} y \programa{GX}) para que trabajen de acuerdo a nuestras necesidades, pero abstrayéndonos de todos los mecanismos implicados en este proceso. También es necesario un mecanismo que permita dar por finalizado un fotograma y enviarlo a la \emph{GPU} para que procese la información de éste y la dibuje en la pantalla, y que además enviara nueva información al \emph{EFB} únicamente si existieran cambios en los gráficos a dibujar (de este modo, se consigue evitar la sobrecarga innecesaria del chip gráfico). Sería conveniente utilizar un sistema de doble \emph{búffer}, es decir, alternar entre dos flujos de datos la hora de dibujar los gráficos y pasarlos al \emph{EFB}, de tal manera que evitaríamos posibles problemas de parpadeos al dibujar en la pantalla.\\

Por otro lado, nos interesa realizar el procesamiento necesario para crear una textura a partir de la información de una imagen que se encuentre en memoria, organizando los bits que componen dicha imagen en \emph{tiles} de 4x4, y también sería adecuado abstraer al usuario de los mecanismos implicados en el trabajo con los descriptores de \programa{GX}.\\

Como tercer y último bloque de funcionalidad, se encuentra el objetivo de dibujar en la pantalla texturas previamente procesadas, tanto de forma completa como únicamente una parte de ellas. Además, se dará la oportunidad de dibujar las siguientes formas geométricas rellenas de un color plano: un punto (o píxel), una recta con anchura, un rectángulo determinado por cuatro puntos y un círculo.\\

Por supuesto, todas las operaciones aquí descritas deberán ser eficientes, ya que no interesa sobrecargar el procesador gráfico de la videoconsola.

\subsection{Diseño e implementación}

Como resultado de toda la información concretada en los dos puntos anteriores, se llega a la conclusión de que lo más apropiado para satisfacer el requisito \emph{Controlar el sistema de vídeo} es construir una clase que implemente el patrón \emph{Singleton}, es decir, que únicamente exista una instancia de esta clase en el sistema, y que sea accesible desde cualquier punto de la aplicación. La decisión sobre la utilización de este patrón de diseño se basó, principalmente, en que la Nintendo Wii trabaja con sólo una pantalla, y por tanto no tendría sentido que coexistieran en memoria varios objetos de la clase.\\

En esta clase, debería existir un método que se encargue de inicializar los módulos de \programa{libogc} descritos con anterioridad. El proceso de inicialización debe establecer el modo de vídeo, crear los dos flujos de datos que conformarían el sistema de doble \emph{búffer}, configurarlos para que trabajen con el modo de vídeo detectado, e indicar a la \emph{GPU} dicho modo de vídeo. A continuación, debe reservarse una zona de memoria, alineada a 32 bytes, que se utilizará para copiar el contenido del búffer activo en cada fotograma al \emph{EFB}. En la figura \ref{procesargraficos} puede observarse un esquema de cómo queda la organización de los flujos de datos del sistema de doble \emph{búffer}.\\

\figura{procesargraficos.png}{scale=0.5}{Esquema que ilustra el funcionamiento del sistema gráfico de Wii}{procesargraficos}{h}

Justo después se inicializará la librería \programa{GX}, aportándole los parámetros de configuración necesarios para establecer una proyección ortográfica, calcular la resolución de pantalla, el tratamiento de color y el canal \emph{alpha}, y otros detalles más.\\

Para evitar que se envíe al chip gráfico información que no cambia de un fotograma al siguiente (y por tanto, conseguir una mayor eficiencia al procesar los gráficos únicamente cuando hay cambios en el \emph{EFB}) nuestra clase necesita una bandera interna, representada como una variable de tipo booleano, que indique si hay cambios respecto al estado anterior del \emph{EFB}. Esta bandera se activará cuando se utilice cualquier método de dibujo, y se desactivará en el método que se encarga de enviar los datos a la \emph{GPU}.\\

Una vez implementados los métodos de inicialización del sistema gráfico y de finalización de fotograma, pasamos al siguiente punto, el trabajo con texturas. Se necesita un método que, a partir de una zona de memoria que contenga información de una imagen en formato RGB5A3, se encargue de crear una textura organizada en \emph{tiles} de 4x4 e inicialice una variable que contenga la estructura que necesita \emph{GX} para trabajar con ella.\\

A continuación, se escriben los métodos privados de la clase que abstraen el trabajo de los descriptores, indicando uno de ellos los descriptores necesarios para utilizar color directo, y el otro para utilizar una textura previamente procesada.\\

Por último, se crean los métodos de dibujo que necesitamos para cubrir la funcionalidad identificada. El dibujo de una textura es trivial, ya que consiste en dibujar un rectángulo (a partir de cuatro puntos) relleno con la textura, lo cual se consigue ajustando los descriptores. El punto se consigue dibujando un píxel de color, la recta con anchura, con dos triángulos rectángulos que tengan la hipotenusa común, y el rectángulo relleno de color es parecido a dibujar una textura, pero ajustando los descriptores de \programa{GX} para que en lugar de utilizar ésta, se rellene con un color liso. Para dibujar parte de una textura, se delimita dicha parte con un rectángulo indicado por un punto sobre la textura, y un ancho y un alto en píxeles, siendo las coordenadas del punto relativas a la esquina superior izquierda de la textura original. Por último, para dibujar un círculo relleno de color, se utilizan 32 triángulos que comparten un vértice (que coincide con el centro del círculo).\\

La interfaz pública de la clase puede observarse en la figura \ref{screen}.\\

\figura{screen.png}{scale=0.6}{Interfaz pública de la clase Screen}{screen}{h}

\subsection{Pruebas}

La batería de pruebas diseñada para esta clase en particular consta de dos grupos de comprobaciones: por un lado, había que probar que el sistema gráfico se inicializa, trabaja correctamente con el sistema de doble \emph{búffer}, se crean bien las texturas a partir de una imagen cargada en memoria con formato RGB5A3 y que se finalizan adecuadamente los fotogramas. El otro conjunto de pruebas iba dirigido a confirmar que los métodos de dibujo realizan bien su trabajo.\\

La inicialización del sistema gráfico, el trabajo con el doble \emph{búffer} y la finalización de fotogramas se comprobaron utilizando el subsistema de consola que incorpora \programa{libogc}, imprimiendo mensajes en la pantalla a medida que se completaban las operaciones.\\

Por otro lado, los métodos de dibujo se probaron directamente, una vez validada la funcionalidad descrita en el párrafo anterior. Sin embargo, cabe destacar que, al no disponer aún de acceso a la tarjeta SD para cargar imágenes, las pruebas de creación de texturas y dibujo de las mismas tuvieron que realizarse con la utilidad \programa{raw2c} (incluida en \programa{DevKitPPC} \cite{website:devkitpro}), un pequeño software que transforma cualquier archivo binario (imágenes, efectos de sonido, etc.) en un archivo de cabecera de C con toda su información en forma de \emph{array} de datos. De esta manera, y a partir de una imagen BMP sin compresión, se diseñó un pequeño programa que creaba la textura a partir del flujo de datos generado por \programa{raw2c} y la dibujaba en la pantalla, tanto completa como una parte de ella.



\section{Acceso al lector de tarjetas}
\label{labeltarjeta}
% Este archivo es parte de la memoria de libWiiEsp, protegida bajo la 
% licencia GFDL. Puede encontrar una copia de la licencia en el archivo fdl-1.3.tex

% Fuente tomada de la plantilla LaTeX para la realización de Proyectos Final 
% de Carrera de Pablo Recio Quijano.

% Copyright (C) 2009 Pablo Recio Quijano
% Copyright (C) 2011 Ezequiel Vázquez de la Calle

% -*-tarjeta.tex-*-

Una vez cubierto el requisito de controlar el sistema de vídeo, era momento de investigar sobre el funcionamiento del lector de tarjetas de Nintendo Wii.

\subsection{Módulo \emph{wiisd\_io} y \programa{libFat}}

\programa{Libogc} proporciona acceso al dispositivo hardware del lector de tarjetas a través de la variable global \emph{\_\_io\_wiisd}, la cual está incluida en el fichero de cabecera \emph{sdcard/wiisd\_io.h}. Esta variable global es una instancia de la estructura DISC\_INTERFACE, incluida en la cabecera \emph{ogc/disc\_io.h}, y que se utiliza como interfaz única para todos los dispositivos de almacenamiento de los que dispone la videoconsola. La estructura está formada por dos variables que contienen información (indican el tipo de dispositivo) y otras seis, que son punteros a función. Estas seis funciones se definen para cada tipo de dispositivo, y son las siguientes:

\begin{itemize}
\item \textbf{startup}: inicializa el dispositivo.
\item \textbf{isInserted}: comprueba si el medio de almacenamiento está disponible en el lector correspondiente.
\item \textbf{readSectors}: lee un grupo de sectores desde el dispositivo a un flujo de datos.
\item \textbf{writeSectors}: escribe un flujo de datos en un grupo de sectores.
\item \textbf{clearStatus}: indica si el dispositivo está listo para realizar una operación.
\item \textbf{shutdown}: apaga el dispositivo.
\end{itemize}

Cuando el dispositivo esté listo para ser accedido, hay que montar la partición en la que se encuentre la información que se quiere cargar. Las tarjetas de memoria SD trabajan con el sistema de ficheros FAT o FAT32, por lo que es necesario utilizar \programa{libFat} adaptada para trabajar con Nintendo Wii. La interfaz de esta biblioteca es muy sencilla, ya que se puede montar un sistema de ficheros FAT con la función \emph{fatMountSimple}, que recibe como parámetros el nombre que queremos asignarle a la unidad montada, y la estructura DISC\_INTERFACE correspondiente al dispositivo (en el caso de la tarjeta SD será, como se comentó anteriormente, la variable \emph{\_\_io\_wiisd}). Desmontar una partición que esté montada es tan sencillo como llamar a la función \emph{fatUnmount}, indicándole el nombre de la unidad montada.\\

Hay que tener en cuenta una limitación de este sistema, y es que sólo es accesible la primera partición de la tarjeta SD; por tanto, se debe tener especial cuidado de que la tarjeta tenga su primera partición formateada con FAT o FAT32.

\subsection{Identificación de funcionalidad necesaria}

Tras revisar toda la documentación y las cabeceras de los archivos implicados, hay que definir de una forma clara qué se quiere conseguir para cumplir el requisito de acceder al lector de tarjetas. Básicamente, se necesita poder montar una partición FAT/FAT32 que se encuentre en la tarjeta SD, asegurando que quede accesible desde cualquier punto de la aplicación, y también poder desmontar esta partición.\\

Por otro lado, no estaría de más poder conocer en todo momento si la partición está montada o no, y el nombre asignado que tiene la unidad montada.

\subsection{Diseño e implementación}

Al igual que ocurre con el sistema gráfico de la videoconsola, lo más adecuado para tratar el acceso al lector de tarjetas es implementar una clase con el patrón \emph{Singleton}, ya que sólo existe en el hardware un lector de tarjetas (al contrario ocurre con los dos puertos USB traseros o las dos ranuras para tarjetas de memoria de \emph{Game Cube}, pero no es el caso que nos atañe). Por otro lado, la implementación del mencionado patrón debe dejar accesible la instancia de la clase desde cualquier punto del sistema.\\

La clase tendrá un método de inicialización, en el que reciba el nombre que se quiere asignar a la unidad una vez montada. Dicho método intentará inicializar el dispositivo del lector de tarjetas (utilizando la función \emph{startup} de la estructura descrita antes) y, en caso de tener éxito la operación, tratará de montar la primera partición que se encuentre en la tarjeta SD con un sistema de ficheros FAT. Si algo saliera mal, el proceso se repetirá hasta 10 veces. Superado ese número de intentos sin éxito, el programa interrumpiría su ejecución, saliendo con un código de error 1.\\

Además, la clase contará con métodos consultores sobre una bandera (variable booleana) que indique si la partición está montada o no, y sobre el nombre asignado a la unidad. Por último, se proporcionará también un método que intente desmontar la partición, en caso de que se encuentre activa.\\

Para acceder al sistema de ficheros montado mediante esta clase, basta con utilizar las funciones estándar de lectura y escritura de archivos. Es recomendable que siempre se empleen rutas absolutas, precedidas del nombre de la unidad seguida de los dos puntos (:). Por ejemplo, sabiendo que la unidad se llama "SD", para cargar un archivo cuya ruta sea \emph{/directorio/archivo.bmp} se podría utilizar el siguiente código:\\

\begin{lstlisting}[style=C++]
  ifstream archivo;
  archivo.open("SD:/directorio/archivo.bmp", ios::binary);
\end{lstlisting}

Por último, en la figura \ref{sdcard} puede observarse la interfaz pública de la clase:\\

\figura{sdcard.png}{scale=0.8}{Interfaz pública de la clase Sdcard}{sdcard}{h}

\subsection{Pruebas}

Las pruebas realizadas para validar esta clase fueron sencillas, ya que consistieron en un programa que utilizaba  el subsistema de consola de \programa{libogc} para ir imprimiendo mensajes a medida que se intentaba acceder al hardware del lector de tarjetas y, una vez activado, realizar el montaje y desmontaje de la partición. También se incluyó en el programa de prueba la apertura y escritura en un archivo de texto plano.



\section{Los mandos}
\label{labelmandos}
% Este archivo es parte de la memoria de libWiiEsp, protegida bajo la 
% licencia GFDL. Puede encontrar una copia de la licencia en el archivo fdl-1.3.tex

% Fuente tomada de la plantilla LaTeX para la realización de Proyectos Final 
% de Carrera de Pablo Recio Quijano.

% Copyright (C) 2009 Pablo Recio Quijano
% Copyright (C) 2011 Ezequiel Vázquez de la Calle

% -*-mandos.tex-*-

En la tercera iteración del proceso de desarrollo del proyecto se entra de lleno en la gestión de los dispositivos de entrada, es decir, los mandos. La mayor innovación de Nintendo Wii respecto a las demás videoconsolas fue precisamente su revolucionario sistema de control de los juegos, denominado \emph{Wii Remote} (o, coloquialmente, \emph{wiimote}), que permitían ir mucho más allá de participar en el videojuego apretando botones.

\subsection{Módulo \emph{Wpad}}

A la hora de gestionar los mandos de la consola, \programa{libogc} utiliza el módulo llamado \emph{Wpad}, que brinda acceso a las estructuras de control y las funciones dedicadas a leer el estado de los controles. Cada \emph{wiimote} que quiera utilizarse, como ya se comenta en el capítulo de descripción general de este documento, debe estar sincronizado permanentemente a la videoconsola.\\

\emph{Wpad} asigna a cada mando que esté conectado una estructura de tipo \emph{WPADData}, en la cual se encuentra toda la información relacionada con su estado en un momento dado. Cada \emph{wiimote} se identifica mediante un número entero, denominado \emph{chan}, que se genera automáticamente según el orden en el que se conectan los mandos (el primero tendrá \emph{chan} cero, el segundo \emph{chan} uno, y así).\\

La información sobre la pulsación de los botones se obtiene con variables de tipo entero de 32 bits sin signo, en la que cada dígito binario del entero representa el estado de un botón. Para saber si un botón concreto está pulsado basta con comparar (a nivel de bit, con un AND) la variable correspondiente con el valor binario único del botón (ver figura \ref{pulsacion}). Existen cuatro variables que indican si los botones acaban de ser pulsados, si se están manteniendo pulsados, si se acaban de soltar, o si no están pulsados. Por otro lado, hay variables también numéricas que indican el estado de la batería (menor carga cuanto menor es la cifra) o si el mando tiene alguna nueva información pendiente de recibir (normalmente, efectos de sonido para reproducir en el altavoz del \emph{wiimote}).\\

\figura{pulsacion.png}{scale=0.8}{Operación AND para conocer si hay pulsación de un botón}{pulsacion}{h}

Los últimos elementos que constituyen esta estructura son, a su vez, cinco estructuras más que almacenan los datos de orientación del mando (ángulos de viraje, cabeceo y rotación), de los acelerómetros, la fuerza con la que se mueve, las coordenadas en pantalla del puntero infrarrojo y la expansión conectada al \emph{wiimote}.\\

Respecto a los ángulos que determinan la orientación del mando, el cabeceo es el ángulo que forma éste respecto a un plano horizontal y perpendicular a la pantalla si se le hace girar sobre el eje X, el viraje es el ángulo formado por el \emph{wiimote} respecto a un plano vertical y perpendicular a la pantalla si se le hace girar sobre el eje Z, y la rotación es el ángulo que mide el giro de éste sobre el eje Y, como si estuviera dando una vuelta de campana. En la figura \ref{wiimote} puede apreciarse un esquema donde se describen gráficamente estos ángulos.\\

\figura{wiimote.png}{scale=0.6}{Ángulos de giro en la orientación de un \emph{Wii Remote}}{wiimote}{h}

En el módulo \emph{Wpad} se definen varias funciones que permiten trabajar con la estructura \emph{WPADData}, siendo la más importante de ellas \emph{WPAD\_ScanPads}, que lee el estado de todos los mandos conectados y actualiza la información de cada uno en su respectiva estructura. Por otro lado, existen funciones consultoras para conocer el estado de pulsación de los botones, pero no para el resto de información relativa a los \emph{wiimotes}, así pues, hay que acceder directamente a la estructura para conocer la orientación del mando, el puntero infrarrojo, etc.\\

El resto de funciones del módulo permite activar o desactivar la vibración, inicializar o desconectar el sistema de \emph{Bluetooth} con el que funcionan los mandos, desactivar uno de éstos conociendo su número identificador o enviar un flujo de datos con un efecto de sonido para reproducirlo en el altavoz del \emph{wiimote}.

\subsection{Identificación de funcionalidad necesaria}

A partir de toda la información recopilada en el epígrafe anterior, se debe decidir qué funcionalidad se quiere obtener para cubrir el requisito de \emph{Utilizar hasta cuatro mandos}.\\

Aunque sería muy interesante proporcionar control sobre todos los aspectos del \emph{wiimote}, existe un problema conocido en la lectura de los acelerómetros utilizando el módulo \emph{Wpad}, por lo que se descarta, de momento, esta funcionalidad. Igualmente, la utilización de la \emph{Wii Balance Board} (más conocida como la tabla de \emph{Wii Fit}), la guitarra de títulos como \emph{Guitar Hero} o \emph{Rock Band}, el \emph{Wii Motion Plus} y el mando clásico podrían aportar inmensas posibilidades a \programa{LibWiiEsp}, pero al no disponer de estos periféricos no puedo desarrollar un módulo completo para ello, por tanto, se descarta su uso. Sin embargo, sí puedo aportar la funcionalidad relativa a la extensión del \emph{wiimote} por excelencia, el \emph{Nunchuck}.\\

Recapitulando, el módulo de gestión de entrada de la biblioteca va a permitir conectar hasta cuatro \emph{Wii Remotes} simultáneos, cada uno con su respectiva expansión \emph{Nunchuck}. Se van a mapear todos los botones de ambos periféricos, para conocer tanto si se encuentran pulsados o no, si se acaban de presionar o si se acaban de soltar en un momento determinado.\\

Respecto a las características especiales de los mandos de la videoconsola, se ofrecerán métodos consultores sobre la orientación de los mandos y sobre la situación del puntero infrarrojo sobre la pantalla, quedando el acceso a la información de los acelerómetros pendiente para una futura versión (cuando una actualización del módulo \emph{Wpad} consiga que la lectura de dicha información sea correcta). Se controlará además el estado de la palanca del \emph{Nunchuck}, y la vibración del \emph{wiimote}.

\subsection{Diseño e implementación}

Una vez fijados los objetivos a conseguir, se crea una clase que gestione todos los aspectos relativos a la gestión de los \emph{wiimotes}. La clase debe tener un puntero a la estructura de tipo \emph{WPADData} del mando que tendrá asociado cada instancia, además de almacenar el \emph{chan} (número identificador único para el dispositivo), una variable que indique la expansión conectada y un diccionario en el que se guarde el mapeo de los botones, tanto del propio \emph{wiimote} como del \emph{Nunchuck}. El sistema de \emph{Bluetooth} bajo el que trabajan los mandos con la videoconsola tiene que inicializarse antes de poder usarse, así que se crea un método de clase que se encargue de ello.\\

La identificación de cada botón se realiza mediante un tipo enumerado que se utiliza como clave en el diccionario de botones. A cada una de estas claves se le asigna el valor binario de su botón correspondiente. Por otra parte, se crean dos \emph{arrays} de valores booleanos, del mismo tamaño que el diccionario, y que guardarán el estado actual de todos los botones (pulsado o no), y el estado anterior. De esta forma se podrá identificar si se acaba de pulsar o soltar un botón.\\

A la hora de actualizar el estado del \emph{wiimote}, se definen dos funciones. Una de ellas será un método de clase que realice la lectura y actualización de la estructura \emph{WPADData} de todos los controles conectados (no se puede hacer de forma individual), y la otra función obtendrá, a partir del contenido de la estructura actualizada, toda la información relativa al mando que esté asociado con la instancia concreta de la clase. En este último método se comprueba también el estado de pulsación de los botones, refrescando la información de los \emph{arrays} de datos booleanos.\\

Prácticamente la totalidad del resto de métodos son de consulta, tomando la información desde los \emph{arrays} de booleanos (los métodos que comprueban el estado de un botón concreto) o bien desde la estructura \emph{WPADData} (orientación, puntero infrarrojo, palanca del \emph{Nunchuck}, si el mando está conectado o no, si el \emph{Nunchuck} está conectado\ldots). El único método restante no observador es el que activa la vibración del \emph{wiimote} durante la cantidad de microsegundos que se le indique.\\

Un detalle a mencionar es la implementación de los métodos relacionados con el \emph{Nunchuck}, ya que se ofrece el valor crudo de los ejes de la palanca (son enteros sin signo, de 8 bits), donde el centro es, aproximadamente, un valor de 128. Pero ocurre que esto no es exacto, y por eso se proporcionan dos métodos (uno para cada eje) que indican si la palanca está pulsada hacia arriba o abajo, hacia la derecha o la izquierda, o si está en centrada en el eje.\\

Por último, mencionar que no se permite la copia ni asignación de una instancia, debido a que cada una está ligada a un único \emph{wiimote}, y no es posible asignar un mismo dispositivo a distintos objetos de la clase, ni utilizar una misma instancia para controlar más de un mando.\\

En la figura \ref{mando} se muestra la interfaz pública de la clase Mando.\\

\figura{mando.png}{scale=0.5}{Interfaz pública de la clase Mando}{mando}{h}

\subsection{Pruebas}

Comprobar la detección de pulsaciones de los botones fue sencillo, se hizo a partir de un programa que escribía en el sistema de consola que aporta \programa{libogc} un mensaje distinto según el botón que se pulsaba. Los métodos observadores se validaron imprimiendo directamente en la pantalla los valores que devolvían, y comprobando que fueran correctos. En cambio, para probar los métodos observadores para el estado del puntero infrarrojo sobre la pantalla, se utilizó el método \emph{dibujarPunto} de la clase \emph{Screen} con las coordenadas que devolvían las mencionadas funciones.



\section{Recursos multimedia}
\label{labelrecursos}
% Este archivo es parte de la memoria de libWiiEsp, protegida bajo la 
% licencia GFDL. Puede encontrar una copia de la licencia en el archivo fdl-1.3.tex

% Fuente tomada de la plantilla LaTeX para la realización de Proyectos Final 
% de Carrera de Pablo Recio Quijano.

% Copyright (C) 2009 Pablo Recio Quijano
% Copyright (C) 2011 Ezequiel Vázquez de la Calle

% -*-recursos.tex-*-

Tras obtener la funcionalidad necesaria para gestionar el sistema gráfico de Nintendo Wii, el lector de tarjetas y el soporte para hasta cuatro mandos, es el momento de trabajar en la carga y funcionamiento de los recursos multimedia. En este punto es especialmente necesario que se optimice al máximo, ya que de no hacerlo podemos provocar que los tiempos de carga se alarguen demasiado.\\

Tal y como ya se ha comentado, \programa{LibWiiEsp} ofrece soporte para utilizar imágenes, efectos de sonido, pistas de música y fuentes de texto.

\subsection{Módulos \emph{asndlib}, \emph{libmad} y biblioteca \emph{FreeType2}}

Como primer detalle a tener en cuenta, sabemos que la clase \emph{Sdcard} nos permite acceder a cualquier archivo que se encuentre en la tarjeta SD insertada en la videoconsola mediante, por ejemplo, un flujo de datos de entrada de ficheros (\emph{ifstream}). Otra cuestión ya mencionada, pero muy importante, es que toda reserva de memoria que se realice para almacenar un flujo de bytes debe estar alineado a 32 bytes y tener un tamaño múltiplo exacto de 32 bytes (consultar el capítulo 2 para más información).\\

En la primera iteración del proceso de desarrollo, la que cubre la construcción de la clase \emph{Screen}, se detalla todo lo relacionado con \emph{GX}, el módulo de \emph{libogc} que se encarga del sistema gráfico. Gracias a los avances conseguidos en aquel punto, ahora disponemos de un método que, a partir de una imagen almacenada en un flujo de datos, con sus bits organizados en RGB5A3 (formato de vídeo nativo de Nintendo Wii), crea un objeto de textura de tipo \emph{GXTexObj}, con el que trabaja \emph{GX}. Por otra parte, también disponemos en esta clase de dos funciones que nos permiten dibujar en la pantalla una textura de este tipo al completo, o una parte de ella. Respecto al formato de archivo de la imagen, se necesita, para una primera versión, un formato de archivo de imagen que no tenga compresión y que tenga la información de color de forma directa, sin utilizar paletas de color. El formato adecuado según estas premisas sería el mapa de bits (BMP) de 24 bits de color por píxel (8 bits por componente en cada píxel). Un detalle importante es la limitación que sufre el formato \emph{GXTexObj}, que consiste en que las medidas de toda imagen que vaya a ser transformada en textura deben ser múltiplo de 8 (tanto el ancho como el alto en píxeles); en caso de no respetarse, se produciría un error en el sistema.\\

Respecto al sonido, \programa{libogc} incorpora el módulo \emph{asndlib}, que es un conjunto de funciones que procesan flujos de bytes con efectos de sonido. \emph{Asndlib} utiliza un dispositivo hardware especial para mezclar el sonido, llamado DSP, y que soporta hasta 16 voces simultáneas (una voz es flujo de bytes que contiene un sonido). Nintendo Wii trabaja a 48000Hz de frecuencia, a 16 bits por \emph{sample} y en estéreo de forma nativa. Este módulo de gestión de sonido debe inicializarse antes de ser usado, y apagarse antes de finalizar la ejecución del programa. \emph{Asndlib} incorpora dos funciones de reproducción, una para reproducir una sola vez un efecto de sonido, y otra para reproducirlo infinitamente.\\

Las pistas de música son aún más fáciles de reproducir que los efectos de sonido, gracias a que \programa{libogc} incorpora el módulo \emph{libmad}. A partir de una zona de memoria que contenga un archivo de música en un formato MP3 válido, \emph{libmad} reproduce la pista de música en una voz reservada del DSP, y además abstrae de la necesidad de cambiar la forma de representación del fichero MP3, ya que se encarga de realizar el cambio desde \emph{Little Endian} a \emph{Big Endian} si fuera necesario.\\

En lo que concierne a la escritura en pantalla utilizando fuentes de texto, \programa{libogc} no ofrece ningún módulo para trabajar con ellas, pero existe una adaptación de la biblioteca \emph{FreeType2} para operar con Nintendo Wii, y que abstrae también del cambio de \emph{Endian} al cargar las fuentes y de la reserva de memoria alineada y con tamaño múltiplo de 32 bytes. Una vez cargada una fuente con \emph{FreeType2}, se puede extraer un carácter con una sencilla función que devuelve una imagen bitmap con dicho carácter dibujado. La biblioteca permite extraer la imagen del carácter en formato monocromo, es decir, en blanco y negro puro; por tanto, para dibujar el carácter bastaría con recorrer píxel a píxel esta imagen, dibujando en la pantalla los píxeles que correspondan en el color seleccionado (utilizando el método \emph{dibujarPunto} de la clase \emph{Screen}). Como último punto de interés, \emph{FreeType2} soporta cualquier carácter de cualquier codificación, siempre que esté soportado en la fuente de texto que se utilice.

\subsection{Identificación de funcionalidad necesaria}

Con respecto a las texturas, se necesita cargar desde la SD una imagen de mapa de bits de 24 bits de profundidad de color directo y almacenarla en un flujo de bytes en formato RGB5A3. Después de eso, se necesita crear una textura que se pueda dibujar con los métodos de la clase \emph{Screen}, es decir, de tipo \emph{GXTexObj}. Hay que permitir que se indique un color transparente (es decir, que no se dibuje) para que las texturas dibujadas en pantalla no se dibujen únicamente con forma rectangular, y ofrecer métodos observadores para el ancho y el alto en píxeles de la imagen. Por supuesto, también es necesario un método que dibuje la textura en la pantalla.\\

Los efectos de sonido únicamente necesitan ser cargados desde la tarjeta SD a un flujo de bytes, y poder ser reproducidos en cualquier momento. Como son sonidos estéreo (con dos canales), se debería permitir controlar el volumen de cada canal de forma independiente, para que así el usuario pueda conseguir efectos de sonido 3D.\\

Una pista de música debe ser cargada desde la tarjeta SD y almacenada en una zona de memoria, se debe poder reproducir, detener, y mantener en repetición si así se desea, así como controlar su volumen de reproducción.\\

Para trabajar con las fuentes de texto, primero hay que cargarlas desde la tarjeta SD y almacenarlas. Después de eso, hay que proporcionar al menos un método que permita escribir una cadena de caracteres, independientemente del juego de caracteres que utilice. Para ello, se puede usar el tipo de C++ \emph{wstring}, que es idéntico a \emph{string} salvo por el tipo de carácter base que utiliza, \emph{wchar\_t}, que permite trabajar con caracteres de hasta 32 bits.\\

Por supuesto, los cuatro recursos multimedia deben respetar las condiciones de utilizar alineación a 32 bytes cuando se reserve memoria, y se debe rellenar dichas zonas de memoria para que su tamaño sea múltiplo de 32 bytes. Además, deben cuidarse todas las situaciones que puedan derivar en error, comprobando las condiciones necesarias antes de cada operación.

\subsection{Diseño e implementación}

Con toda la información recopilada, procedemos a crear cuatro clases, una para representar cada uno de los recursos multimedia presentados. A continuación se aporta una descripción de cómo cumple los objetivos marcados cada una de ellas.\\

La clase \textbf{Imagen} es la que se encarga de proporcionar la funcionalidad necesaria para dibujar texturas en la pantalla. Dispone de un atributo de clase público que indica el color considerado como transparente, que es compartido por todas las instancias de la clase, y que al cargar la imagen corresponde con un píxel con el canal \emph{alpha} activado (es decir, un píxel que en la imagen tenga el mismo color que el transparente será un píxel invisible en la textura). Se ofrecen métodos consultores para el ancho y el alto en píxeles de la imagen, así como a la textura \emph{GXTexObj} creada a partir de ella.\\

En la primera versión de \programa{LibWiiEsp} sólo se dará soporte a imágenes de mapa de bits de 24 bits de profundidad de color directo, como ya se mencionó antes. Como se pretende que sea sencillo implementar soporte para un abanico amplio de formatos de imagen, el constructor de la clase Imagen únicamente inicializará las variables internas de la instancia creada. Para cargar una imagen se utilizará el método \emph{cargarXXX}, donde XXX corresponde con la extensión de la imagen en la tarjeta SD; así, para cargar \emph{archivo.bmp} se implementa el método \emph{cargarBmp}. Con esto se consigue que si en un futuro se quisiera implementar la carga de archivos PNG, bastaría con crear el método \emph{cargarPng}.\\

Para trabajar con archivos BMP son necesarias dos estructuras, que corresponden con las dos cabeceras de todo archivo que siga este formato: la primera cabecera ocupa 14 bytes, e indica los detalles del archivo; la segunda son 40 bytes, y contiene información relativa a la imagen en sí.\\

Entrando en la implementación del método de carga de mapas de bits, el proceso que sigue es sencillo: primero se comprueba que la tarjeta SD está montada y accesible, y se lee el archivo BMP a un flujo de entrada de ficheros. Después, se lee la primera cabecera y se comprueba que el archivo sigue efectivamente un formato de mapa de bits. Seguidamente, se lee la segunda cabecera, y de ahí se extraen los datos de ancho y alto de la imagen (recordemos, deben ser ambos múltiplo de 8) y la profundidad del color (que, de momento, sólo será válido si es 24 bits). A continuación, se recorre el resto del archivo con un bucle, leyendo en cada iteración 3 bytes (que corresponden con un píxel, siguiendo el formato BBGGRR, es decir, componente azul de 8 bits, después verde y por último rojo). La información de cada píxel se transforma al formato RGB5A3, haciendo que si el píxel coincide con el color invisible, tenga el canal \emph{alpha} activo al máximo (transparencia total). Como detalle, una imagen BMP se lee desde abajo a la izquierda, hacia la derecha y hacia arriba, pero las texturas \emph{GXTexObj} necesitan la información desde la esquina superior izquierda, hacia la derecha y hacia abajo. Una vez leída la imagen al flujo de bytes, se crea la textura utilizando el método correspondiente de la clase \emph{Screen} y se da por finalizado el proceso.\\

Para dibujar la textura, se proporciona un método que llama a \emph{dibujarTextura} de \emph{Screen}, pasándole como parámetro la textura asociada con la imagen. La interfaz pública de la clase se muestra en la figura \ref{imagen}.\\

\figura{imagen.png}{scale=0.6}{Interfaz pública de la clase Imagen}{imagen}{h}

Respecto a la clase \textbf{Sonido}, se proporciona un método de clase que inicializa el módulo \emph{asndlib}, y en el constructor se comprueba que la tarjeta SD está accesible y se lee el archivo de sonido (que debe tener \emph{samples} de 16 bits con signo, estéreo, una frecuencia de 48000 Hz y representación \emph{Big Endian}, formato que puede obtenerse procesando el archivo con SoX \cite{website:sox}) a una zona de memoria alineada. Se ofrecen dos métodos para controlar el volumen de los dos canales del sonido por separado, y otro para reproducir el efecto. En la figura \ref{sonido} puede observarse la interfaz pública de la clase.\\

\figura{sonido.png}{scale=0.6}{Interfaz pública de la clase Sonido}{sonido}{h}

Las pistas de música funcionan con la clase \textbf{Musica}, que tiene una implementación parecida a la anterior, solo que también incorpora un método que detiene la reproducción, y otro que permite mantener la reproducción en bucle (el método comienza la reproducción de la pista si ésta ya hubiera terminado). Se muestra la interfaz pública de la clase en la figura \ref{musica}.\\

\figura{musica.png}{scale=0.6}{Interfaz pública de la clase Musica}{musica}{h}

La clase \textbf{Fuente}, por último, utiliza el método de carga del archivo de fuentes que proporciona la adaptación de \emph{FreeType2}, de tal manera que no hay que preocuparse por la memoria alineada y el \emph{Endian}. Se aportan dos métodos de escritura en pantalla, uno recibe una cadena de caracteres \emph{string}, y el otro una \emph{wstring}, de tal manera que se cubren todos los posibles caracteres (de hasta 32 bits). Para escribir el texto, se recorre la cadena recibida, cargando el bitmap asociado a cada carácter, y dibujándolo en la pantalla píxel a píxel en el color que se indique. También se ofrece un método de clase que inicializa la biblioteca \emph{FreeType2}, y que debe ser llamado antes de poder utilizarla. La interfaz pública de la clase se puede observar en la figura \ref{fuente}.\\

\figura{fuente.png}{scale=0.6}{Interfaz pública de la clase Fuente}{fuente}{h}

Todos los destructores de estas clases se encargan de liberar la memoria ocupada por todos los flujos de bytes y zonas de memoria reservada.

\subsection{Pruebas}

La batería de pruebas relativa a los recursos multimedia consistieron en un programa que cargaba una imagen desde un archivo, en distintos formatos (soportados o no), y trataba de dibujarla en pantalla. Las comprobaciones de las condiciones necesarias para la carga y utilización de imágenes BMP se cumplieron escrupulosamente, y sólo se dibujó en pantalla la que tenía el formato adecuado.\\

El sonido y la música se probaron en el mismo programa, cargando los archivos correspondientes y reproduciéndolos al pulsar un determinado botón del mando. Las fuentes de texto, por otra parte, se probaron escribiendo textos en la pantalla en distintos idiomas (inglés, español, ruso y japonés), resultando positivas todas las comprobaciones. Los caracteres no latinos se mostraban correctamente.



\section{Trabajar con XML}
\label{labelparser}
% Este archivo es parte de la memoria de libWiiEsp, protegida bajo la 
% licencia GFDL. Puede encontrar una copia de la licencia en el archivo fdl-1.3.tex

% Fuente tomada de la plantilla LaTeX para la realización de Proyectos Final 
% de Carrera de Pablo Recio Quijano.

% Copyright (C) 2009 Pablo Recio Quijano
% Copyright (C) 2011 Ezequiel Vázquez de la Calle

% -*-parser.tex-*-

En los requisitos sobre interfaces externas se hace referencia a la utilización de la tecnología XML para conseguir una separación efectiva entre código fuente y datos. En esta quinta fase del desarrollo se recoge información sobre la creación de un módulo que permita trabajar cómodamente con este formato de archivos.

\subsection{Biblioteca \emph{TinyXML}}

El primer paso para cubrir la funcionalidad deseada consiste en una investigación sobre las distintas herramientas ya existentes que permitan cargar información desde un archivo XML y, por supuesto, que sean compatibles con Nintendo Wii. Tras la lectura de diversas fuentes, se encuentra que la única herramienta disponible actualmente para procesar este tipo de ficheros en la videoconsola es \emph{TinyXML}.\\

Esta pequeña utilidad proporciona abstracción sobre la lectura del árbol formado por los nodos del documento XML, y además, ya viene adaptada para su funcionamiento bajo la plataforma de Nintendo, por lo que no hay que preocuparse sobre la alineación de datos ni del \emph{Endian}. Trabaja cargando el documento en memoria, y accediendo a los distintos nodos que lo componen mediante punteros. Da acceso a elementos, atributos y contenido de nodos, además de permitir una fácil navegación descendente (es decir, de padre a hijo), por lo que cubre las necesidades básicas para nuestros propósitos.\\

Al poderse cargar completamente el árbol XML con esta herramienta, podemos implementar fácilmente un \emph{parser} que siga el método de trabajo DOM, que funciona precisamente cargando el árbol completo en memoria y accediendo posteriormente a los elementos necesarios.

\subsection{Identificación de funcionalidad necesaria}

Para trabajar cómodamente con archivos XML, necesitamos una interfaz accesible desde cualquier punto del sistema, y que permita cargar un documento completo y acceder al nodo raíz. A partir de este elemento (o de cualquier otro que hayamos localizado previamente), debemos poder buscar otro elemento bajo él a partir de su valor, y sería muy interesante poder navegar entre los hermanos (elementos que se encuentran bajo el mismo padre en el árbol) del mismo valor. Por supuesto, también se requiere poder leer atributos de tipo cadena de caracteres, entero y decimal de coma flotante, y al texto contenido en un elemento.\\

Por último, también es interesante que todas estas operaciones se realicen eficientemente, para no cargar demasiado el sistema al trabajar con este tipo de archivos de datos.

\subsection{Diseño e implementación}

Para obtener la funcionalidad especificada en el epígrafe anterior, se decide crear una clase que implemente el patrón \emph{Singleton}, así podemos tener la certeza de que sólo habrá una instancia de esta herramienta en el sistema, ahorrando costes de memoria (ya que sólo se permite tener cargado un único árbol XML a la vez) y teniendo la utilidad siempre disponible en cualquier punto del sistema.\\

Se implementa un método que, a partir de la ruta absoluta de un archivo XML en la tarjeta SD, carga un árbol XML completo en memoria. Este método sobreescribe cualquier otro documento XML que estuviera previamente cargado, por lo que se consigue que no haya fugas de memoria en este aspecto. También se crean varios métodos consultores que proporcionan acceso al nodo raíz del documento, al valor y el contenido de un elemento dado, y a un atributo de un elemento a partir de su nombre y el elemento al que pertenece (existiendo métodos de consulta de atributos para cadenas de caracteres, números enteros y decimales de coma flotante).\\

Para facilitar la navegación por el árbol del documento, se crea un método que, dado un valor y un elemento, buscará y devolverá el primer elemento cuyo valor coincida con el indicado. Si no se le indica un elemento a partir del cual buscar, recorrerá todo el documento a partir del elemento raíz. Además, existe otro método que, a partir de un elemento, busca entre los elementos hermanos (que tienen el mismo nodo padre) por el siguiente elemento con el que comparta valor.\\

El recorrido del árbol XML del documento realizado por estos métodos de navegación se realiza en preorden, de ahí los criterios de ordenación de elementos (por ejemplo, "siguiente hermano" se refiere al elemento, hijo del mismo padre, que es posterior al dado en preorden). Comentar también que los posibles errores de inexistencia de un elemento con un valor dado o de un atributo solicitado están controlados haciendo uso de las comprobaciones pertinentes.\\

En la figura \ref{parser} puede observarse la interfaz pública de la clase Parser, que nace como resultado de esta fase de desarrollo.\\

\figura{parser.png}{scale=0.6}{Interfaz pública de la clase Parser}{parser}{h}

\subsection{Pruebas}

La batería de pruebas de esta clase consistió en la carga de varios documentos XML distintos, a partir de los cuales se realizaron varias lecturas y recorridos en los árboles, imprimiendo en el sistema de consola de \emph{libogc} los valores de cada elemento visitado, así como los atributos solicitados. Se probaron tanto casos que se esperaban correctos como situaciones en las que se solicita un atributo inexistente, la búsqueda de un elemento no válido o la obtención del valor o el contenido de elementos que no existen.



\section{Organización de los recursos multimedia}
\label{labelgaleria}
% Este archivo es parte de la memoria de libWiiEsp, protegida bajo la 
% licencia GFDL. Puede encontrar una copia de la licencia en el archivo fdl-1.3.tex

% Fuente tomada de la plantilla LaTeX para la realización de Proyectos Final 
% de Carrera de Pablo Recio Quijano.

% Copyright (C) 2009 Pablo Recio Quijano
% Copyright (C) 2011 Ezequiel Vázquez de la Calle

% -*-galeria.tex-*-

Tras construir un módulo que nos permita trabajar cómodamente con archivos de datos en formato XML y de disponer de cuatro clases para hacer uso de recursos multimedia en el sistema, llega el momento de gestionar eficazmente dichos recursos.

\subsection{Identificación de funcionalidad necesaria}

Se pretende desarrollar un módulo que gestione todos los recursos multimedia existentes en el sistema, teniéndolos localizados desde cualquier punto de éste, y clasificados por tipo. Cada recurso debe estar identificado por un código único dentro de su tipo. Además, para obtener una separación efectiva entre código fuente y datos, se contempla la posibilidad de indicar, mediante un listado en formato XML, la relación de todos los recursos multimedia asociados con su código identificador.

\subsection{Diseño e implementación}

Para satisfacer la funcionalidad arriba indicada, se procede a diseñar una clase que implemente el patrón \emph{Singleton}, ya que sólo necesitamos una instancia que gestione todos los recursos multimedia que se encuentren en el sistema. Además, la utilización de este patrón de diseño nos aporta la ventaja de que la instancia única de esta clase será accesible desde cualquier punto del sistema.\\

La clase dispondrá de un diccionario para cada tipo de recurso: imágenes, efectos de sonido, pistas de música y fuentes de texto. Cada entrada en el diccionario estará formada por una clave (una cadena de caracteres de tipo \emph{string}), que será única en el diccionario, y asociado a ella habrá un puntero al objeto que contenga al recurso.\\

La carga de todos los recursos se realizará desde un método de inicialización que no será el constructor, debido a la implementación del patrón \emph{Singleton}. Dicho método comprobará que la tarjeta SD está accesible, cargará un archivo XML cuya ruta absoluta en la tarjeta reciba por parámetro y recorrerá el documento XML, cargando todos los recursos multimedia que se indiquen en el archivo. Se espera que cada elemento hijo del nodo raíz del documento contenga la información de creación de un recurso, siendo identificado el tipo de recurso por el valor del elemento. A continuación se muestra un ejemplo válido del formato esperado para el documento XML:

\begin{lstlisting}[style=XML]
<?xml version="1.0" encoding="UTF-8"?>
<galeria>
  <imagen codigo="fondo" formato="bmp" ruta="/directorio/fondo.bmp" />
  <musica codigo="rock" volumen="128" ruta="/directorio/rock.mp3" />
  <sonido codigo="bang" volumen="255" ruta="/directorio/bang.pcm" />
  <fuente codigo="arial" ruta="/directorio/arial.ttf" />
</galeria>
\end{lstlisting}

El método inicializador se encargará de llamar al constructor de la clase del recurso, utilizando como parámetros los atributos que acompañan a cada elemento en el archivo XML, y almacenará la dirección del objeto creado junto a su código en el diccionario correspondiente.\\

Para acceder a los recursos, la clase proporciona un método observador para cada tipo de recurso (es decir, para cada diccionario). Estos métodos reciben el código de identificación del recurso que se desea utilizar, y devuelven una referencia constante a él, abstrayendo al usuario de la lógica de punteros. El destructor de la clase se encarga de destruir todos los objetos de recursos almacenados en los diccionarios y liberar la memoria ocupada por éstos.\\

En la figura \ref{galeria} se muestra la interfaz pública de la clase Galeria.\\

\figura{galeria.png}{scale=0.6}{Interfaz pública de la clase Galeria}{galeria}{h}

\subsection{Pruebas}

La clase Galeria se ha probado con varios ficheros XML, tanto en casos de documentos mal formados como válidos, cargando los recursos indicados y utilizándolos en un programa de prueba. También se probó el caso de intentar cargar un recurso que no existe en la tarjeta SD, y el de código identificador repetido.



\section{Soporte de internacionalización (i18n)}
\label{labelidiomas}
% Este archivo es parte de la memoria de libWiiEsp, protegida bajo la 
% licencia GFDL. Puede encontrar una copia de la licencia en el archivo fdl-1.3.tex

% Fuente tomada de la plantilla LaTeX para la realización de Proyectos Final 
% de Carrera de Pablo Recio Quijano.

% Copyright (C) 2009 Pablo Recio Quijano
% Copyright (C) 2011 Ezequiel Vázquez de la Calle

% -*-idiomas.tex-*-

Continuando con las posibilidades que nos ofrece el disponer de un módulo que nos ayude a trabajar cómodamente con el formato de datos XML, y dado que la clase \emph{Fuente} puede escribir en la pantalla cualquier carácter gracias a las cadenas de caracteres anchos (de tipo \emph{wstring}), vamos a aplicar una idea parecida a la de la clase \emph{Galeria}, pero enfocada a proporcionar un buen soporte para la internacionalización de los juegos desarrollados con \emph{LibWiiEsp}.

\subsection{Identificación de funcionalidad necesaria}

Lo que se quiere conseguir con este módulo es que, en lugar de escribir texto en la pantalla directamente desde el código fuente, se asigne una \emph{etiqueta} o código identificador al punto del programa en el que se quiere mostrar el texto. La idea es que, en tiempo de ejecución, se sustituya dicha etiqueta por el texto correspondiente a ella en el idioma que se indique.\\

Un idioma, pues, debe ser un conjunto de textos, escritos todos en la misma lengua, identificado cada texto por un código o nombre de etiqueta. Todos los idiomas deben tener las mismas etiquetas, solo que el valor de la misma etiqueta para distintos idiomas será la misma cadena de caracteres traducida al que corresponda.\\

Para permitir una ampliación sencilla y rápida de los idiomas disponibles en un videojuego (además de separar código fuente y datos), se necesita que todas las etiquetas se definan en un archivo externo que sea cargado en tiempo de ejecución. Lo ideal es utilizar la tecnología XML y el módulo de \programa{LibWiiEsp} que permite trabajar con ella.\\

Se requiere también que sea sencillo cambiar entre idiomas, para lo que parece adecuado tener un idioma marcado como activo en un momento dado, y que se permita conocer qué idioma se encuentra activo en un momento dado, la existencia o no de un idioma, seleccionarlo y alternar entre todos los idiomas disponibles de forma circular.\\

Por último, es necesario que el módulo que se encargue de gestionar el valor de las etiquetas de texto se encuentre disponible desde cualquier punto del sistema, y que exista únicamente una instancia en éste.

\subsection{Diseño e implementación}

Para cubrir la funcionalidad descrita se construye una clase que implementa el patrón \emph{Singleton}, ya que uno de los requisitos del módulo consiste en disponer de una instancia de la clase, y que sea accesible desde cualquier punto del sistema.\\

La estructura de datos diseñada para almacenar un idioma consiste en un diccionario, donde su primera componente es una cadena de caracteres de alto nivel (de tipo \emph{string}) que almacena el nombre o código de una etiqueta de texto (y mediante la cual se identifica la etiqueta), y su segunda componente es otra cadena de caracteres de alto nivel, pero de caracteres anchos (de tipo \emph{wstring}), en la cual se guarda el texto asociado a la etiqueta traducido al idioma correspondiente.\\

Cada idioma se almacena en otro diccionario en la segunda componente de éste, identificado por su nombre (cadena de caracteres anchos \emph{wstring}) en la primera componente del diccionario.\\

El método de inicialización del soporte de internacionalización requiere como parámetro una cadena de caracteres con la ruta absoluta en la tarjeta SD de un archivo XML que contenga la información de las etiquetas de texto para todos los idiomas que se utilizarán. Este archivo se carga en memoria y se recorre, creando una entrada en el diccionario de idiomas para cada uno que encuentre, y guardando cada etiqueta de texto asociada a él. A continuación se muestra un ejemplo de archivo de idiomas soportado:

\begin{lstlisting}[style=XML, texcl=true, escapechar=']
<?xml version="1.0" encoding="UTF-8"?>
<lang>
  <idioma nombre="espa'ñ'ol">
    <tag nombre="PUNTOS" valor="Puntuaci'ó'n" />
    <tag nombre="VIDAS" valor="Vidas" />
    <tag nombre="NIVEL" valor="Nivel" />
  </idioma>
  <idioma nombre="english">
    <tag nombre="PUNTOS" valor="Score" />
    <tag nombre="VIDAS" valor="Lives" />
    <tag nombre="NIVEL" valor="Level" />
  </idioma>
  <idioma nombre="fran'ç'ais">
    <tag nombre="PUNTOS" valor="Points" />
    <tag nombre="VIDAS" valor="Vies" />
    <tag nombre="NIVEL" valor="Niveau" />
  </idioma>
</lang>
\end{lstlisting}

Al leer el contenido de una etiqueta, éste se almacena temporalmente en una cadena de caracteres \emph{string}. Antes de guardarlo en el diccionario del idioma correspondiente, se transforma a \emph{wstring}, para así asegurar que los caracteres son los correctos. La explicación de esta transformación es sencilla: cuando se lee un texto con caracteres anchos (que necesitan un espacio mayor que 1 byte por carácter), cada carácter ancho se 'rompe' en trozos de 1 byte cada uno. Un carácter ASCII normal ocupará el espacio esperado, es decir, 1 byte. Durante la transformación se crea una cadena de caracteres anchos (de tipo \emph{wchar\_t}), y se rellena con el valor de la etiqueta después de pasarlo por la función de C llamada \emph{mbstowcs}. Esta función se encarga de detectar los caracteres anchos que están 'rotos' en trozos de 1 byte, los une, y devuelve toda la cadena recibida como una cadena de caracteres anchos. Por último, se almacena lo devuelto en una variable \emph{wstring}, y se introduce en el idioma correspondiente.\\

Se proporciona un método observador para conocer el nombre del idioma activo, otro que comprueba si un nombre de idioma introducido existe como clave en el diccionario que almacena los idiomas, y dos métodos para cambiar el idioma: uno que recibe directamente como parámetro el nombre del idioma a activar, y otro que alterna entre todos los idiomas registrados a través de una función que va activando cada uno según se lo encuentre al recorrer el diccionario. Este último método obtiene un iterador al idioma activo actual dentro del diccionario, y lo avanza una posición. Si no se ha llegado al final del diccionario, se activa el idioma encontrado; en caso contrario, se activa el primero del diccionario.\\

Por último, para obtener el texto asociado a una etiqueta concreta en el idioma activo actual, existe una función que, dado el nombre de la etiqueta mediante una cadena de caracteres \emph{string}, devuelve el texto asociado a ésta mediante una cadena de caracteres anchos \emph{wstring}, el cual se puede dibujar en pantalla (si la fuente seleccionada para ello tiene soporte para el juego de caracteres necesario), guardar en un fichero, o utilizar en cualquier operación deseada.\\

En la figura \ref{lang} se muestra la interfaz pública de la clase Lang.\\

\figura{lang.png}{scale=0.6}{Interfaz pública de la clase Lang}{lang}{h}

\subsection{Pruebas}

Respecto a las pruebas realizadas para validar este módulo de soporte de internacionalización, se han realizado comprobaciones que consistían en imprimir en la pantalla las etiquetas de texto en diversos idiomas, como son español, inglés y ruso (para los caracteres no latinos). También se probaron los métodos de cambio de idioma activo, y la reacción del módulo al pedirle encontrar un nombre de idioma o etiqueta que no existen.



\section{Animaciones}
\label{labelanimaciones}
% Este archivo es parte de la memoria de libWiiEsp, protegida bajo la 
% licencia GFDL. Puede encontrar una copia de la licencia en el archivo fdl-1.3.tex

% Fuente tomada de la plantilla LaTeX para la realización de Proyectos Final 
% de Carrera de Pablo Recio Quijano.

% Copyright (C) 2009 Pablo Recio Quijano
% Copyright (C) 2011 Ezequiel Vázquez de la Calle

% -*-animaciones.tex-*-

Llegados a este punto del desarrollo, prácticamente hemos cubierto todas las herramientas necesarias para crear un videojuego, pero de momento, únicamente disponemos de la posibilidad de dibujar una textura estática, lo que nos limita a la hora de crear elementos dinámicos en nuestro juego. En este capítulo, se consigue un nuevo módulo, partiendo de la clase \emph{Imagen}, con el que se pueden crear animaciones fácilmente.

\subsection{¿Qué es una animación?}

En primer lugar, vamos a definir qué tipo de animación queremos conseguir. La idea de animación que se persigue conseguir es una simulación de movimiento mediante la visión de una secuencia de imágenes (denominadas fotogramas), una detrás de otra, pero con leves variaciones entre sí. Es la misma idea que se utiliza en los dibujos animados, medio en el que se muestra una serie de imágenes estáticas con leves diferencias entre sí que, al ser observadas secuencialmente, consiguen que el cerebro humano interprete un movimiento reconstruyendo los "huecos" que existen entre una imagen y la siguiente.\\

\figura{fotogramas.png}{scale=0.6}{Animación formada por 5 fotogramas}{fotogramas}{h}

Partiendo de este concepto de animación, se va a construir un módulo para \emph{LibWiiEsp} que, tomando como base la información de una textura contenida en un objeto de la clase \emph{Imagen}, se encargue de animar los fotogramas que se indiquen en dicha textura.

\subsection{Identificación de funcionalidad necesaria}

La funcionalidad principal para este módulo es dibujar en pantalla una secuencia de imágenes, tomadas todas ellas a partir de una misma textura en la que se encuentren todos los fotogramas de la secuencia. La animación se reproducirá continuamente, es decir, después del último paso se volverá a comenzar por el primero. Además, se debe permitir que entre un cambio de paso y el siguiente cambio puedan existir algunos fotogramas en los que se dibuje el mismo paso de la animación, para poder regular así la velocidad de reproducción de ésta.\\

Es importante también que se permita dibujar la animación tal cual, o que cada paso de ésta pueda ser invertido respecto al eje vertical; esto se utilizará para optimizar el espacio en memoria en los casos en que dos animaciones distintas sean idénticas, pero con distinta orientación (izquierda y derecha), ya que permitirá utilizar la misma textura para reproducir ambas animaciones.\\

Por último, se necesitará acceder al ancho y alto en píxeles de la animación, y al objeto de la clase \emph{Imagen} que sirve de base para la animación, así como conocer en qué paso de la animación se encuentra ésta en un momento dado.

\subsection{Diseño e implementación}

Se crea una clase nueva para cubrir la funcionalidad descrita en el epígrafe anterior. 

Cuando se crea una animación, se le deben pasar al constructor de la clase una serie de parámetros, que son un puntero a la imagen (previamente cargada en memoria) que contenga la rejilla de fotogramas, una cadena de caracteres de alto nivel (de tipo \emph{string}) que contenga una secuencia de números enteros positivos, separados por comas y sin espacios (del tipo "0,1,2,3,4,5"), el número de filas y de columnas que hay en la rejilla, y el retardo a aplicar a la visualización de la animación (el retardo indica el número de fotogramas consecutivos en los que se dibuja el mismo paso de la animación; el valor por defecto 1 indica que se dibuja un nuevo cuadro a cada fotograma).\\

La secuencia de números enteros que se recibe indica el orden en el que se van dibujando los distintos cuadros de la animación. Cada cuadro de la rejilla corresponde con un número entero positivo, siendo el cero el fotograma de arriba a la izquierda, y avanzando de derecha a izquierda y de arriba abajo. Por ejemplo, una rejilla de tres columnas y dos filas tendrá en su primera fila (de izquierda a derecha) los cuadros cero, uno y dos; y la segunda fila, los cuadros tres, cuatro y cinco (igualmente, de izquierda a derecha). Un mismo cuadro se puede repetir cuantas veces se quiera en la secuencia de una misma animación.\\

Un objeto de animación proporciona varios métodos observadores para conocer y acceder fácilmente a las variables que lo controlan, por ejemplo, la imagen, el ancho y alto en píxeles de un cuadro (todos los cuadros se consideran iguales), y cuáles son los índices del primer cuadro y del cuadro actual. También existen dos métodos para modificar el estado de la animación, concretamente son reiniciar (método que establece el paso actual al primero de la secuencia) y avanzar (que se encarga de calcular el siguiente cuadro a dibujar teniendo en cuenta el retardo y la propia secuencia de cuadros).\\

El método dibujar se encarga de, como su propio nombre indica, plasmar en la pantalla el fotograma correspondiente al cuadro actual de la animación. Se da la opción de invertir el fotograma respecto al eje vertical, funcionalidad que proporciona la propia clase Screen a través de su método \emph{dibujarCuadro} (que es el método que sirve de base para dibujar la parte correspondiente de la textura origen). Este método realiza una llamada a avanzar, con lo que no es necesario preocuparse por la gestión de cuadros desde el exterior.\\

Una cosa más a tener en cuenta es que el destructor de la clase es el predeterminado, por lo que no se destruye la imagen asociada a la animación (se utiliza un puntero constante a la imagen), y hay que destruirla manualmente en caso de que se quiera liberar la memoria ocupada por ésta.\\

En la figura \ref{animacion} puede verse la interfaz pública de la clase Animacion.\\

\figura{animacion.png}{scale=0.6}{Interfaz pública de la clase Animacion}{animacion}{h}

\subsection{Pruebas}

Las pruebas realizadas a este módulo consistieron en crear varias animaciones a partir de otras tantas texturas previamente cargadas en objetos de la clase \emph{Imagen}, y dibujarlas en pantalla con distintos retardos, con inversión respecto al eje vertical o sin ella, y con distintas secuencias de pasos.



\section{Registro de mensajes}
\label{labelmensajes}
% Este archivo es parte de la memoria de libWiiEsp, protegida bajo la 
% licencia GFDL. Puede encontrar una copia de la licencia en el archivo fdl-1.3.tex

% Fuente tomada de la plantilla LaTeX para la realización de Proyectos Final 
% de Carrera de Pablo Recio Quijano.

% Copyright (C) 2009 Pablo Recio Quijano
% Copyright (C) 2011 Ezequiel Vázquez de la Calle

% -*-mensajes.tex-*-

Debido a la complejidad de la depuración de código en la plataforma Nintendo Wii (especialmente en lo referente a depuración en tiempo de ejecución), se hace patente la necesidad de un módulo que permita al desarrollador conocer el estado de las variables y expresiones cuando se está ejecutando un programa en el que se detecta un error. Para cubrir estas situaciones, se crea un sistema de gestión de errores basado en excepciones, que se aplica a todas las clases implementadas hasta el momento, y un módulo que permita generar un registro de mensajes, o \emph{logs}, a partir de la información que el programador estime oportuna.

\subsection{Excepciones}

\programa{LibWiiEsp} proporciona una jerarquía de excepciones que permite controlar los distintos comportamientos erróneos que sucedan durante la ejecución de un programa. Todas las excepciones derivan de la clase \emph{Excepcion}, que a su vez hereda de la clase \emph{std::exception}, de tal manera que es compatible con una gestión de excepciones estándar.\\

Para conocer el mensaje de error que contienen todas las excepciones, puede utilizarse el método \emph{what}, que devuelve una cadena de caracteres de tipo \emph{string} con la información que se haya indicado al lanzar la excepción.\\

Las excepciones que se contemplan son las siguientes:

\begin{itemize}
\item \textbf{Excepcion}: Clase de excepción que sirve como base a todas las demás.
\item \textbf{ArchivoEx}: Excepción para controlar errores relacionados con la apertura de archivos.
\item \textbf{CodigoEx}: Sirve para gestionar los fallos que puedan producirse al trabajar con códigos identificativos.
\item \textbf{ImagenEx}: Para tratar los errores concretos de la clase Imagen.
\item \textbf{NunchuckEx}: Esta clase de excepción se utiliza para informar de situaciones relacionadas con el \emph{Nunchuck}.
\item \textbf{TarjetaEx}: Generalmente sólo se utiliza para indicar que la tarjeta SD no está disponible.
\item \textbf{XmlEx}: Excepción que permite identificar cuándo sucede un error al trabajar con un árbol XML.
\end{itemize}

\subsection{Identificación de funcionalidad necesaria}

La gestión de mensajes del sistema debe estar centralizada en un punto del sistema, que además resulte accesible desde cualquier punto de un programa. Lo que se pretende con la gestión de errores y el registro de mensajes consiste en almacenar textos con información relevante sobre el estado del sistema en un momento determinado. Dicha información se guardará temporalmente en un \emph{búffer} en memoria y será posteriormente volcada a un archivo de texto plano en la tarjeta SD.\\

Se permitirán tres niveles de prioridad para los mensajes: errores (sólo se muestran los mensajes de situaciones que generen una interrupción en la ejecución del programa), avisos (se muestran los anteriores y también aquellos derivados de los casos en que, sin llegar a interrumpir la ejecución, provocan un comportamiento anómalo en el sistema), e información, que muestra todos los mensajes definidos.\\

También debe poderse apagar el sistema de errores, de tal manera que no genere un archivo de \emph{logs} o registro cuando un programa ya esté en producción.

\subsection{Diseño e implementación}

Para el registro de mensajes del sistema, se crea una clase que implementa el patrón \emph{Singleton}, ya que queremos centralizar toda la gestión de mensajes en un único punto del sistema, y tenerlo siempre accesible. Esta clase tendrá como atributos privados el nivel de los mensajes que quieren observar, un \emph{búffer} para almacenar todos los textos que se escribirán en el archivo de \emph{logs} (que es una cadena de caracteres \emph{string}) y un flujo de salida de bytes a fichero, en el que se realizará el volcado de información desde el \emph{búffer} hacia el archivo.\\

Los niveles de registro están agrupados en una enumeración (con los valores OFF, ERROR, AVISO, INFO) que, a su vez, se define como tipo público de la clase para facilitar su utilización. El orden es desde el nivel más restrictivo (OFF, no se registra nada) a más amplio (INFO, se registra todo). Un nivel más amplio incluye a los más restrictivos que él, lo cual quiere decir que si, por ejemplo, se activara el nivel AVISO, se registrarían tanto mensajes de aviso como de error.\\

Cuando se inicializa el sistema de registro de mensajes, se guarda el nivel de \emph{log} introducido como parámetro, y si éste no es OFF (no registrar nada), se sigue adelante. Se comprueba que la tarjeta SD esté disponible para realizar operaciones de entrada/salida, se abre el archivo de \emph{log} indicado en el primer parámetro en modo escritura (si no existiera, se crea), y se establece el \emph{búffer} para los mensajes. La instancia irá registrando mensajes en el búfalo, cada uno en una línea, y comenzando por las cadenas [INFO] para el nivel de información, [AVISO] para el nivel de aviso, y [ERROR] para una cadena que contenga un mensaje con nivel de registro error.\\

El guardado de mensajes se realiza por orden de ejecución, es decir, si un mensaje se encuentra por debajo de otro en el archivo de texto plano, significa que se ejecutó posteriormente al segundo. Por último, hay que tener en cuenta que el volcado del \emph{búffer} en el archivo se realiza cuando se destruye la instancia de la clase. La ventaja de esta decisión es que, al realizarse un único guardado, el rendimiento del programa no se ve afectado con las muchas operaciones de escritura que pueden darse durante la ejecución. Sin embargo, existe un inconveniente, ya que si ocurriera un error que no estuviera controlado en el código, se generaría un \emph{pantallazo} y se finalizaría la ejecución sin guardar el registro de mensajes en el archivo. En versiones posteriores se mejorará este aspecto, haciendo posible que se puedan guardar la información generada hasta un cierto momento, a voluntad del programador.\\

En la figura \ref{logger} se muestra la interfaz pública de la clase Logger.\\

\figura{logger.png}{scale=0.6}{Interfaz pública de la clase Logger}{logger}{h}

\subsection{Pruebas}

Para comprobar el buen funcionamiento de esta clase, se creó un pequeño programa de prueba que escribía mensajes de sistema en distintas prioridades, y se probó este programa con los distintos niveles posibles para la clase (OFF, ERROR, AVISO, INFO).



\section{Detección de colisiones}
\label{labelcolisiones}
% Este archivo es parte de la memoria de libWiiEsp, protegida bajo la 
% licencia GFDL. Puede encontrar una copia de la licencia en el archivo fdl-1.3.tex

% Fuente tomada de la plantilla LaTeX para la realización de Proyectos Final 
% de Carrera de Pablo Recio Quijano.

% Copyright (C) 2009 Pablo Recio Quijano
% Copyright (C) 2011 Ezequiel Vázquez de la Calle

% -*-colisiones.tex-*-

La décima etapa de desarrollo del proyecto aborda uno de los módulos más importantes a la hora de crear un videojuego en dos dimensiones, ya que debe ser escrupulosamente eficiente y, a la vez, permitir una gran flexibilidad. En esta sección se detallan todos los pormenores del sistema de colisiones de \emph{LibWiiEsp}.\\

Este módulo de detección de colisiones no depende de ninguna otra parte del sistema, así que puede ser reutilizado para otros desarrollos no relacionados con Nintendo Wii.

\subsection{Identificación de funcionalidad necesaria}

Las colisiones se calcularán en base a figuras geométricas planas, y una colisión ocurrirá siempre entre dos de estas figuras. Se necesita que, cuando se evalúe una colisión, se indique únicamente si efectivamente ha ocurrido o no, no es necesario identificar el punto o puntos de colisión. Las figuras entre las que se calcularán colisiones serán un punto, un círculo y un rectángulo cuyos lados estén alineados a los ejes.\\

Las figuras de colisión estarán normalmente asociadas a entidades propias del juego que tendrán unas coordenadas respecto a un punto (0,0) determinado por la esquina superior izquierda del escenario, por lo que se ha planteado que existen dos formas de representar la posición de las figuras de colisión asociadas:

\begin{enumerate}
\item Figuras de colisión con posición absoluta respecto al mismo origen de coordenadas que las entidades del juego.
\item Que las figuras tengan su posición relativa al punto superior izquierdo del objeto del juego con el que están asociadas.
\end{enumerate}

Ambas aproximaciones tienen sus ventajas e inconvenientes. El primer punto de vista facilita los cálculos, pero necesita que se actualice la posición de las figuras cada vez que el objeto al que están asociadas modifique la suya. Por otro lado, la segunda opción complica ligeramente la detección de la colisión (ya que habría que tener en cuenta la posición de los objetos contenedores de las figuras), pero elimina la necesidad de actualización de la posición de las figuras.\\

Se ha elegido la segunda opción por considerarla más eficiente, así que será necesario que en los cálculos matemáticos se tengan en cuenta las posiciones de los objetos asociados a las figuras de colisión.\\

Por último, se necesita que una figura de colisión pueda ser cargada desde un elemento de un árbol XML.

\subsection{Diseño e implementación}

En base a lo decidido en el punto anterior, se crea una clase abstracta pura, llamada Figura, y de la cual derivarán todas las demás clases que representen los distintos polígonos de colisión. Concretamente, de la clase Figura derivarán las clases Punto, Rectángulo y Círculo.\\

La clase punto tendrá dos enteros para indicar sus coordenadas respecto a la esquina superior izquierda del objeto del sistema con el que estará asociado, un círculo estará determinado por un punto y un número decimal que indique el radio, y un rectángulo estará formado por cuatro puntos. A pesar de que los rectángulos tendrán sus lados alineados a los ejes de coordenadas, se utiliza la representación de cuatro puntos en lugar de un punto y las longitudes de los lados para que, en el futuro, se pueda eliminar fácilmente la restricción de los lados paralelos a los ejes.\\

La clase abstracta Figura es la base del módulo de gestión de colisiones integrado en \emph{LibWiiEsp}. Se basa en una implementación de la técnica \emph{double dispatch} (que a su vez es una implementación del patrón \emph{Visitante})consistente en una especie de polimorfismo en tiempo de ejecución, donde la elección del método a ejecutar depende no sólo del objeto que lo ejecuta, si no del tipo del parámetro que recibe. En la práctica, al emplear esta técnica se consigue evitar la comprobación de tipos mediante estructuras de tipo condicional (\emph{if}, \emph{switch} \ldots) cuando se quiere evaluar una colisión entre dos objetos de clases derivadas de Figura.\\

Entrado un poco más en detalle, \emph{double dispatch} implica que toda clase derivada de Figura implementará un método denominado \emph{hayColision}, en el que recibirá un puntero a zona de memoria donde se almacenará otra Figura. Este método devolverá un valor booleano, que será verdadero si hay colisión entre las dos figuras, o falso en caso contrario. Internamente, devuelve el resultado de llamar al método \emph{hayColision} de la segunda figura, pasando como parámetro la figura actual (es decir, el objeto \emph{this} en el contexto de la primera figura). Con este intercambio de parámetros se consigue que esta segunda llamada se realice conociendo el tipo exacto del parámetro que se pasa, ya que en el contexto de ejecución (se hace desde un método de la primera figura) se conoce el tipo del objeto \emph{this}. Al llamarse el método \emph{hayColision} de la segunda figura con un parámetro con el tipo perfectamente identificado, esta segunda figura ya sabe los tipos de ambos objetos (conoce su propio tipo por el contexto de ejecución, y el de la primera figura por lo explicado anteriormente), de tal manera que se ejecuta el método adecuado con los cálculos necesarios para averiguar si hay colisión entre ambas figuras o no. En la figura \ref{doubledispatch} se puede observar un ejemplo de implementación de la técnica.\\

\figura{doubledispatch.png}{scale=0.7}{Sencillo ejemplo de implementación de la técnica de \emph{Double Dispatch}}{doubledispatch}{h}

Gracias a esta técnica de \emph{double dispatch} se consigue una mayor escalabilidad del código de las colisiones, ya que resulta muy sencillo añadir nuevas figuras que detecten colisiones entre sí y entre las ya existentes.\\

El otro aspecto importante relativo a las figuras de colisión es el desplazamiento de éstas. Una figura de colisión está determinada, como mínimo, por una pareja de coordenadas cuyo punto de origen (es decir, el (0,0)) corresponde con el punto superior izquierdo de un objeto del sistema; objeto que tendrá, a su vez, una pareja de coordenadas (x,y) respecto al límite izquierdo del escenario (coordenada X, cuanto mayor sea, más alejado hacia la derecha estará el objeto de este límite izquierdo) y al límite superior del escenario (coordenada Y, cuanto mayor sea, más alejado hacia abajo estará el objeto de este límite superior). Es decir, las coordenadas de una figura de colisión son relativas a un punto que no tiene por qué ser el origen (0,0) de las coordenadas de los objetos del sistema.\\

Por ejemplo, suponiendo un personaje en un juego de plataformas que tenga asociado un rectángulo de colisiones. Este personaje tiene unas coordenadas (x,y) respecto al punto superior izquierdo del escenario, y que determinan su posición en dicho escenario. Esta pareja de coordenadas variará en función del comportamiento del personaje durante la ejecución del juego. Sin embargo, las parejas de coordenadas de los puntos que determinan el rectángulo de colisiones se mantendrán fijas en todo momento, ya que son relativas al punto superior izquierdo del personaje, y no al del escenario. Esto provoca que, para conocer la posición absoluta de un punto del rectángulo respecto al origen de coordenadas del escenario, haya que sumar el valor de la coordenada X del personaje con el de la coordenada X del punto; y análogamente para las coordenadas Y. En la figura \ref{desplazamiento} se puede observar un ejemplo del sistema de coordenadas de las figuras de colisión, donde el origen de coordenadas es el punto superior izquierdo del objeto con el que están las figuras asociadas.\\

\figura{desplazamiento.png}{scale=0.6}{Ejemplo para ilustrar el desplazamiento de las figuras de colisión}{desplazamiento}{h}

Tomando en consideración todo lo explicado, se ha creado una clase abstracta pura, llamada Figura, que será la base para todas las clases derivadas que representen a una figura concreta. Esta clase Figura incluye métodos de clase (estáticos) que se encargan de crear una figura concreta a partir de un elemento de un árbol XML, así como los métodos virtuales puros que implementan el patrón \emph{Visitante} y la técnica del \emph{double dispatch}.\\

A partir de esta clase, se crean las clases Punto, Rectangulo y Circulo, cada una con los métodos necesarios para evaluar las posibles colisiones entre sí. También incluyen los métodos observadores y modificadores para todos sus atributos. La interfaz pública de la clase Figura pueden observarse en la figura \ref{colisiones}.\\

\figura{colisiones.png}{scale=0.6}{Interfaz pública de la clase Figura}{colisiones}{h}

Por último, comentar que se ha conseguido una escalabilidad alta, ya que para añadir una figura nueva (por ejemplo, un triángulo), bastaría con crear una nueva clase que herede de Figura, y añadir a las existentes los métodos necesarios para detectar colisiones entre ellas y la nueva.

\subsection{Pruebas}

Las comprobaciones realizadas a este módulo consistieron en un programa que generaba varias figuras y comprobaba las colisiones entre ellas. Previamente a la ejecución del programa, se sabía qué figuras debían colisionar con otras. A la hora de probar la figura Punto se creó un programa interactivo, en el que se asignaban las coordenadas del puntero infrarrojo del mando a un objeto de la clase, y se actualizaba su posición en cada fotograma. Además, se incluían varias figuras (rectángulos y círculos) que cambiaban de color según apuntáramos el puntero infrarrojo sobre ellas o no (es decir, si se detectaba una colisión entre el punto y las figuras).



\section{Plantillas}
\label{labelplantillas}
% Este archivo es parte de la memoria de libWiiEsp, protegida bajo la 
% licencia GFDL. Puede encontrar una copia de la licencia en el archivo fdl-1.3.tex

% Fuente tomada de la plantilla LaTeX para la realización de Proyectos Final 
% de Carrera de Pablo Recio Quijano.

% Copyright (C) 2009 Pablo Recio Quijano
% Copyright (C) 2011 Ezequiel Vázquez de la Calle

% -*-plantillas.tex-*-

Con todo lo desarrollado hasta el momento, ya se dispone de una herramienta completa para construir videojuegos para Nintendo Wii. A partir de este punto, se considera necesario la creación de una serie de plantillas (clases abstractas) para facilitar la implementación de videojuegos. El uso o no de estas plantillas son totalmente opcionales, pero suponen una ayuda para aquellos programadores que no tengan mucha experiencia en el desarrollo de videojuegos en dos dimensiones.

\subsection{Actores}

Un actor es la unidad básica en el desarrollo de un videojuego en dos dimensiones, y consiste en un objeto con entidad propia que se puede visualizar en la pantalla, y que almacena información sobre sus distintas características. Estas características son las siguientes: posición actual dentro del universo del juego (pareja de coordenadas (x,y)), posición previa del actor en el universo del juego (otro par de coordenadas (x,y)), velocidad de movimiento horizontal y vertical (número de píxeles que avanza el actor en un fotograma en cada uno de los ejes), estado actual del actor, estado previo del actor y el tipo de actor (una cadena de caracteres). Por supuesto, al derivar esta clase abstracta, se pueden añadir más características según se crea necesario.\\

Realizando un acercamiento más técnico y concreto, un actor no es más que un conjunto de las características antes mencionadas, además de dos diccionarios donde, a cada estado del actor, le corresponde una animación y un conjunto de cajas de colisión, y que proporciona una manera sencilla y efectiva de cargar todas estas características desde un archivo XML almacenado en la tarjeta SD de la consola.\\

Existen dos referencias para las coordenadas de un actor. En primer lugar, están las coordenadas de posición del actor dentro del escenario que conforma el propio juego, y que se refieren al desplazamiento en píxeles hacia la derecha del actor respecto al límite izquierdo del escenario (coordenada X), y al desplazamiento hacia abajo, también en píxeles, del actor respecto al límite superior de dicho escenario (coordenada Y). Por otra parte, existe otra referencia para las coordenadas de dibujo de un actor en la pantalla, y que se refieren a la posición respecto a los límites izquierdo y superior de la pantalla, además de la capa en la que se dibuja el actor (consultar la documentación de la clase Screen para más información respecto al dibujo de texturas en la pantalla).\\

El comportamiento de un actor viene definido por los distintos estados que puede adoptar a lo largo de la ejecución del juego. Cada estado tiene asociado un comportamiento concreto, y que se ejecutará en cada iteración del bucle principal del juego en la que el actor tenga activo ese estado concreto. Las transiciones entre estados se realizan externamente (generalmente, desde la clase que gestiona el escenario, y que deriva de la clase abstracta Nivel). Los estados se identifican mediante una cadena de caracteres de alto nivel (de tipo \emph{string}), y se crean según aparezcan en el archivo XML que define las animaciones y las cajas de colisión para cada estado, es decir, un estado sólo se creará si aparece en algún momento en el mencionado archivo XML. El comportamiento del actor según su estado actual debe definirse en el método virtual puro \emph{actualizar} cuando se derive la clase Actor.

Un detalle muy importante es que existe un estado obligatorio que hay que incluir forzosamente, y es el estado "normal", que es el que se toma por defecto. Si este estado faltara en el archivo XML del actor, se produciría un error y/o un comportamiento inesperado.\\

La cadena de caracteres que identifica a un estado se toma como clave para dos diccionarios, uno que almacena una animación asociada al estado concreto, y otra que almacena un conjunto (de tipo \emph{set<Figura*>}) de figuras de colisión también asociadas al mismo estado. Estos diccionarios que tienen códigos de estado como clave identificativa se utilizan para que, en cada iteración del bucle principal, se dibuje la animación del actor correspondiente con el estado actual de éste, y sólo se tengan en cuenta las cajas de colisión asociadas al estado actual del actor para evaluar sus colisiones. Se proporcionan métodos para evaluar colisiones entre dos actores de una manera muy sencilla, y también para dibujar el actor en la pantalla en base a unas coordenadas de pantalla que se reciben desde fuera de la clase (la diferencia entre las coordenadas de posición y las de pantalla se explican unos párrafos más arriba).\\

Otro detalle más es que se almacena un identificador de tipo de actor, es decir, todos los actores que estén gestionados por la misma clase derivada de Actor deberán tener este atributo con el mismo valor.\\

Como ya se ha comentado, se consigue una separación completa de código fuente y datos, de tal manera que para cambiar las figuras de colisión, los estados o las animaciones de un actor, basta con modificar el archivo XML desde el cual se lee toda esta información. Este archivo tendrá una estructura parecida a esta:

\begin{lstlisting}[style=XML]
<?xml version="1.0" encoding="UTF-8"?>
  <actor vx="3" vy="3" tipo="jugador">
    <animaciones>
    <animacion estado="normal" img="chief" sec="0" filas="1" columnas="5" retardo="3" />
    <animacion estado="mover" img="chief" sec="0,1,2,3,4" filas="1" columnas="5" retardo="3" />
    <animacion estado="muerte" img="chief" sec="5" filas="1" columnas="6" retardo="0" />
  </animaciones>
  <colisiones>
    <rectangulo estado="normal" x1="27" y1="21" x2="55" y2="21" x3="55" y3="96" x4="27" y4="96" />
    <circulo estado="normal" cx="41" cy="13" radio="8" />
    <rectangulo estado="mover" x1="27" y1="21" x2="55" y2="21" x3="55" y3="96" x4="27" y4="96" />
    <circulo estado="mover" cx="41" cy="13" radio="8" />
    <sinfigura estado="muerte" />
    </colisiones>
  </actor>
\end{lstlisting}

En el elemento raíz se observan tres atributos, que son la velocidad de movimiento horizontal del actor en píxeles por fotogramas, la velocidad de movimiento vertical, y el tipo de actor.\\

Se pueden apreciar dos grandes bloques, uno para las animaciones, y otro para las cajas de colisión. Cada animación y figura incluye la información necesaria para ser creada, además de un estado al que se asociará. Sólo puede haber una animación por estado; pero no ocurre así para las cajas de colisión, de las cuales puede haber todas las que se deseen en cada estado. Cade destacar que, en ambos bloques, aparece el estado obligatorio "normal", como ya se indicó anteriormente.\\

Por último, cabe destacar que al derivar la clase Actor, se le pueden añadir nuevos atributos y métodos según se considere necesario, consiguiendo así partir de una base como es la propia clase Actor, pero pudiendo llegar a la complejidad que se desee. También hay que mencionar que a la hora de crear un actor, es apropiado diseñar los cambios entre estados como un autómata finito determinado, que como ya se ha explicado, realizaría sus transiciones entre estados de forma externa. Si se realiza de esta manera, es muy fácil desarrollar un actor en un período de tiempo relativamente pequeño.

\subsection{Niveles}

Un nivel, desde el punto de vista de \emph{LibWiiEsp}, no es más que un escenario donde los actores participan en el juego. Está formado por tres elementos principales que son los actores, el conjunto de \emph{tiles} que componen el propio nivel y una imagen de fondo. También tiene asociada una pista de música propia, una imagen desde la que se toman los \emph{tiles} que se dibujan, y unas coordenadas de \emph{scroll} para la pantalla, así como un ancho y un alto.\\

El escenario que presenta un nivel es de forma rectangular, y tiene un ancho y un alto definidos por el número de \emph{tiles} que lo componen. A pesar de ello, en todo momento están disponibles sus medidas en píxeles. Dentro de este escenario, todos los elementos tienen una pareja de coordenadas que indican la distancia horizontal hacia la derecha respecto al límite izquierdo del nivel (coordenada X), y la distancia vertical hacia abajo respecto al límite superior del nivel (coordenada Y). Así pues, el origen de coordenadas del nivel (el punto (0,0)) es el vértice superior izquierdo del rectángulo que forma el escenario.\\

En la pantalla sólo puede dibujarse una parte del escenario del nivel, ya que ésta tiene un tamaño determinado por el modo de vídeo. La parte del nivel que se dibuja en la pantalla en un momento determinado es la ventana del nivel, y es un rectángulo con las mismas medidas que la pantalla, y que tiene unas coordenadas respecto al punto (0,0) del escenario. Estas coordenadas son el \emph{scroll} del nivel, y se puede modificar mediante el método \emph{moverScroll}.\\

La imagen de fondo del nivel debe tener el tamaño de la pantalla completa (para una televisión PAL de proporciones 4:3, las medidas en píxeles son 640x528), y será fija durante todo el transcurso del nivel. La idea de esta imagen de fondo estática es para situar un paisaje que acompañe al nivel, ya que la sensación de avance se produce con los propios \emph{tiles} que componen la estructura del nivel.\\

Existen dos tipos de \emph{tiles} en un nivel, que se distinguen únicamente en que los \emph{tiles} no atravesables tienen un rectángulo de colisión con el mismo tamaño y posición, y por lo tanto, provocará que un actor que colisione con él pueda reaccionar de una manera; sin embargo, los \emph{tiles} atravesables no disponen de esta figura de colisión, y tienen únicamente un objetivo decorativo, para dotar de mayor detalle al escenario del nivel, pero no provocarán una colisión cuando un actor los toque. Todos estos \emph{tiles} se almacenan en la misma estructura (de tipo \emph{Escenario}), y para comprobar si un actor concreto colisiona con algún elemento del nivel se proporciona el método \emph{colision} (información útil para saber, por ejemplo en un juego de plataformas, si el actor está cayendo o está sobre una plataforma). Si el programador necesita un mayor detalle en la detección de colisiones entre actores y escenario, siempre puede añadir los métodos y atributos que considere necesarios para ello al crear la clase derivada de Nivel. La detección de colisiones entre un actor y los \emph{tiles} del escenario está optimizada para que sólo se evalúe la colisión con los \emph{tiles} sobre los que se encuentre el actor, evitando cálculos innecesarios.\\

Se proporciona un método virtual puro \emph{actualizarEscenario} en el que se pueden implementar comportamientos para el escenario (como por ejemplo, plataformas destructibles, o en movimiento, además de cambiar las coordenadas del \emph{scroll} de la pantalla), y un método que dibuja la ventana del nivel en la pantalla de la consola.\\

Los actores que participan en un nivel se distinguen entre los actores que son dirigidos por los jugadores, y los actores que controla la máquina. Cada uno de ellos se almacena en una estructura distinta, y dispone de un método concreto para actualizarlo. Concretamente, el método para actualizar los actores no jugadores es \emph{actualizarNpj}, en el que se supone que se debe implementar la actualización de todos los actores controlados por la máquina (la decisión sobre cómo llevar a cabo esta operación queda, por entero, en manos del programador). Igualmente, para actualizar los actores jugadores, se dispone del método \emph{actualizarPj}, que debe actualizar a un único jugador en cada llamada, y lo hace recibiendo el código identificador único del jugador, y el mando asociado a él.\\

Hay que destacar especialmente la filosofía de esta clase Nivel junto con la clase Actor. En cada clase derivada de Actor se debe definir el comportamiento de cada actor dependiendo de su estado actual. Sin embargo, las transiciones entre estados de un actor pueden realizarse desde el Nivel en el que el actor está participando, o bien en el propio actor, ya que éste dispone de una referencia constante al Nivel en el que se encuentra, aunque lo que se recomienda es tomar la primera opción, y situar las transiciones entre estados en el método de actualización del Nivel, de tal manera que los actores respondan con un comportamiento u otro, dependiendo de cómo se les haya definido, y siempre según el estado que tengan activado.\\

Una vez descritos los elementos que conforman el escenario de un nivel, hay que detallar especialmente el proceso de carga de un nivel desde un archivo TMX creado con el editor de mapas de \emph{tiles} Tiled \cite{website:tiled}:

\begin{itemize}
\item Cuando se crea un nivel, el constructor recibe un archivo TMX que se carga desde la tarjeta SD gracias a la clase Parser. Una vez en memoria, se lee el ancho y alto del nivel en \emph{tiles}, y el ancho y el alto en píxeles de un único \emph{tile} (las medidas de un \emph{tile} deben ser múltiplos de 8, ya que llevarán una imagen asociada). A partir de estos datos, se calcula el ancho y alto del escenario del nivel en píxeles.
\item El proceso de lectura desde el archivo TMX continúa leyendo las propiedades del mapa, que son \emph{imagen\_fondo} (que es el código que debe tener la imagen de fondo del nivel en la galería de medias), \emph{imagen\_tileset} (código que la imagen del tileset deberá tener asociado en la galería de medias del sistema) y musica (código identificador de la pista de música en la galería de medias del sistema). Tras leer las propiedades del mapa de \emph{tiles}, se procede a leer las tres capas que debe contener el archivo TMX, que son las siguientes:
	\begin{enumerate}
	\item Capa 'escenario': capa de patrones del editor Tiled, debe contener los \emph{tiles} atravesables.
	\item Capa 'plataformas': capa de patrones del editor Tiled, que contiene los \emph{tiles} no atravesables (los que tendrán figura de colisión asociada).
	\item Capa 'actores': capa de objetos del editor Tiled. Cada objeto definido en esta capa debe tener la propiedad XML con la dirección absoluta en la tarjeta SD del archivo XML de descripción del actor como valor. Si además, el actor es un actor jugador, se espera que tenga otra propiedad llamada jugador, y cuyo valor debe ser el código identificador del jugador. También debe tener el identificador de su tipo en el campo Tipo.
	\end{enumerate}

\item Hay que tener en cuenta otro detalle importante, y es que, al llamar al constructor, la información de los actores se guardan en una estructura temporal. Por ese motivo, existe el método virtual puro \emph{cargarActores}, que ha de ser implementado por el programador al derivar la clase Nivel, y que se debe encargar de recorrer esta estructura temporal, crear cada actor utilizando el constructor de la clase correspondiente al tipo del actor, y almacenarlo en la estructura correspondiente (dependiendo de si es un actor jugador o no jugador). Por último, es conveniente que este método vacíe la estructura temporal para no malgastar memoria.\\
\end{itemize}

Esto último es necesario ya que, en la propia clase abstracta Nivel que forma parte de \emph{LibWiiEsp} no se conoce qué clases derivadas habrá de Actor, y por tanto, se deja en manos del programador implementar la creación de Actores en el nivel. A continuación, se establece el \emph{scroll} del nivel al valor (0,0), y se da por concluida la carga del nivel. Para conocer todos los detalles sobre la creación de niveles con el editor Tiled, consultar el manual de \emph{LibWiiEsp}.

\subsection{Juego}

Esta clase abstracta proporciona una plantilla sobre la que construir el objeto principal de toda aplicación desarrollada con \emph{LibWiiEsp}. Consiste en dos apartados bien diferenciados, que son la inicialización de todos los subsistemas de la consola y de la biblioteca (esto se realiza desde el constructor), y la ejecución del bucle principal de la aplicación. Como clase abstracta que es, no se puede instanciar directamente, si no que hay que crear a partir de ella una clase derivada en la que se definan los métodos que el programador considere oportunos para gestionar el programa.\\

Además de la inicialización de la consola y el control del bucle principal del programa, esta clase también se encarga de gestionar la entrada de hasta cuatro mandos en la consola. Para ello, dispone de un diccionario en el que cada jugador, que está identificado por un código único (de tipo \emph{string}), tiene asociado un mando concreto de la consola. Para acceder al mando de un jugador, basta con buscar en el diccionario mediante el código identificativo del éste.\\

El constructor de la clase recibe como parámetro un archivo XML con un formato concreto, en el que se deben especificar las opciones de configuración para el programa, como son el nivel de registro de mensajes deseado y la ruta completa del archivo de registro que se generará en el caso de estar el subsistema activado, el color considerado transparente y que no se dibujará en la pantalla (en formato 0xRRGGBBAA), el número de fotogramas por segundo (FPS) que se quiere que tenga el videojuego, la ruta absoluta hasta el archivo XML donde se especifican los recursos multimedia que deben ser cargados en la galería de medias, la ruta absoluta hasta el archivo XML donde se especifican las etiquetas de texto del sistema de internacionalización y el nombre del idioma por defecto, y por último, la configuración de jugadores, consistente en cuatro atributos de tipo cadena de caracteres en los que se deben especificar los códigos identificadores para cada uno de los jugadores.\\

\begin{lstlisting}[style=XML]
<?xml version="1.0" encoding="UTF-8"?>
<conf>
  <log valor="/apps/wiipang/info.log" nivel="3" />
  <alpha valor="0xFF00FFFF" />
  <fps valor="25" />
  <galeria valor="/apps/wiipang/xml/galeria.xml" />
  <lang valor="/apps/wiipang/xml/lang.xml" defecto="english" />
  <jugadores pj1="pj1" pj2="pj2" pj3="" pj4="" />
</conf>
\end{lstlisting}

El constructor de la clase Juego, que debe ser llamado en el constructor de la correspondiente clase derivada, se encarga en primer lugar de montar la primera partición de la tarjeta SD (debe ser una partición con sistema de ficheros FAT), cargar el árbol XML del archivo de configuración en memoria mediante la clase Parser, e inicializar el sistema de registro de mensajes según se haya especificado en el archivo de configuración. Si en alguna de estas operaciones se produjera un error, se saldría del programa mediante una llamada a la función \emph{exit}.\\

A partir de ese momento, el constructor inicializará los sistemas de vídeo (clase Screen), controles (clase Mando), sonido (clase Sonido) y la biblioteca \emph{FreeType2} de gestión de fuentes de texto (clase Fuente). A continuación, establece el color transparente del sistema, el número de FPS, leerá la configuración de los jugadores, y creará una instancia de Mando para cada jugador que, en el archivo de configuración, no tenga una cadena vacía como identificador. Posteriormente, guarda cada mando asociado al código del jugador correspondiente en el diccionario de controles.\\

Por último, se cargan todos los recursos media en la galería, y las etiquetas de texto para los idiomas del sistema en la clase Lang.\\

Después de haber inicializado la consola partiendo del archivo de configuración, el método virtual \emph{run} proporciona una implementación básica del bucle principal del programa, en la que se controlan las posibles excepciones que puedan lanzarse (las cuales captura y almacena su mensaje en el archivo de mensajes), se mantiene constante la tasa de fotogramas por segundo, y se actualizan todos los mandos conectados a la consola y se gestiona correctamente la actualización de los gráficos a cada fotograma. Junto a este método también se proporciona un método virtual puro, llamado \emph{cargar}, y que se ejecuta antes de entrar en el bucle principal. Este método permite realizar las operaciones que se estimen necesarias antes de comenzar la ejecución del bucle, en el caso de que fueran necesarias (en caso contrario, bastaría con definir el método como una función vacía a la hora de derivar la clase Juego).\\

Igualmente, al ser el método \emph{run} virtual, si el programador necesita otro tipo de gestión para el bucle principal de la aplicación, basta con que lo redefina en la clase derivada; a pesar de ello, tendrá disponible un sencillo ejemplo de control del bucle de la aplicación en la definición del método en la clase Juego.\\

Por último, especificar un detalle, y es que el destructor de la clase Juego se encarga de liberar la memoria ocupada por los objetos de la clase Mando, de tal manera que no hay que preocuparse por ello.



\section{Ejemplos de \programa{LibWiiEsp}}
% Este archivo es parte de la memoria de libWiiEsp, protegida bajo la 
% licencia GFDL. Puede encontrar una copia de la licencia en el archivo fdl-1.3.tex

% Fuente tomada de la plantilla LaTeX para la realización de Proyectos Final 
% de Carrera de Pablo Recio Quijano.

% Copyright (C) 2009 Pablo Recio Quijano
% Copyright (C) 2011 Ezequiel Vázquez de la Calle

% -*-ejemploslibwiiesp.tex-*-

Una vez concluido el desarrollo de \programa{LibWiiEsp}, quise aportar una serie de ejemplos para ilustrar hasta qué punto es útil la herramienta con la intención añadida de que sirvieran como iniciación al uso de la biblioteca. En esta sección se trata el funcionamiento de los tres juegos que acompañan a \programa{LibWiiEsp}, explicando los detalles más relevantes de su implementación.

\subsection{Arkanoid}

Este primer juego es un clon de clásico de 1986, publicado por la empresa japonesa Taito. Consiste en una bola que se mueve por la zona de juego, rebotando en las paredes, y con la cual el jugador debe intentar destruir una serie de ladrillos que se encuentran en el escenario. Para conseguir su objetivo, el jugador controla una pala con la que debe evitar que la bola caiga por la zona inferior de la pantalla, en cuyo caso pierde una oportunidad. Además, cuando se destruye un ladrillo puede generarse aleatoriamente un objeto que cae verticalmente, y que si es recogido con la pala, puede aportar beneficios al jugador (una oportunidad extra, un multiplicador temporal de puntuación obtenida, o un aumento del tamaño de la pala).\\

La novedad de este desarrollo respecto al original es que se permite controlar la pala con los botones de la cruceta del mando, o bien mediante el ángulo de cabeceo (al más puro estilo del Mario Kart Wii de Nintendo).\\

Entrando en el apartado técnico, el juego tiene cuatro actores, que son la bola, los ladrillos, la pala y los objetos, que tienen sus animaciones y figuras de colisión organizadas por estados, información que se carga desde sendos archivos XML. Los escenarios, que tienen toda la lógica del juego implementada en la clase Escenario (derivada de Nivel), están diseñados con el editor Tiled \cite{website:tiled}, y una última clase, llamada Juego, se encarga de controlar el bucle principal.\\

La pala se mueve horizontalmente según se indique mediante el mando hasta tocar los límites laterales de la zona de juego; por otro lado, los objetos simplemente caen hasta la zona inferior de ésta, y cuando llegan simplemente desaparecen. Los ladrillos no se mueven.\\

El detalle más reseñable es el movimiento de la bola, que se compone de dos variables, velocidad y dirección, que son el módulo y el ángulo de un vector de movimiento, respectivamente. Cada vez que hay un rebote de la bola, se calcula un nuevo vector de movimiento para ésta, y cada fotograma la bola por la zona de juego se mueve en horizontal y en vertical un número de píxeles equivalentes a la parte entera de las componentes horizontal y vertical del vector. Cuando se destruye un ladrillo, el módulo del vector aumenta un poco, hasta llegar al límite establecido para la velocidad máxima en el archivo XML de la bola. Cabe destacar también el hecho de que el punto de impacto de la bola con la pala influye en el nuevo vector que tendrá tras el rebote.\\

Este juego resulta muy sencillo de ampliar mediante escenarios nuevos, e incorpora el soporte de internacionalización (i18n) que incluye \programa{LibWiiEsp}.\\

Por último, mencionar que durante el desarrollo de Arkanoid se tuvieron que efectuar algunas modificaciones en la biblioteca, principalmente relacionadas con la clase abstracta Actor. En su momento, se diseñó esta clase de tal forma que se facilitaba el desarrollo de juegos de plataforma, pero rápidamente se comprobó que atributos como la dirección no resultarían útiles como tipos enumerados, lo que ocasionó que se eliminara esa característica. También hubo que corregir la representación del mapa de \emph{tiles} del escenario, ya que antes se guardaba en un conjunto \emph{set}, y por tanto había que recorrer todos los \emph{tiles} para identificar si un actor colisionaba con el escenario. Con la implementación actual, se consigue optimizar mucho el rendimiento de la detección de colisiones, ya que sólo se evalúan los \emph{tiles} que se encuentren alrededor del actor.\\

En la figura \ref{arkanoid} puede observarse una captura de pantalla de este juego.\\

\figura{arkanoid.png}{scale=0.6}{Captura de pantalla de Arkanoid}{arkanoid}{h}

\subsection{Duck Hunt}

El segundo juego desarrollado es una versión del juego de Nintendo de 1984. A diferencia del original, en esta implementación de Duck Hunt pueden jugar dos personas simultáneamente, y los patos salen cada poco tiempo desde uno de los laterales de la pantalla hasta el otro, con un ángulo de inclinación. El objetivo del juego es llegar a abatir una cierta cantidad de patos, configurable desde el archivo XML del escenario.\\

En esta ocasión, son sólo dos los actores que participan en el juego, siendo uno los patos y el otro el que controla los dos puntos de mira.\\

La clase Escenario, derivada de Nivel, se encarga de generar un pato nuevo a uno de los lados de la pantalla cada cierto tiempo, siendo este periodo un intervalo que va desde el segundo y medio a los tres segundos. El lateral del escenario en el que aparece el pato es aleatorio, al igual que el ángulo que tomará en su vuelo, que puede ir desde -30 grados hasta +30 grados. Si el pato escapa sin ser abatido, se elimina el objeto para no congestionar la memoria, sin embargo, cuando se dispara a un pato, éste queda suspendido durante medio segundo en el lugar del impacto (con una animación que indica que ha sido alcanzado), y después cae verticalmente hacia la zona inferior de la pantalla.\\

Por otro lado, las coordenadas de los puntos de mira equivalen al puntero infrarrojo de su correspondiente mando asociado, y un pato será derribado cuando un punto de colisión que coincida con el puntero de un mando colisione con el pato y además se haya pulsado el botón B del mando. Tras efectuar un disparo, no se puede volver a disparar hasta pasados 0,8 segundos, que es el tiempo que dura la recarga de la escopeta. Esto supone un cambio de estado para el actor del punto de mira, que vuelve al estado normal tras agotarse el periodo de tiempo indicado.\\

Al desarrollar este juego se realizaron modificaciones a la clase Nivel de \programa{LibWiiEsp}, ya que los patos no se dibujaban bien cuando se acercaban al lateral izquierdo de la pantalla. El error consistía en que, para determinar si un pato estaba en pantalla o no, se tomaba únicamente la referencia de su punto superior izquierdo, en lugar de utilizar ambos laterales del actor. Al solucionar este error, se aprovechó para solucionar la misma situación que ocurría con los componentes del mapa de \emph{tiles}, de tal manera que se dibujara correctamente el lateral izquierdo de un escenario compuesto por \emph{tiles}.\\

En la figura \ref{duck} puede observarse una captura de pantalla de este juego.\\

\figura{duck.png}{scale=0.6}{Captura de pantalla de Duck Hunt}{duck}{h}

\subsection{Wii Pang}

El tercer y último juego completo de ejemplo que acompaña a \programa{LibWiiEsp} es Wii Pang, una versión sencilla del clásico Pang publicado por Mitchell Co. en 1989. En este juego controlamos a un personaje que puede moverse horizontalmente sobre el escenario, hasta tocar las paredes laterales de éste, y que debe evitar ser aplastado por unas bolas que rebotan con los límites del nivel. Para destruir las bolas, puede lanzar ganchos, que van subiendo hacia la parte superior de la pantalla con una dirección vertical desde el punto donde se han lanzado. Cuando un gancho colisiona con una bola, ésta se deshace, produciendo que dos bolas de un tamaño inferior aparezcan en su lugar, así hasta que se destruyen las bolas de tamaño más pequeño, que no generan ninguna otra nueva.\\

Las bolas tienen un movimiento horizontal con velocidad constante, y su movimiento vertical se ve modificado por una aceleración de gravedad, de tal manera que resulta una parábola. Para implementar este movimiento, una bola que caiga verá incrementada su velocidad vertical cada fotograma, hasta que colisione con el suelo, momento en el que se le asigna su velocidad inicial, pero hacia arriba. Esta velocidad va descendiendo hasta que llega a cero, momento en el que alcanza el punto más alto de rebote, y de nuevo comienza a bajar, ya que la velocidad cambia de signo y comienza a aumentar de nuevo en función de la aceleración de gravedad. Cada tamaño de bola llegará a una altura determinada por una velocidad inicial diferente del resto de tamaños de bola.\\

El personaje podrá lanzar dos ganchos simultáneos. Estos ganchos empiezan siendo un objeto pequeño a nivel del suelo del escenario, pero cada fotograma van creciendo de tamaño hacia arriba hasta que llegan a tocar la parte superior del escenario, momento en el que desaparecen. Si colisionan con una bola, desaparecen también. El método de dibujo de este actor recorre toda la extensión vertical de éste, de arriba hacia abajo, dibujando en cada paso una parte del cable que acompaña al gancho. También su figura de colisión asociada va aumentando de tamaño a medida que crece.\\

Otro detalle interesante es el movimiento del personaje cuando es aplastado por una bola, momento en el que, con una animación de muerte, inicia un movimiento parabólico similar (pero no igual) al de las bolas, rebotando con el lateral del escenario y después cayendo hacia fuera de la pantalla.\\

Un nivel acaba cuando no quedan más bolas en el escenario, y cuando el personaje muere, tiene que volver a empezar el nivel desde el principio. Al igual que ocurre con Arkanoid, resulta muy sencillo añadir escenarios nuevos al juego.\\

Respecto a las modificaciones implementadas a \programa{LibWiiEsp} en base al desarrollo de este juego, el principal cambio constó en dar la posibilidad de redefinir el método de dibujo de los actores, ya que los ganchos son actores cuyo tamaño no se mantiene constante durante la ejecución del programa. También hubo que añadir un método nuevo para saber si un actor colisiona con los cuatro bordes del escenario, ya que el que había hasta ese momento disponible únicamente indicaba si un actor chocaba con un componente del escenario.\\

En la figura \ref{pang} puede observarse una captura de pantalla de este juego.\\

\figura{pang.png}{scale=0.6}{Captura de pantalla de Wii Pang}{pang}{h}

Hay que mencionar que las capturas de pantalla de los tres juegos no tienen una alta calidad por haberse realizado mediante fotografías disparadas a la pantalla de una televisión CRT (es decir, de tubo de rayos catódicos).



