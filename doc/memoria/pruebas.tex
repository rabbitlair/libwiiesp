% Este archivo es parte de la memoria de libWiiEsp, protegida bajo la 
% licencia GFDL. Puede encontrar una copia de la licencia en el archivo fdl-1.3.tex

% Fuente tomada de la plantilla LaTeX para la realización de Proyectos Final 
% de Carrera de Pablo Recio Quijano.

% Copyright (C) 2009 Pablo Recio Quijano
% Copyright (C) 2011 Ezequiel Vázquez de la Calle

% -*-pruebas.tex-*-

\section{Plan de pruebas y validación del sistema}

El plan de pruebas se ha realizado durante el desarrollo de la biblioteca y una vez se finalizó el producto, y ha incluido los siguientes tipos de pruebas:

\begin{itemize}
\item \textbf{Pruebas de clase}: Tras completar el desarrollo de cada una de las clases o módulos que componen\emph{LibWiiEsp} se procedió a realizar una serie de pruebas concretas para el apartado correspondiente.
\item \textbf{Pruebas de sistema}: Una vez completada la biblioteca, se realizaron pruebas de conjunto, en las que se utilizaban prácticamente todos los módulos desarrollados, y en los que se comprobó la cohesión entre todas las partes del sistema.
\item \textbf{Pruebas de \emph{makefile}}: Este apartado merece una mención aparte, y es que el archivo \emph{makefile} instala, desinstala y empaqueta para la distribución a la biblioteca, además de realizar una recompilación eficiente.
\item \textbf{Pruebas de juegos}: Las pruebas relacionadas con los tres juegos de ejemplo desarrollados se realizaron al completar éstos.
\end{itemize}

\section{Especificación del diseño de pruebas}

\begin{itemize}
\item \textbf{Pruebas de clase}: Estas pruebas se fueron realizando durante el desarrollo, y se han ido comentando en los puntos correspondientes a cada uno de los módulos implementados. A modo de vista general, estas pruebas, en lo general comprobaciones de \emph{caja negra} y  \emph{caja blanca}, han ido enfocadas a probar cada uno de los métodos implementados en cada clase, de tal manera que se pudieran comprobar los resultados para datos de entrada válidos y no válidos.
\item \textbf{Pruebas entre módulos}: Una vez se finalizó el desarrollo, la batería de pruebas realizada en ese momento estuvo enfocada a comprobar las relaciones entre distintas clases que deben trabajar haciendo uso las unas de las otras.\\
Concretamente, las pruebas de sistema se realizaron atendiendo a las siguientes relaciones entre clases:
	\begin{itemize}
	\item Screen (\ref{labelvideo}), Imagen (\ref{labelrecursos}) y Animación (\ref{labelanimaciones}): pruebas orientadas a comprobar que todas las funciones de dibujo de texturas operaban correctamente.
	\item Screen (\ref{labelvideo}) y Fuente (\ref{labelrecursos}): conjunto de validaciones que evaluaban la escritura de textos en la pantalla, haciendo uso de fuentes de texto y caracteres con diversas codificaciones.
	\item Sonido y Música (\ref{labelrecursos}): pruebas que comprobaron que la reproducción de pistas de música y efectos de sonido se producía correctamente, sin provocar errores en el procesador DSP.
	\item Sdcard (\ref{labeltarjeta}) y los recursos multimedia (\ref{labelrecursos}): esta batería de pruebas comprobó que las clases que gestionan los recursos multimedia realizaban un uso acceso correcto a la tarjeta SD.
	\item Sdcard (\ref{labeltarjeta}) y Parser (\ref{labelparser}): se realizaron pruebas para verificar que el parser XML trabajaba correctamente al cargar archivos desde la tarjeta SD.
	\item Sdcard (\ref{labeltarjeta}) y Logger (\ref{labelmensajes}): pruebas que comprobaban la correcta escritura del archivo de registro de mensajes en la tarjeta SD.
	\item Parser (\ref{labelparser}), Galería (\ref{labelgaleria}), Lang (\ref{labelidiomas}), Actor, Nivel y Juego (\ref{labelplantillas}): estas pruebas estaban orientadas a comprobar que las clases que utilizan XML lo hacían de una forma correcta.
	\item Galería (\ref{labelgaleria}) y recursos multimedia (\ref{labelrecursos}): este conjunto de pruebas se encargó de demostrar que la galería de recursos operaba correctamente con las clases que gestionan éstos.
	\item Actor y Nivel (\ref{labelplantillas}): multitud de pruebas se encargaron de validar el tratamiento que da la clase Nivel a los distintos Actores.
	\item Nivel y Juego (\ref{labelplantillas}): pruebas enfocadas a validar el tratamiento que la clase Juego da a los objetos de clases derivadas de Nivel.
	\item Juego (\ref{labelplantillas}) y Mando (\ref{labelmandos}): conjunto de validaciones que comprobaban la cohesión entre la clase Juego y los distintos objetos de la clase Mando.
	\end{itemize}
\item \textbf{Análisis estático de código}: Además de comprobar la cohesión entre los distintos módulos que componen \programa{LibWiiEsp}, se utilizó la herramienta \programa{CppCheck} para validar todo el código fuente generado de forma estática. Este software realiza comprobaciones sobre los archivos fuente, asegurando que no ocurran fallos típicos como fugas de memoria, variables no inicializadas, reservas de memoria no utilizadas, punteros nulos, uso de funciones obsoletas, así como varios errores comunes sobre el uso del lenguaje y la biblioteca estándar de plantillas (STL).
\item \textbf{Pruebas de \emph{makefile}}: estas pruebas se realizaron a lo largo de todo el desarrollo, ya que han habido muchas ocasiones en las que se ha añadido funcionalidad al archivo de recompilación. Las comprobaciones realizadas se realizaron objetivo por objetivo.
\item \textbf{Pruebas de juegos}: la batería de pruebas de juegos se condensaron en un plan intensivo de \emph{testing}, en el que varias personas se dedicaron a jugar a las tres demostraciones para comprobar su funcionamiento.
\end{itemize}

\section{Especificación de los casos de prueba}

A la hora de realizar las comprobaciones por módulos, se disponía de unos datos de entrada que se proporcionaban a los métodos probados de la clase correspondiente. Se probaron todos los métodos de todas las clases. Esta entrada incluía tanto información válida, de la que se había previamente calculado el resultado que debían producir, como datos no válidos, que debían generar una excepción controlada y su correspondiente entrada en el registro de mensajes del sistema.\\

Las pruebas de sistema estuvieron basadas en esta misma metodología, y supusieron una ampliación de los casos de prueba para módulos individuales. Se hizo especial hincapié en el intercambio de datos entre distintas clases, en forma de paso de parámetros, mensajes o activación y desactivación de banderas. Las comprobaciones se realizaban tanto a la salida de un primer módulo como a la entrada del siguiente en la ejecución. La validación con \programa{CppCheck} se realizó con la opción de activar todas las comprobaciones, y se lanzó para las carpetas \emph{include} y \emph{src}.\\

Las pruebas realizadas al archivo \emph{makefile} se fueron realizando a medida que se desarrollaba nueva funcionalidad en él, y consistieron en comprobar que los comandos que se ejecutaban realizaban su tarea correctamente.\\

Respecto a las pruebas realizadas a los juegos, se realizaron por parte de personas que ya tenían experiencia en los juegos clásicos en los cuales se basan las demostraciones de \emph{LibWiiEsp}, y se centraron principalmente en la detección de comportamientos anómalos de los juegos.

\section{Documentación de la ejecución de las pruebas}

La batería de pruebas individual de cada una de las clases del sistema arrojó, para algunas de ellas, diversos fallos de baja consideración, que fueron resueltos en poco tiempo. Principalmente, dieron problemas las funciones de dibujo de texturas, ya que se producían fallos al dibujarlas en la pantalla debido a errores en la creación del formato RGB5A3. Los efectos de sonido no se reproducían correctamente, al no estar bien tratados con el programa SoX \cite{website:sox}, pero esto se resolvió leyendo en profundidad la documentación del software. Otra situación redundante ocurría con errores de memoria, en los que los punteros no se estaban gestionando correctamente, pero que pudo resolverse tras utilizar la herramienta \programa{CppCheck}. En general, prácticamente todos los módulos del sistema necesitaron ajustes tras realizarse las pruebas.\\

Aparte de los típicos errores por descuido a la hora de implementar el código, no ocurrieron incidentes reseñables durante el desarrollo de las pruebas.

