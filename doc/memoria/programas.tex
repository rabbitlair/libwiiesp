% Este archivo es parte de la memoria de libWiiEsp, protegida bajo la 
% licencia GFDL. Puede encontrar una copia de la licencia en el archivo fdl-1.3.tex

% Fuente tomada de la plantilla LaTeX para la realización de Proyectos Final 
% de Carrera de Pablo Recio Quijano.

% Copyright (C) 2009 Pablo Recio Quijano
% Copyright (C) 2011 Ezequiel Vázquez de la Calle

% -*-programas.tex-*-

\section*{Doxygen}

\programa{Doxygen} \cite{website:doxygen} realiza una documentación automática de código fuente. Puede generar esta documentación en varios formatos, incluyendo HTML y \LaTeX{}.

\section*{Dia}

\programa{Dia} es un editor de gráficos vectoriales el cual incluye distintas plantillas para distintos tipos de gráficos, como pueden ser UML, ERe, diagramas de flujo, esquemas Cisco de red y un larguísimo etcétera.

\section*{GNU Make}

\programa{GNU Make} \cite{website:make} \cite{pdf:make} es el programa de recompilación y de control de dependencias por excelencia. Se puede utilizar para compilar proyectos software en diversos códigos.

\section*{Tiled}

\programa{Tiled} \cite{website:tiled} es un editor de mapas de \emph{tiles} que permite construir escenarios para juegos en dos dimensiones. La gran ventaja que proporciona este software consiste en que exporta el escenario construido en formato XML, facilitando enormemente el proceso de carga y utilización de los niveles que se diseñen utilizando esta herramienta. Está licenciado con GPLv2.

\section*{Homebrew Channel}

\programa{Homebrew Channel} \cite{website:hbc} es una aplicación no oficial para Nintendo Wii cuya principal funcionalidad es la de permitir la ejecución de software sin firmar digitalmente por Nintendo. Aunque no se ha publicado su código fuente, es de distribución gratuita, y su gran ventaja consiste en que utiliza el lector de tarjetas SD de la videoconsola como medio de almacenamiento para los ejecutables. Tiene un aspecto gráfico realmente cuidado.

\section*{SoX}

\programa{SoX} \cite{website:sox} es una aplicación libre, que se ejecuta en línea de comandos, y que está considerada como la \emph{navaja suiza de los formatos de sonido}. Permite cambiar prácticamente todos los aspectos de un archivo de audio, pudiendo trabajar con multitud de formatos. También puede aplicar varios efectos a estos archivos de sonido, e incluso reproducirlos.

\section*{Subversion}

\programa{Subversion} es un sistema de control de revisiones de código libre, y que permite gestionar eficiente y cómodamente los cambios aplicados a un proyecto por uno o varios desarrolladores.

\section*{DevKitPro}

\programa{DevKitPro} \cite{website:devkitpro} es un conjunto de compiladores, bibliotecas estándar de C y C++ y varias utilidades que permiten compilar código fuente escrito en esos lenguajes para poder ser ejecutado en varias videoconsolas actuales. Concretamente, la versión para Nintendo Wii que he utilizado se denomina \programa{DevKitPPC}.

\section*{LibOgc}

\programa{LibOgc} \cite{website:libogc} es una biblioteca de muy bajo nivel, escrita en C, y que permite acceder a todo el hardware de la videoconsola Nintendo Wii. Está desarrollada por varios \emph{sceners}, y trabaja junto con la herramienta \programa{DevKitPPC}.

\section*{LibFat}

Es una biblioteca libre que permite realizar operaciones con particiones cuyo sistema de ficheros sea FAT o FAT32. Se ha utilizado una versión adaptada para trabajar con Nintedo Wii.

\section*{LibFreeType2}

Biblioteca libre con la que se pueden utilizar todo tipo de fuentes de texto en las aplicaciones, ya que proporciona una interfaz completa para cargarlas en memoria a partir de un archivo y trabajar con ellas. Se trata de una versión adaptada a Nintendo Wii.

\section*{TinyXML}

\programa{TinyXml} es una sencilla biblioteca libre, adaptada para trabajar con Nintendo Wii, y cuya utilidad reside en poder operar fácilmente con árboles XML cargados desde un archivo de datos que utilice este formato de almacenamiento.

\section*{\LaTeX}

\LaTeX{} \cite{casc03} es un sistema de composición de textos, orientado especialmente a la creación de libros, documentos científicos y técnicos que contengan fórmulas matemáticas. Se ha utilizado para la redacción de este documento y del manual de instalación y uso de la biblioteca. \LaTeX{} es software libre bajo licencia LPPL.

\section*{Gimp}

\programa{Gimp} \cite{tril07} es un programa de edición de imágenes digitales en forma de mapa de bits, tanto dibujos como fotografías. Es un programa libre y gratuito. Forma parte del proyecto GNU y está disponible bajo la Licencia pública general de GNU.

\section*{Gantt Project}

\programa{Gantt Project} es un software escrito en Java y libre, cuyo propósito es el de realizar diagramas de Gantt que reflejen la planificación temporal de un proyecto.

\section*{Cppcheck}

\programa{Cppcheck} es una utilidad libre que se encarga de realizar un exhaustivo análisis estático de cualquier código escrito en C++. Indica desde posibles fugas de memoria hasta el uso de funciones obsoletas, pasando por analizar el uso de la biblioteca estándar de plantillas y la seguridad del sistema de excepciones.


