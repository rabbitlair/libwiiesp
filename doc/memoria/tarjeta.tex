% Este archivo es parte de la memoria de libWiiEsp, protegida bajo la 
% licencia GFDL. Puede encontrar una copia de la licencia en el archivo fdl-1.3.tex

% Fuente tomada de la plantilla LaTeX para la realización de Proyectos Final 
% de Carrera de Pablo Recio Quijano.

% Copyright (C) 2009 Pablo Recio Quijano
% Copyright (C) 2011 Ezequiel Vázquez de la Calle

% -*-tarjeta.tex-*-

Una vez cubierto el requisito de controlar el sistema de vídeo, era momento de investigar sobre el funcionamiento del lector de tarjetas de Nintendo Wii.

\subsection{Módulo \emph{wiisd\_io} y \programa{libFat}}

\programa{Libogc} proporciona acceso al dispositivo hardware del lector de tarjetas a través de la variable global \emph{\_\_io\_wiisd}, la cual está incluida en el fichero de cabecera \emph{sdcard/wiisd\_io.h}. Esta variable global es una instancia de la estructura DISC\_INTERFACE, incluida en la cabecera \emph{ogc/disc\_io.h}, y que se utiliza como interfaz única para todos los dispositivos de almacenamiento de los que dispone la videoconsola. La estructura está formada por dos variables que contienen información (indican el tipo de dispositivo) y otras seis, que son punteros a función. Estas seis funciones se definen para cada tipo de dispositivo, y son las siguientes:

\begin{itemize}
\item \textbf{startup}: inicializa el dispositivo.
\item \textbf{isInserted}: comprueba si el medio de almacenamiento está disponible en el lector correspondiente.
\item \textbf{readSectors}: lee un grupo de sectores desde el dispositivo a un flujo de datos.
\item \textbf{writeSectors}: escribe un flujo de datos en un grupo de sectores.
\item \textbf{clearStatus}: indica si el dispositivo está listo para realizar una operación.
\item \textbf{shutdown}: apaga el dispositivo.
\end{itemize}

Cuando el dispositivo esté listo para ser accedido, hay que montar la partición en la que se encuentre la información que se quiere cargar. Las tarjetas de memoria SD trabajan con el sistema de ficheros FAT o FAT32, por lo que es necesario utilizar \programa{libFat} adaptada para trabajar con Nintendo Wii. La interfaz de esta biblioteca es muy sencilla, ya que se puede montar un sistema de ficheros FAT con la función \emph{fatMountSimple}, que recibe como parámetros el nombre que queremos asignarle a la unidad montada, y la estructura DISC\_INTERFACE correspondiente al dispositivo (en el caso de la tarjeta SD será, como se comentó anteriormente, la variable \emph{\_\_io\_wiisd}). Desmontar una partición que esté montada es tan sencillo como llamar a la función \emph{fatUnmount}, indicándole el nombre de la unidad montada.\\

Hay que tener en cuenta una limitación de este sistema, y es que sólo es accesible la primera partición de la tarjeta SD; por tanto, se debe tener especial cuidado de que la tarjeta tenga su primera partición formateada con FAT o FAT32.

\subsection{Identificación de funcionalidad necesaria}

Tras revisar toda la documentación y las cabeceras de los archivos implicados, hay que definir de una forma clara qué se quiere conseguir para cumplir el requisito de acceder al lector de tarjetas. Básicamente, se necesita poder montar una partición FAT/FAT32 que se encuentre en la tarjeta SD, asegurando que quede accesible desde cualquier punto de la aplicación, y también poder desmontar esta partición.\\

Por otro lado, no estaría de más poder conocer en todo momento si la partición está montada o no, y el nombre asignado que tiene la unidad montada.

\subsection{Diseño e implementación}

Al igual que ocurre con el sistema gráfico de la videoconsola, lo más adecuado para tratar el acceso al lector de tarjetas es implementar una clase con el patrón \emph{Singleton}, ya que sólo existe en el hardware un lector de tarjetas (al contrario ocurre con los dos puertos USB traseros o las dos ranuras para tarjetas de memoria de \emph{Game Cube}, pero no es el caso que nos atañe). Por otro lado, la implementación del mencionado patrón debe dejar accesible la instancia de la clase desde cualquier punto del sistema.\\

La clase tendrá un método de inicialización, en el que reciba el nombre que se quiere asignar a la unidad una vez montada. Dicho método intentará inicializar el dispositivo del lector de tarjetas (utilizando la función \emph{startup} de la estructura descrita antes) y, en caso de tener éxito la operación, tratará de montar la primera partición que se encuentre en la tarjeta SD con un sistema de ficheros FAT. Si algo saliera mal, el proceso se repetirá hasta 10 veces. Superado ese número de intentos sin éxito, el programa interrumpiría su ejecución, saliendo con un código de error 1.\\

Además, la clase contará con métodos consultores sobre una bandera (variable booleana) que indique si la partición está montada o no, y sobre el nombre asignado a la unidad. Por último, se proporcionará también un método que intente desmontar la partición, en caso de que se encuentre activa.\\

Para acceder al sistema de ficheros montado mediante esta clase, basta con utilizar las funciones estándar de lectura y escritura de archivos. Es recomendable que siempre se empleen rutas absolutas, precedidas del nombre de la unidad seguida de los dos puntos (:). Por ejemplo, sabiendo que la unidad se llama "SD", para cargar un archivo cuya ruta sea \emph{/directorio/archivo.bmp} se podría utilizar el siguiente código:\\

\begin{lstlisting}[style=C++]
  ifstream archivo;
  archivo.open("SD:/directorio/archivo.bmp", ios::binary);
\end{lstlisting}

Por último, en la figura \ref{sdcard} puede observarse la interfaz pública de la clase:\\

\figura{sdcard.png}{scale=0.8}{Interfaz pública de la clase Sdcard}{sdcard}{h}

\subsection{Pruebas}

Las pruebas realizadas para validar esta clase fueron sencillas, ya que consistieron en un programa que utilizaba  el subsistema de consola de \programa{libogc} para ir imprimiendo mensajes a medida que se intentaba acceder al hardware del lector de tarjetas y, una vez activado, realizar el montaje y desmontaje de la partición. También se incluyó en el programa de prueba la apertura y escritura en un archivo de texto plano.

