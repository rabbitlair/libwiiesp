% Este archivo es parte de la memoria de libWiiEsp, protegida bajo la 
% licencia GFDL. Puede encontrar una copia de la licencia en el archivo fdl-1.3.tex

% Fuente tomada de la plantilla LaTeX para la realización de Proyectos Final 
% de Carrera de Pablo Recio Quijano.

% Copyright (C) 2009 Pablo Recio Quijano
% Copyright (C) 2011 Ezequiel Vázquez de la Calle

% -*-descripciongeneral.tex-*-

En este capítulo se profundiza en el proyecto, definiendo algunos de los aspectos más importantes de \programa{LibWiiEsp}.

\section{Perspectiva del producto}

La biblioteca es un producto software independiente, es decir, no forma parte de ningún otro proyecto de mayor envergadura, aunque sí depende de las herramientas de bajo nivel proporcionadas por los \emph{sceners}. La herramienta más importante de éstas, y sobre la que trabaja \programa{LibWiiEsp}, es la biblioteca \programa{libogc}. Esta biblioteca, desarrollada mediante ingeniería inversa, ofrece acceso a todo el hardware de Wii, pero realiza las operaciones a muy bajo nivel, resultando relativamente incómoda para el programador.\\

\programa{LibWiiEsp} no necesita comunicarse con otros sistemas, y carece de interfaz gráfica de usuario. Las interfaces de que dispone son de tipo hardware, y consisten en el acceso a los distintos componentes de la videoconsola, como son los mandos (vía \emph{Bluetooth}), la tarjeta SD para el almacenamiento, y sus distintos subsistemas.\\

Un programa desarrollado con \programa{LibWiiEsp} podrá acceder al sistema gráfico de la consola, al de sonido, a la tarjeta SD y podrá leer el estado de hasta cuatro mandos conectados permanentemente al aparato. En su momento se decidió dejar para más adelante la posibilidad de acceder al sistema de \emph{WiFi} de la videoconsola, así como a los puertos \emph{USB} traseros y las ranuras para tarjetas de memoria de \emph{Game Cube}; en futuras versiones se añadirán las estructuras necesarias para ello. De todas formas, sigue siendo posible acceder a estos subsistemas mediante las funciones que proporciona \programa{libogc}.\\

Un detalle más a comentar, es que la biblioteca está pensada para hacer juegos en dos dimensiones, pero en una primera versión las plantillas pueden resultar incómodas para trabajar con un videojuego que no tenga utilice una perspectiva ortográfica.

\section{Requisitos}

A continuación se describen de forma general los requisitos que deberá cumplir la biblioteca para hacer posible el desarrollo de videojuegos completos.

\subsection{Requisitos funcionales}

En este apartado se describe la funcionalidad que incluye \programa{LibWiiEsp}, mencionando los distintos módulos que se aportan al usuario:

\begin{itemize}
\item \textbf{Controlar el sistema de vídeo}: el aspecto más importante de un videojuego es su visualización, y hay que conseguir reproducir gráficos en la pantalla a partir de las escasas herramientas que proporciona \programa{libogc}. Hay que facilitar el uso de texturas creadas a partir de una imagen y de formas geométricas con colores planos. Además, hay que proporcionar un sistema de doble \emph{búffer} para evitar posibles problemas a la hora de dibujar gráficos y crear un mecanismo para finalizar cada fotograma, así como optimizar el rendimiento de todo el sistema gráfico de la videoconsola.
\item \textbf{Acceder al lector de tarjetas SD}: se necesita garantizar el acceso a un medio de almacenamiento desde donde poder cargar los recursos y los datos que sean necesarios, en este caso, se ha seleccionado el lector de tarjetas SD por existir una mayor compatibilidad de las tarjetas con el software que permite ejecutar los videojuegos creados con \programa{LibWiiEsp}, el \programa{Homebrew Channel} \cite{website:hbc}.
\item \textbf{Utilizar hasta cuatro mandos}: para no limitar la creación de los usuarios a juegos de un único jugador, es importante aportar la posibilidad de utilizar todos los mandos que llega a soportar Nintendo Wii, que son cuatro a la vez.
\item \textbf{Utilizar recursos multimedia}: se proporcionará la posibilidad de cargar imágenes para crear texturas, efectos de sonido, pistas de música y fuentes de texto desde el medio de almacenamiento elegido, el lector de tarjetas. Una vez cargado en memoria el contenido multimedia, la utilización de éstos debe ser lo más sencilla posible. Ha de permitirse dibujar una textura o una parte de ella, reproducir varios efectos de sonido a la vez, reproducir una pista de música y escribir texto con cualquier carácter soportado por la fuente de texto seleccionada.
\item \textbf{Organizar recursos multimedia}: todo el contenido multimedia debe organizarse de una forma que resulte sencilla, para facilitar el acceso a estos recursos desde cualquier punto de la aplicación.
\item \textbf{Soporte de internacionalización (i18n)}: en un mundo globalizado como el actual, es muy importante que una aplicación (en este caso, un videojuego) pueda obtenerse en múltiples idiomas, no sólo en el del creador. Se proporcionará un soporte de idiomas sencillo de utilizar, pero suficientemente potente y eficiente, además de facilitar la adición de idiomas nuevos.
\item \textbf{Reproducir animaciones}: a partir de una textura que contenga un \emph{spritesheet} se debe poder dibujar en la pantalla la animación determinada por los distintos cuadros de la rejilla de la imagen.
\item \textbf{Registro de mensajes del sistema}: debido a las dificultades que entraña la depuración de código en la plataforma Nintendo Wii, será útil ofrecer un sistema de registro de mensajes con distintos niveles (avisos, errores\ldots) para hacer menos compleja la detección de errores y creación de \emph{logs} del sistema.
\item \textbf{Detectar las colisiones}: otro aspecto importantísimo de la creación de videojuegos es la detección de colisiones. Para ello, \programa{LibWiiEsp} tendrá un módulo especializado en detectar colisiones entre formas geométricas planas, de forma eficiente y fácilmente escalable.
\item \textbf{Creación de actores}: en un videojuego, todos los elementos no estáticos son llamados actores, y aunque puede darse una gran variedad de comportamientos en ellos, hay varios aspectos comunes en todos. Se dará la posibilidad de utilizar una plantilla (una clase abstracta) que recoja todos estos detalles compartidos entre los distintos actores, y reduzca la complejidad de creación de éstos.
\item \textbf{Diseño de escenarios}: tan importante como los actores es el escenario en el que se desarrolla el juego. Se debe facilitar la creación de escenarios basados en mapas de \emph{tiles}, y que resulten sencillos de cargar en el programa.
\item \textbf{Inicialización de la videoconsola}: todos los subsistemas de la consola que se emplean al ejecutar un videojuego necesitan ser inicializados y configurados antes de entrar en funcionamiento, así que se aportará una forma cómoda de realizar estas operaciones.
\end{itemize}

\subsection{Requisitos de interfaces externas}

Como ya se ha comentado, se necesita acceder al lector de tarjetas SD y montar la partición en la que se encuentran los recursos multimedia.\\

Por otro lado, es necesaria una separación efectiva de código fuente y datos, de tal manera que se puedan añadir niveles a un juego, modificar los parámetros de configuración, añadir idiomas nuevos y ajustar la información de los distintos actores que participan en el videojuego sin necesidad de recompilar cada vez que se requiera un cambio. Para ello, lo más adecuado es utilizar el formato XML, por lo que es imprescindible que \programa{LibWiiEsp} ofrezca una manera sencilla y eficiente de trabajar con este tipo de archivos.

\subsection{Requisitos de rendimiento}

Respecto al rendimiento, es muy importante cuidar el rendimiento de los distintos módulos de la biblioteca, ya que, por una parte, el hardware de la Nintendo Wii es relativamente limitado en comparación con las otras dos videoconsolas de su misma generación (\emph{Microsoft Xbox 360} y \emph{Sony PlayStation 3}). Concretamente, Wii dispone de un procesador de 729 MHz, 64 megabytes de memoria principal y 24 de memoria de vídeo, debido a ello hay que optimizar el máximo posible el producto.

\subsection{Atributos del sistema software}

Las características indispensables que deben conseguirse con el desarrollo del sistema son la fiabilidad, escalabilidad y mantenibilidad.\\

La biblioteca debe ser completamente fiable, ya que la depuración es un proceso bastante complejo en la Nintendo Wii. Hay que conseguir que todos los componentes de \programa{LibWiiEsp} respondan con amplia fiabilidad, y no provoquen errores, de tal manera que el usuario pueda estar seguro de que es su programa el que produce el fallo, en caso de que lo hubiera.\\

Es deseable conseguir la característica de la escalabilidad, debido a que \programa{LibWiiEsp} cubrirá una parte importante de las operaciones que realiza un videojuego en dos dimensiones, pero hay otras que no tocará. Estas partes son, entre otras, las conexiones de red y la detección de colisiones (siempre es bueno aumentar la cantidad de figuras de colisión). Para obtener esta propiedad, es necesario que desde un principio se guarde cuidado respecto a otra característica como es la mantenibilidad: si \programa{LibWiiEsp} va a crecer con el tiempo (y espero que así sea) es muy importante que pueda ser mantenida y corregida de una manera sencilla.

\section{Características de los usuarios}

\programa{LibWiiEsp} está orientada a usuarios con un mínimo conocimiento sobre el desarrollo de videojuegos en dos dimensiones, e incluso podría llegar a utilizarse como herramienta de iniciación en este campo, pero esto último no se recomienda. Por supuesto, es una herramienta con suficiente potencia como para desarrollar videojuegos de una complejidad media-alta, por lo que también es adecuada para programadores que tengan una amplia experiencia en el campo. Al estar implementada en C++ y utilizando el paradigma de la orientación a objetos, se aconseja a los posibles usuarios que adquieran previamente una base de conocimiento suficiente en los dos aspectos mencionados.

\section{Restricciones generales}

La programación para Nintendo Wii tiene implícitas ciertas restricciones relacionadas con la arquitectura hardware concreta de la videoconsola. A continuación se enumeran estas limitaciones:

\begin{itemize}
\item \textbf{Big Endian}: el procesador de Nintendo Wii trabaja con una representación de la información en memoria con \emph{Big Endian}, al contrario que las plataformas de arquitectura Intel. Esto obliga a que se compruebe la estructura de toda información que se cargue desde un medio de almacenamiento externo y que se haya tratado con un sistema \emph{Little Endian} (como por ejemplo, cualquier ordenador personal de hoy en día). Para una explicación gráfica sobre las diferencias de representación entre los distintos tipos de \emph{Endian}, ver figura \ref{endianess}.

\figura{endianess.png}{scale=0.6}{Diferencia entre las representaciones \emph{Big Endian} y \emph{Little Endian}}{endianess}{h}

\item \textbf{Alineación de los datos}: otra limitación referente a la arquitectura en la que está construida Wii consiste en que cada dato en memoria debe estar alineado a su tamaño. Así, un entero de 32 bits sólo podrá encontrarse en las posiciones 0, 4, 8, etc., un carácter (8 bits) podrá encontrarse en cualquier posición \ldots Si, por ejemplo, reservamos memoria para un carácter, y justo después para un entero de 32 bits, entonces el procesador rellenará las tres posiciones de memoria que restan hasta la cuarta con ceros, produciéndose una pérdida de espacio considerable si no se cuida este detalle. En la figura \ref{alineacion} puede observarse el estado de la memoria si no se respeta esta indicación, comparando la situación con una declaración de variables adecuada.

\figura{alineacion.png}{scale=0.6}{Ejemplo sobre la necesidad de alineación de los datos}{alineacion}{h}

\item \textbf{Relleno y alineación al leer desde un periférico}: el mecanismo de la caché de lectura desde periféricos que utiliza Nintendo Wii emplea un \emph{búffer} de 32 bytes, que requiere que la zona de memoria principal donde se escribe esté alineada a esta cantidad, y que además tenga un tamaño exactamente múltiplo de 32 bytes. De no respetarse esta limitación, se podrían producir errores de pérdida de datos por \emph{machacamiento}, ya que la lectura se hace en bloques de 32 bytes alineados exactos, sin comprobar que esos 32 bytes pertenezcan al mismo fichero. Es decir, puede darse el caso (bastante común) de que la información de un archivo tenga su parte inicial en una zona de memoria, y el resto en otra distinta, lo cual provocaría un error.
\item \textbf{Tipos de datos}: A la hora de programar en la consola Wii deben utilizarse los tipos de datos definidos por \programa{libogc}, que incluyen variantes para todo tipo de datos numéricos (enteros con y sin signo, y decimales de coma flotante). Para el resto de datos (booleanos, caracteres, etc.) no existen redefiniciones de tipos.
\item \textbf{Cantidad de memoria}: un aspecto importante a tener en cuenta a la hora de utilizar \programa{LibWiiEsp} es la limitación de memoria que tiene la videoconsola. La Nintendo Wii dispone sólo de 64 megabytes de memoria principal, por lo que es obligatorio que el usuario tenga especial cuidado de no sobrecargar la consola (por ejemplo, cargando demasiadas pistas de música).
\end{itemize}

\section{Suposiciones y dependencias}

Toda la funcionalidad descrita anteriormente precisa de una serie de condiciones y dependencias que, si bien no son demasiadas, son todas necesarias para la construcción y ejecución de un videojuego desarrollado con \programa{LibWiiEsp}.\\

En la propia Nintendo Wii se debe tener instalado algún método que permita la ejecución de código sin firmar digitalmente por el fabricante, siendo la mejor de las posibilidades \programa{Homebrew Channel} \cite{website:hbc}. Este software se instala como un canal en el menú principal de la consola, y permite correr ejecutables almacenados en la tarjeta SD.\\

Para poder utilizar hasta cuatro mandos, éstos deben estar sincronizados permanentemente con la consola; en caso contrario, sólo se reconocerá el mando principal asociado. Para más información sobre cómo sincronizar permanentemente un mando a Nintendo Wii, consultar el manual de la consola \cite{website:manualoficial}.\\

Respecto a la propia herramienta \programa{LibWiiEsp}, ésta tiene varias dependencias de código. Concretamente son las bibliotecas \programa{libogc} (proporciona acceso al hardware de Wii), \programa{libfat} (permite montar y desmontar particiones en los sistemas de archivos FAT y FAT32), \programa{TinyXML} (para operar con archivos de datos en formato XML) y \programa{FreeType2} (proporciona una interfaz para poder trabajar con fuentes de texto). Todas excepto \programa{libogc} necesitan estar portadas (compiladas específicamente) para trabajar con Nintendo Wii. Pueden obtenerse todas desde la forja del proyecto \cite{website:forja}.\\

Otra herramienta, quizá la más importante, es un conjunto de compiladores y bibliotecas de C/C++, modificados para generar ejecutables que puedan ser lanzados en Nintendo Wii. Este conjunto de herramientas y utilidades varias se llama \programa{DevKitPPC}, y se encuentra englobado dentro de un proyecto denominado \programa{DevKitPro} \cite{website:devkitpro} que incluye este paquete y otros dos similares para la programación en \emph{Sony PlayStation Portable} y \emph{Nintendo DS}. \programa{LibWiiEsp} utiliza la versión \emph{r21} de este paquete de herramientas.

