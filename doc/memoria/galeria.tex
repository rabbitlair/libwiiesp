% Este archivo es parte de la memoria de libWiiEsp, protegida bajo la 
% licencia GFDL. Puede encontrar una copia de la licencia en el archivo fdl-1.3.tex

% Fuente tomada de la plantilla LaTeX para la realización de Proyectos Final 
% de Carrera de Pablo Recio Quijano.

% Copyright (C) 2009 Pablo Recio Quijano
% Copyright (C) 2011 Ezequiel Vázquez de la Calle

% -*-galeria.tex-*-

Tras construir un módulo que nos permita trabajar cómodamente con archivos de datos en formato XML y de disponer de cuatro clases para hacer uso de recursos multimedia en el sistema, llega el momento de gestionar eficazmente dichos recursos.

\subsection{Identificación de funcionalidad necesaria}

Se pretende desarrollar un módulo que gestione todos los recursos multimedia existentes en el sistema, teniéndolos localizados desde cualquier punto de éste, y clasificados por tipo. Cada recurso debe estar identificado por un código único dentro de su tipo. Además, para obtener una separación efectiva entre código fuente y datos, se contempla la posibilidad de indicar, mediante un listado en formato XML, la relación de todos los recursos multimedia asociados con su código identificador.

\subsection{Diseño e implementación}

Para satisfacer la funcionalidad arriba indicada, se procede a diseñar una clase que implemente el patrón \emph{Singleton}, ya que sólo necesitamos una instancia que gestione todos los recursos multimedia que se encuentren en el sistema. Además, la utilización de este patrón de diseño nos aporta la ventaja de que la instancia única de esta clase será accesible desde cualquier punto del sistema.\\

La clase dispondrá de un diccionario para cada tipo de recurso: imágenes, efectos de sonido, pistas de música y fuentes de texto. Cada entrada en el diccionario estará formada por una clave (una cadena de caracteres de tipo \emph{string}), que será única en el diccionario, y asociado a ella habrá un puntero al objeto que contenga al recurso.\\

La carga de todos los recursos se realizará desde un método de inicialización que no será el constructor, debido a la implementación del patrón \emph{Singleton}. Dicho método comprobará que la tarjeta SD está accesible, cargará un archivo XML cuya ruta absoluta en la tarjeta reciba por parámetro y recorrerá el documento XML, cargando todos los recursos multimedia que se indiquen en el archivo. Se espera que cada elemento hijo del nodo raíz del documento contenga la información de creación de un recurso, siendo identificado el tipo de recurso por el valor del elemento. A continuación se muestra un ejemplo válido del formato esperado para el documento XML:

\begin{lstlisting}[style=XML]
<?xml version="1.0" encoding="UTF-8"?>
<galeria>
  <imagen codigo="fondo" formato="bmp" ruta="/directorio/fondo.bmp" />
  <musica codigo="rock" volumen="128" ruta="/directorio/rock.mp3" />
  <sonido codigo="bang" volumen="255" ruta="/directorio/bang.pcm" />
  <fuente codigo="arial" ruta="/directorio/arial.ttf" />
</galeria>
\end{lstlisting}

El método inicializador se encargará de llamar al constructor de la clase del recurso, utilizando como parámetros los atributos que acompañan a cada elemento en el archivo XML, y almacenará la dirección del objeto creado junto a su código en el diccionario correspondiente.\\

Para acceder a los recursos, la clase proporciona un método observador para cada tipo de recurso (es decir, para cada diccionario). Estos métodos reciben el código de identificación del recurso que se desea utilizar, y devuelven una referencia constante a él, abstrayendo al usuario de la lógica de punteros. El destructor de la clase se encarga de destruir todos los objetos de recursos almacenados en los diccionarios y liberar la memoria ocupada por éstos.\\

En la figura \ref{galeria} se muestra la interfaz pública de la clase Galeria.\\

\figura{galeria.png}{scale=0.6}{Interfaz pública de la clase Galeria}{galeria}{h}

\subsection{Pruebas}

La clase Galeria se ha probado con varios ficheros XML, tanto en casos de documentos mal formados como válidos, cargando los recursos indicados y utilizándolos en un programa de prueba. También se probó el caso de intentar cargar un recurso que no existe en la tarjeta SD, y el de código identificador repetido.

