% Este archivo es parte de la memoria de libWiiEsp, protegida bajo la 
% licencia GFDL. Puede encontrar una copia de la licencia en el archivo fdl-1.3.tex

% Fuente tomada de la plantilla LaTeX para la realización de Proyectos Final 
% de Carrera de Pablo Recio Quijano.

% Copyright (C) 2009 Pablo Recio Quijano
% Copyright (C) 2011 Ezequiel Vázquez de la Calle

% -*-idiomas.tex-*-

Continuando con las posibilidades que nos ofrece el disponer de un módulo que nos ayude a trabajar cómodamente con el formato de datos XML, y dado que la clase \emph{Fuente} puede escribir en la pantalla cualquier carácter gracias a las cadenas de caracteres anchos (de tipo \emph{wstring}), vamos a aplicar una idea parecida a la de la clase \emph{Galeria}, pero enfocada a proporcionar un buen soporte para la internacionalización de los juegos desarrollados con \emph{LibWiiEsp}.

\subsection{Identificación de funcionalidad necesaria}

Lo que se quiere conseguir con este módulo es que, en lugar de escribir texto en la pantalla directamente desde el código fuente, se asigne una \emph{etiqueta} o código identificador al punto del programa en el que se quiere mostrar el texto. La idea es que, en tiempo de ejecución, se sustituya dicha etiqueta por el texto correspondiente a ella en el idioma que se indique.\\

Un idioma, pues, debe ser un conjunto de textos, escritos todos en la misma lengua, identificado cada texto por un código o nombre de etiqueta. Todos los idiomas deben tener las mismas etiquetas, solo que el valor de la misma etiqueta para distintos idiomas será la misma cadena de caracteres traducida al que corresponda.\\

Para permitir una ampliación sencilla y rápida de los idiomas disponibles en un videojuego (además de separar código fuente y datos), se necesita que todas las etiquetas se definan en un archivo externo que sea cargado en tiempo de ejecución. Lo ideal es utilizar la tecnología XML y el módulo de \programa{LibWiiEsp} que permite trabajar con ella.\\

Se requiere también que sea sencillo cambiar entre idiomas, para lo que parece adecuado tener un idioma marcado como activo en un momento dado, y que se permita conocer qué idioma se encuentra activo en un momento dado, la existencia o no de un idioma, seleccionarlo y alternar entre todos los idiomas disponibles de forma circular.\\

Por último, es necesario que el módulo que se encargue de gestionar el valor de las etiquetas de texto se encuentre disponible desde cualquier punto del sistema, y que exista únicamente una instancia en éste.

\subsection{Diseño e implementación}

Para cubrir la funcionalidad descrita se construye una clase que implementa el patrón \emph{Singleton}, ya que uno de los requisitos del módulo consiste en disponer de una instancia de la clase, y que sea accesible desde cualquier punto del sistema.\\

La estructura de datos diseñada para almacenar un idioma consiste en un diccionario, donde su primera componente es una cadena de caracteres de alto nivel (de tipo \emph{string}) que almacena el nombre o código de una etiqueta de texto (y mediante la cual se identifica la etiqueta), y su segunda componente es otra cadena de caracteres de alto nivel, pero de caracteres anchos (de tipo \emph{wstring}), en la cual se guarda el texto asociado a la etiqueta traducido al idioma correspondiente.\\

Cada idioma se almacena en otro diccionario en la segunda componente de éste, identificado por su nombre (cadena de caracteres anchos \emph{wstring}) en la primera componente del diccionario.\\

El método de inicialización del soporte de internacionalización requiere como parámetro una cadena de caracteres con la ruta absoluta en la tarjeta SD de un archivo XML que contenga la información de las etiquetas de texto para todos los idiomas que se utilizarán. Este archivo se carga en memoria y se recorre, creando una entrada en el diccionario de idiomas para cada uno que encuentre, y guardando cada etiqueta de texto asociada a él. A continuación se muestra un ejemplo de archivo de idiomas soportado:

\begin{lstlisting}[style=XML, texcl=true, escapechar=']
<?xml version="1.0" encoding="UTF-8"?>
<lang>
  <idioma nombre="espa'ñ'ol">
    <tag nombre="PUNTOS" valor="Puntuaci'ó'n" />
    <tag nombre="VIDAS" valor="Vidas" />
    <tag nombre="NIVEL" valor="Nivel" />
  </idioma>
  <idioma nombre="english">
    <tag nombre="PUNTOS" valor="Score" />
    <tag nombre="VIDAS" valor="Lives" />
    <tag nombre="NIVEL" valor="Level" />
  </idioma>
  <idioma nombre="fran'ç'ais">
    <tag nombre="PUNTOS" valor="Points" />
    <tag nombre="VIDAS" valor="Vies" />
    <tag nombre="NIVEL" valor="Niveau" />
  </idioma>
</lang>
\end{lstlisting}

Al leer el contenido de una etiqueta, éste se almacena temporalmente en una cadena de caracteres \emph{string}. Antes de guardarlo en el diccionario del idioma correspondiente, se transforma a \emph{wstring}, para así asegurar que los caracteres son los correctos. La explicación de esta transformación es sencilla: cuando se lee un texto con caracteres anchos (que necesitan un espacio mayor que 1 byte por carácter), cada carácter ancho se 'rompe' en trozos de 1 byte cada uno. Un carácter ASCII normal ocupará el espacio esperado, es decir, 1 byte. Durante la transformación se crea una cadena de caracteres anchos (de tipo \emph{wchar\_t}), y se rellena con el valor de la etiqueta después de pasarlo por la función de C llamada \emph{mbstowcs}. Esta función se encarga de detectar los caracteres anchos que están 'rotos' en trozos de 1 byte, los une, y devuelve toda la cadena recibida como una cadena de caracteres anchos. Por último, se almacena lo devuelto en una variable \emph{wstring}, y se introduce en el idioma correspondiente.\\

Se proporciona un método observador para conocer el nombre del idioma activo, otro que comprueba si un nombre de idioma introducido existe como clave en el diccionario que almacena los idiomas, y dos métodos para cambiar el idioma: uno que recibe directamente como parámetro el nombre del idioma a activar, y otro que alterna entre todos los idiomas registrados a través de una función que va activando cada uno según se lo encuentre al recorrer el diccionario. Este último método obtiene un iterador al idioma activo actual dentro del diccionario, y lo avanza una posición. Si no se ha llegado al final del diccionario, se activa el idioma encontrado; en caso contrario, se activa el primero del diccionario.\\

Por último, para obtener el texto asociado a una etiqueta concreta en el idioma activo actual, existe una función que, dado el nombre de la etiqueta mediante una cadena de caracteres \emph{string}, devuelve el texto asociado a ésta mediante una cadena de caracteres anchos \emph{wstring}, el cual se puede dibujar en pantalla (si la fuente seleccionada para ello tiene soporte para el juego de caracteres necesario), guardar en un fichero, o utilizar en cualquier operación deseada.\\

En la figura \ref{lang} se muestra la interfaz pública de la clase Lang.\\

\figura{lang.png}{scale=0.6}{Interfaz pública de la clase Lang}{lang}{h}

\subsection{Pruebas}

Respecto a las pruebas realizadas para validar este módulo de soporte de internacionalización, se han realizado comprobaciones que consistían en imprimir en la pantalla las etiquetas de texto en diversos idiomas, como son español, inglés y ruso (para los caracteres no latinos). También se probaron los métodos de cambio de idioma activo, y la reacción del módulo al pedirle encontrar un nombre de idioma o etiqueta que no existen.

