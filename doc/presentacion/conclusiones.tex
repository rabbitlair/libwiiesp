% Este archivo es parte de la presentación de libWiiEsp, protegida bajo la 
% licencia GFDL. Copyright (C) 2011 Ezequiel Vázquez de la Calle

% -*-conclusiones.tex-*-

\section{Conclusiones}

\begin{frame}	
\frametitle{Conclusiones}
	\begin{block}{Cumplimiento de objetivos}
		\begin{itemize}
			\item Se da acceso a los subsistemas de Nintendo Wii.
			\item Se cubren los puntos básicos de videojuegos 2D:
			\begin{itemize}
				\item Gestión de recursos multimedia.
				\item Animaciones.
				\item Detección de colisiones.
				\item Soporte de internacionalización.
				\item Diseño de escenarios.
				\item Registro de mensajes del sistema.
			\end{itemize}
			\item Documentación completa, útil y en español.
			\item Tres juegos de ejemplo.
			\begin{itemize}
				\item Mecánicas diferentes.
				\item Código fuente totalmente comentado.
			\end{itemize}
		\end{itemize}
	\end{block}
\end{frame}

\begin{frame}	
\frametitle{Conclusiones}
	\begin{block}{Objetivos personales}
		\begin{itemize}
			\item Desarrollo de biblioteca partiendo de una base pequeña.
			\item Aprendizaje de varias herramientas libres:
			\begin{itemize}
				\item Profundización en GNU Make.
				\item Bibliotecas: Libfat, FreeType2 y TinyXML.
				\item Doxygen.
				\item \LaTeX.
				\item Subversion.
			\end{itemize}
			\item Adquisición de conocimientos sobre Nintendo Wii:
			\begin{itemize}
				\item Trabajo con formatos multimedia.
				\item Control de los mandos de la videoconsola.
				\item Comprender cómo funciona Wii internamente.
			\end{itemize}
			\item Puesta en práctica de los conocimientos adquiridos.
			\item Contribución al mundo del Software Libre y el \textit{Homebrew}.
		\end{itemize}
	\end{block}
\end{frame}

\begin{frame}	
\frametitle{Conclusiones}
	\begin{block}{Posibles mejoras}
		\begin{itemize}
			\item Sistema de sonido 3D.
			\item Soporte para otros periféricos de Nintendo Wii.
			\item Puertos USB traseros.
		\end{itemize}
	\end{block}
	\begin{block}{Futuro del proyecto}
		\begin{itemize}
			\item Desarrollo de juegos más complejos.
			\item Creación de comunidad de desarrolladores.
		\end{itemize}
	\end{block}
\end{frame}

