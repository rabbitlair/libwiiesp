% Este archivo es parte de la presentación de libWiiEsp, protegida bajo la 
% licencia GFDL. Copyright (C) 2011 Ezequiel Vázquez de la Calle

% -*-introduccion.tex-*-

\section{Introducción}
\begin{frame}{Introducción}
	\begin{block}{Videoconsolas}
		\noindent Una videoconsola es un sistema electrónico de entretenimiento que ejecuta videojuegos. Pueden tener diversas arquitecturas.
		\begin{itemize}
			\item Nintendo Wii tiene arquitectura \textit{Power PC}.
		\end{itemize}
	\end{block}
	\pause
	\begin{block}{Sistemas cerrados}
	\noindent Los fabricantes de videoconsolas controlan desarrollo y ejecución.
	\begin{itemize}
		\item Kits de desarrollo sólo accesibles mediante contratos.
		\item Soportes con sistemas de ficheros privativos.
		\item No se permite correr ejecutables sin firmar digitalmente.
	\end{itemize}
	\end{block}
\end{frame}

\begin{frame}{Introducción}
	\begin{block}{El \textit{Homebrew} o software casero}
		\begin{itemize}
			\item \textit{Scene}: sacar máximo partido a aparatos electrónicos.
			\item Herramientas libres para crear y ejecutar código sin firmar.
			\item Lanzadores de ejecutables.
			\item \textit{Custom Firmwares}.
			\item Ampliación de la funcionalidad de una videoconsola.
		\end{itemize}
	\end{block}
	\pause
	\begin{block}{Reacciones de los fabricantes}
		\begin{itemize}
			\item Actualizaciones bloquean software casero.
			\item Nintendo 64: cartuchos.
			\item Playstation 3: demandas judiciales.
			\item Xbox 360: XNA.
		\end{itemize}
	\end{block}
\end{frame}

\begin{frame}{Introducción}
	\begin{block}{¿Por qué este PFC?}
	\begin{itemize}
		\item \textit{Libogc}: muy bajo nivel, difícil de usar y centrada en el hardware de Wii.
		\item Todo escrito en C, existiendo soporte para C++.
		\item Escasa documentación, difícil de encontrar y en inglés.
	\end{itemize}
	\end{block}
	\pause
	\begin{block}{Objetivos}
	\begin{itemize}
		\item Crear una herramienta útil y libre para desarrollar para Wii.
		\item Generar documentación amplia, práctica y en español.
		\item Ofrecer una visión general del funcionamiento de Wii.
		\item Proporcionar juegos de ejemplo.
	\end{itemize}
	\end{block}
\end{frame}

