% Este archivo es parte de libWiiEsp. Copyright (C) 2011 Ezequiel Vázquez de la Calle
% Licencia GFDL. Encontrará una copia de la licencia en el archivo fdl-1.3.tex

Con todo lo visto hasta el momento en este manual es posible comenzar el desarrollo de nuestra aplicación, ya que tenemos las herramientas preparadas, sabemos qué hay que tener en cuenta a la hora de utilizarlas, y además contamos con las clases de \emph{LibWiiEsp}, que nos facilitan (mucho) la vida a la hora de cargar recursos multimedia, establecer el soporte de idiomas, acceder a la tarjeta SD de la consola, dibujar texturas y animaciones en pantalla, etc.\\

Pero el punto más interesante de \emph{LibWiiEsp}, además de la abstracción que nos proporciona a la hora de trabajar con estos subsistemas de la videoconsola, está formado por las tres plantillas (clases abstractas) que ofrece para facilitar el desarrollo de un videojuego. Estas plantillas, que son \emph{Actor}, \emph{}Nivel y \emph{Juego}, permiten crear los distintos tipos de actores y niveles, además de la clase principal de nuestro programa, de una manera sencilla y efectiva.\\

Cabe destacar que, en el caso de que las plantillas ofrecidas no se adaptaran a las necesidades del videojuego que tenemos en mente, siempre podemos personalizar la clase que corresponda al derivarla, o bien modificando la clase original desde el código fuente de la biblioteca.\\

En esta sección del manual vamos a desgranar los detalles necesarios para poder sacar el máximo jugo a estas tres plantillas que nos facilitarán el desarrollo de nuestro videojuego:

\subsection{Sistemas de coordenadas}

En primer lugar, hay que comprender cómo funcionan los distintos sistemas de coordenadas que nos encontraremos a la hora de crear el universo de nuestro videojuego. La siguiente imagen ilustra los tres sistemas que pueden darse a la vez en el juego:

\figura{coordenadas.png}{scale=0.6}{Distintos sistemas de coordenadas en el universo del juego}{coordenadas}{H}

La imagen al completo representa un escenario completo de un juego. Este escenario es el universo del juego en sí, tiene forma rectangular, y su esquina superior izquierda es el origen de coordenadas, siendo la X positiva hacia la derecha, y la Y hacia abajo. Todos las parejas de coordenadas, en cualquiera de los tres sistemas que se emplean, se refieren al punto superior izquierdo del objeto al que pertenecen las coordenadas. En la imagen, se distinguen las coordenadas relativas al escenario con el color amarillo.\\

El rectángulo verde que se aprecia en el centro de la imagen representa la parte del escenario que se muestra en la pantalla (actua como si fuera una ventana desplazable). La pareja de coordenadas (x,y) en amarillo de este rectángulo indica la posición de desplazamiento de la pantalla, o \emph{scroll}, respecto al punto (0,0) del escenario. Jugando con el \emph{scroll} conseguimos que la pantalla \emph{se mueva} sobre el escenario, y muestre la parte que queramos de éste. Por otro lado, el vértice superior izquierdo de la pantalla es el origen de otro sistema de coordenadas, que indica la posición de un objeto (un actor, por ejemplo) dentro de la visualización en pantalla.\\

Teniendo en cuenta estos dos sistemas de coordenadas, se entiende fácilmente que el actor (el rectangulo rojo) tenga una pareja de coordenadas que indiquen su posición en el escenario (coordenadas en amarillo), y otra pareja de coordenadas (en verde) que denotan su posición en la pantalla. Sin embargo, las coordenadas de un actor respecto a la pantalla no se almacenan, si no que se calculan restando su posición en el escenario (amarillo) menos la posición del desplazamiento de la pantalla (coordenadas amarillas de la pantalla).\\

Además, el vértice superior izquierdo del rectángulo que representa al actor es el origen de un tercer sistema de coordenadas, que se utiliza como referencia para las cajas de colisión asociadas al actor.

\subsection{Actores}
% Este archivo es parte de libWiiEsp. Copyright (C) 2011 Ezequiel Vázquez de la Calle
% Licencia GFDL. Encontrará una copia de la licencia en el archivo fdl-1.3.tex

Un actor es un objeto que tiene entidad propia dentro del universo del videojuego. En este sentido, son actores tanto los protagonistas manejados por los jugadores, como los enemigos controlados por la máquina, los items que recogemos, cada una de las balas (en el caso de un juego de disparos) también es un actor\ldots \\

Entrando en el apartado técnico, un actor se representa como un objeto que tiene una pareja de coordenadas (x,y) respecto al origen del escenario, una velocidad en píxeles por fotograma (tanto vertical como horizontal), un conjunto de estados, cada uno de los cuales tiene asociada una animación y varias figuras de colisión, y un indicador sobre qué estado de los posibles es el actual.\\

A continuación se explican los diversos aspectos a tener en cuenta a la hora de crear un actor utilizando la clase abstracta que proporciona \emph{LibWiiEsp}. En primer lugar, tenemos que crear una clase derivada de \emph{Actor}. En el constructor de nuestra clase derivada, no debemos olvidar pasarle al constructor de \emph{Actor} la cadena de caracteres con la ruta absoluta hasta el archivo de datos desde el que se cargan los datos del actor, y una referencia al nivel en el que participará. Además, tenemos que definir el método \emph{actualizar()}, que es un método virtual puro, y en el cual tenemos que definir el comportamiento del actor dependiendo de su estado actual.\\

Ambos aspectos se explican con detalle en los siguientes apartados:

\subsubsection{Cargando los datos de un actor}

Cada actor que se cree derivando la clase \emph{Actor} cargará toda la información relativa a él a través del método \emph{cargarDatosIniciales()}, definido en la propia clase base \emph{Actor} (al cual se llama desde el constructor de ésta clase), y que recibe la ruta absoluta, en la tarjeta SD, de un archivo XML con un formato como el siguiente:\\

\begin{lstlisting}[style=XML]
<?xml version="1.0" encoding="UTF-8"?>
<actor vx="3" vy="3" tipo="jugador">
  <animaciones>
    <animacion estado="normal" img="chief" sec="0" filas="1" columnas="6" retardo="3" />
    <animacion estado="mover" img="chief" sec="0,1,2,3,4" filas="1" columnas="6" retardo="3" />
    <animacion estado="muerte" img="chief" sec="5" filas="1" columnas="6" retardo="0" />
  </animaciones>
  <colisiones>
    <rectangulo estado="normal" x1="27" y1="21" x2="55" y2="21" x3="55" y3="96" x4="27" y4="96" />
    <circulo estado="normal" cx="41" cy="13" radio="8" />
    <rectangulo estado="mover" x1="27" y1="21" x2="55" y2="21" x3="55" y3="96" x4="27" y4="96" />
    <circulo estado="mover" cx="41" cy="13" radio="8" />
    <sinfigura estado="muerte" />
  </colisiones>
</actor>
\end{lstlisting}

En el archivo XML anterior pueden observarse dos grandes bloques, uno para las animaciones y otro para las figuras de colisión. A cada estado (en el ejemplo hay tres, \emph{normal}, \emph{mover} y \emph{muerte}) le corresponde una única animación, pero puede tener ninguna, una o varias figuras de colisión. En el caso de que un estado concreto no tenga ninguna figura de colisión asociada, basta con introducir un nodo con el nombre \emph{sinfigura} y su correspondiente atributo \emph{estado}. Para más información sobre las animaciones o las figuras de colisión, consultar las secciones correspondientes en la referencia de la biblioteca.\\

\textbf{Muy importante}: Los estados en los cuales puede encontrarse un actor vienen definidos por los que aparezcan en este archivo de datos, y es \textbf{imprescindible} que definamos el estado \emph{normal} (al menos, en las animaciones), ya que es el estado que un actor toma por defecto, y si no se encontrara entre los datos del actor, se produciría un error en el sistema.\\

Para más información sobre la carga de datos iniciales, consultar la documentación de la clase Actor.

\subsubsection{Definir el comportamiento de un actor}

El comportamiento de un actor, como ya se ha comentado, depende del estado en el que se encuentre. A la hora de crear una clase derivada de \emph{Actor} se deberían implementar tantos métodos como estados se hayan definido en el archivo de datos del actor, de tal manera que cada uno de estos métodos corresponda con el comportamiento esperado en cada uno de los estados.\\

Por ejemplo, si tenemos un actor con tres estados (normal, caer y mover), tendríamos tres nuevos métodos llamados \emph{estado\_normal()}, \emph{estado\_caer()} y \emph{estado\_mover()}. En cada una de estas funciones habría que implementar el comportamiento deseado de nuestro actor para ese estado concreto. En el siguiente código se muestra un ejemplo del método \emph{estado\_mover()}, que se encargaría de desplazar horizontalmente al actor:

\begin{lstlisting}[style=C++]
void estado_mover(void) {
    mover(_x + _vx, _y);
}
\end{lstlisting}

Implementando de esta manera un método por cada estado, únicamente habría que definir el método virtual puro \emph{actualizar()} para que, según el estado actual del actor, se ejecute la función correspondiente.\\

Lo ideal es organizar el comportamiento del actor en un autómata finito determinado, donde se especifiquen los estados posibles, y las transiciones que pueden darse entre los distintos estados. Hay que mencionar que los cambios de estado se realizarán desde una clase derivada de \emph{Nivel}, que será el escenario donde los actores se encontrarán. El motivo de esta decisión no es otro que la falta de conocimiento que tiene un actor sobre lo que ocurre a su alrededor en el escenario del juego, información que sí está disponible en todo momento en la clase que se encarga de gestionar éste.\\

\figura{automata.png}{scale=0.7}{Sencillo autómata de ejemplo para el comportamiento de un actor}{automata}{H}

En la figura 3 puede apreciarse un sencillo ejemplo de autómata finito determinado, formado por cuatro estados, y que define el comportamiento de un actor controlado por un jugador a través de un mando. El actor comienza en el estado denominado \emph{NORMAL}, en el que no sufre ninguna modificación de sus variables internas de posición. Desde este estado, y dependiendo de las condiciones que se cumplan, se puede pasar a los estados \emph{MOVER} (si el jugador pulsa el botón de movimiento) o \emph{CAER} (si no existe colisión entre el actor y el escenario). En el primero, nuestro actor modifica su posición horizontal en base a su velocidad en este eje, y en el segundo, se cambia la posición vertical hacia abajo, en base también a la velocidad del actor respecto al eje vertical. Desde estos dos estados puede llegarse a \emph{MOVER EN CAIDA}, que es una combinación de ambos (movimiento en ambos ejes).\\

Puede comprobarse que, en cada estado, se proporciona un comportamiento para el actor y una serie de condiciones para que sucedan cambios de estado del actor. La manera de comportarse del actor para cada estado debe programarse en el método \emph{actualizar()} de la clase derivada de \emph{Actor} que controla a éste; sin embargo, las comprobaciones sobre el cumplimiento de condiciones que deben satisfacerse para ejecutar un cambio desde un estado a otro puede realizarse en el método correspondiente de la clase que controle el escenario (\emph{actualizarPj()} para los actores jugadores, o \emph{actualizarNpj()} en el caso de los actores controlados por la máquina), o bien en el propio método de actualización del actor, ya que éste tiene un atributo que es una referencia al nivel en el que participa. Se recomienda optar por la primera opción, para así separar el comportamiento del actor según el estado y las transiciones posibles entre éstos.



\subsection{Niveles}
% Este archivo es parte de libWiiEsp. Copyright (C) 2011 Ezequiel Vázquez de la Calle
% Licencia GFDL. Encontrará una copia de la licencia en el archivo fdl-1.3.tex

Un nivel representa el escenario donde los actores, ya estén controlados por un jugador o por la máquina, interactúan entre sí y con los componentes de dicho nivel. Al igual que ocurre con los actores, \emph{LibWiiEsp} proporciona una clase base para crear niveles de una manera sencilla y rápida, que permite diseñar cada nuevo escenario utilizando el software \emph{Tiled}, editor de mapas de tiles.\\

Hay que distinguir entre los conceptos de escenario y nivel. Para \emph{LibWiiEsp}, un nivel es una clase derivada de la clase abstracta \emph{Nivel}, y que define varios escenarios que se comportan de la misma forma, siendo cada escenario un mapa de tiles generado con \emph{Tiled} en el que se especifica la disposición de los tiles y los actores que participan.\\

Con esto se consiguen varias ventajas. En primer lugar, definiendo una sola vez el comportamiento de un tipo de escenario en una clase derivada de \emph{Nivel}, se pueden generar múltiples escenarios cuya lógica sea la implementada en esta clase. Por otro lado, este sistema permite que, en un mismo videojuego, se puedan intercalar fácilmente escenarios con comportamientos distintos (como ejemplo, se puede pensar en las típicas fases de \emph{bonus} de clásicos como \emph{Street Fighter}, en las que se debe destruir un coche o varios barriles en lugar de luchar contra otro oponente controlado por la máquina).

\subsubsection{Partes de un nivel}

Un nivel se compone de tres partes, que son los distintos tipos de objetos que se cargan en un escenario. En primer lugar, se encuentran las \emph{propiedades} del nivel, que son una serie de cadenas de caracteres que indican códigos de recursos en la galeria de medias del sistema; las propiedades del nivel son la imagen de fondo (que es fija, y permite dibujar un paisaje estático en la última capa de dibujo), la pista de música asociada a un nivel y la imagen del \emph{tileset}. Pueden añadirse más propiedades según el videojuego que estemos desarrollando, pero la lectura y carga de esta información deberá ser programada en el constructor de la clase derivada.

\figura{partesescenario.png}{scale=0.5}{Distintas partes de un escenario}{partesescenario}{H}

Llegados a este punto, es necesario explicar brevemente los conceptos de \emph{tile} y \emph{tileset}. Un tile es una imagen pequeña, generalmente cuadrada o rectangular, que se utiliza para componer un escenario en un videojuego en dos dimensiones. En un mismo escenario, se emplean numerosos tiles que se recogen en una única imagen organizados como una tabla (es decir, en filas y columnas). Esta imagen que almacena todos los tiles utilizados en un mismo escenario es lo que conocemos por tileset.

\figura{tileset.png}{scale=1}{Ejemplo de tileset, formado por 6 tiles de 32x32 pixeles}{tileset}{H}

Continuando con los tipos de objeto que componen un escenario, en segundo lugar tenemos la propia composición de éste, que está formada por dos capas de dibujo en las que se define la composición del escenario en sí a partir de los tiles del tileset asociado al nivel. Internamente, cada capa de dibujo se divide en una rejilla de cuadros (filas y columnas), donde cada celda tiene las mismas dimensiones que un tile, y en ella se coloca un tile concreto. Una de las dos capas del escenario, la llamada \emph{PLATAFORMAS}, se caracteriza en que, cuando \emph{LibWiiEsp} la carga en memoria, asigna a cada tile una figura de colisión de su mismo tamaño, de tal manera que los tiles que componen esta capa de dibujo pueden interactuar con los actores del juego. Por otro lado, la capa denominada \emph{ESCENARIO} está compuesta por tiles que no tienen asociada ninguna figura de colisión, y únicamente se añaden al nivel con el objetivo de mejorar visualmente el escenario.\\

Generalmente, esta distinción entre tiles con figura de colisión asociada y sin ella se utiliza para definir los tiles que pueden ser atravesados por los actores del juego (aquellos que no tienen asociada una figura de colisión), y los que no permiten que los actores los atraviesen, que son los que sí tienen figura de colisión asociada.\\

\figura{ejemplotiles.png}{scale=0.8}{Ejemplo de escenario con tiles atravesables y no atravesables}{ejemplotiles}{H}

Por último, nos encontramos con el último tipo de objetos que necesitamos para definir un escenario, que son los actores que participan en el nivel. Se distinguen los actores controlados por los jugadores, y los que son dirigidos por la máquina.\\

Como ya se ha comentado en la sección de los actores, éstos tendrán definido su comportamiento según el estado en el que se encuentren en un momento determinado. Sin embargo, las transiciones entre distintos estados es conveniente realizarlas en el correspondiente método de actualización de la clase que gestiona el escenario, a pesar de que se pueden hacer desde la clase del actor.

\subsubsection{Creación de un escenario con \emph{Tiled}}

El proceso de creación de un escenario a partir de la herramienta \emph{Tiled} es muy sencillo. En su página oficial, \emph{http://www.mapeditor.org/}, podemos descargar la última versión o, dependiendo de si el sistema en el que nos encontremos (en mi caso al redactar este manual, una Ubuntu 10.10) dispone de esta utilidad en los repositorios, se puede instalar con la siguiente orden:

\begin{lstlisting}[style=consola]
sudo apt-get install tiled
\end{lstlisting}

Una vez instalado, ejecutamos \emph{Tiled} y pulsamos en el botón \emph{Nuevo}. Se abrirá una ventana en la que se nos preguntan algunos parámetros del nuevo mapa de tiles. Seleccionamos proyección ortogonal, el ancho y alto en píxeles de cada tile, y el ancho y alto medido en número de tiles que tendrá el escenario. Las medidas de un tile deben ser múltiplos de 8 píxeles, debido a los requisitos para la carga de texturas con \emph{LibWiiEsp}, siendo un tamaño recomendable es 32x32 píxeles por tile. La resolución que utiliza la Nintendo Wii en un sistema PAL y proporción 4:3 es de 640 píxeles de ancho por 528 de alto, por lo que el tamaño mínimo del escenario debería ser (si utilizamos la medida 32x32 píxeles por tile) de 20 tiles de ancho, por 16 tiles de alto. Una vez establecidos estos parámetros, pulsamos en aceptar.\\

A continuación, hay que cargar en la herramienta la imagen que contiene el tileset. Para ello, pulsamos en \emph{Mapa}, y después en \emph{Nuevo conjunto de patrones}. Se abrirá un diálogo en el que debemos introducir el nombre que le daremos a la imagen (esto es un dato interno de \emph{Tiled}, y el valor introducido aquí no es relevante), la propia imagen, y las medidas que tendrá un tile (ancho y alto en píxeles). Un detalle a tener en cuenta, en nuestra imagen no debe haber separación entre los tiles, ni tampoco margen. Al introducir estos datos, hacemos clic en \emph{Aceptar} y ya tendremos el tileset listo para dibujar el mapa.\\

Como siguiente punto, hay que introducir las tres propiedades del escenario. Nos vamos a \emph{Mapa} y entonces a \emph{Propiedades del Mapa}, e introducimos en la lista estas propiedades:

\begin{itemize}
\item imagen\_fondo: Su valor es el código que tendrá la imagen de fondo del escenario en la galería de recursos del sistema.
\item imagen\_tileset: Su valor es el código que tendrá la imagen del tileset del escenario en la galería de recursos del sistema.
\item musica: Su valor es el código que tendrá la pista de música del escenario en la galería de recursos del sistema.
\end{itemize}

Tras introducir las propiedades, pulsamos en \emph{Aceptar}, y procedemos a preparar las tres capas de dibujo de nuestro escenario, que deben llamarse obligatoriamente como se describe en la lista:

\begin{itemize}
\item escenario: Capa de patrones donde dibujaremos los tiles atrabvesables.
\item plataformas: Capa de patrones donde dibujaremos los tiles no atrabvesables, es decir, los que tendrán asociada una figura de colisión.
\item actores: Capa de objetos donde situaremos todos los actores que participen en el nivel.
\end{itemize}

En este punto, ya estamos en disposición de comenzar a dibujar nuestro escenario. Es recomendable comenzar por la capa de tiles no atravesables, continuar decorando con los tiles atravesables, y por último, establecer la posición de los actores. Para situar un actor en el escenario, insertamos un objeto en el lugar que deseemos, y lo redimensionamos adecuadamente para que se vea correctamente cómo quedaría su posición inicial. Seguidamente, hacemos clic con el botón derecho en el actor, y pulsamos en \emph{Propiedades del objeto}. Se mostrará un diálogo en el que debemos indicar, en el campo \emph{Tipo}, el tipo del actor que comenzará aquí (dato necesario a la hora de distinguir qué clase derivada de \emph{Actor} hay que instanciar), y además la propiedad \emph{xml}, cuyo valor debe ser la ruta absoluta hasta el archivo XML que contiene la información de este tipo de actor en la tarjeta SD de la Nintendo Wii. Además, si el actor es controlado por un jugador, debemos añadir también la propiedad \emph{jugador} con el código identificativo del jugador que jugará con este actor como valor.\\

A la hora de definir los distintos actores, se puede utilizar el campo \emph{Nombre} del diálogo de propiedades de un actor para identificarlo en tiempo de diseño del escenario.\\

\subsubsection{Implementando una clase que controle escenarios}

Una vez tenemos creados uno o varios escenarios cuyo comportamiento (la gestión de transiciones entre los estados de los distintos actores que participan en él, el movimiento o no del \emph{scroll} de la pantalla sobre el escenario, etc.) es común, el último paso para poder disfrutar de ellos es definir una clase derivada de la abstracta \emph{Nivel} que controle todos los detalles del nivel. Cada clase derivada definirá una gestión distinta de un grupo de varios escenarios.\\

El primer paso para crear una clase derivada de la abstracta \emph{Nivel} es implementar el método virtual \emph{cargarActores()}. Como se detalla en la documentación de la clase, en el constructor se cargan todos los tiles del escenario, se toma la imagen de fondo y del tileset desde la galería, y se leen los datos de inicialización de cada actor participante, almacenándose en una estructura temporal. La definición de este método debe recorrer esta estructura temporal de datos de actores, creando, para cada ocurrencia, un actor de la clase derivada de \emph{Actor} correspondiente, tal y como se aprecia en el siguiente ejemplo:

\begin{lstlisting}[style=C++]
void cargarActores(void) {
  for(Temporal::iterator i = _temporal.begin(); i != _temporal.end(); ++i)
  {
    if(i->tipo_actor == "jugador") {
      Personaje* p = new Personaje(i->xml, this);
      p->mover(i->x, i->y);
      _jugadores.insert(std::make_pair(i->jugador, p));
    } else if(i->tipo_actor == "bicho") {
      Bicho* p = new Bicho(i->xml, this);
      p->mover(i->x, i->y);
      _actores.push_back(p);
    }
  }
  _temporal.clear();
};
\end{lstlisting}

En el ejemplo, el programador ha definido dos clases derivadas de \emph{Actor}, denominadas \emph{Personaje} y \emph{Bicho}. En el método implementado, se recorre la estructura temporal de datos de actores del escenario, y se crea un actor a partir de la clase correspondiente (según el tipo de actor que se haya indicado desde \emph{Tiled}), se mueve el actor hasta su posición en el escenario, y por último se inserta en la estructura adecuada (\emph{\_jugadores} en el caso de los actores controlados por un jugador, en la que hay que indicar también el código identificador del jugador concreto; o \emph{\_actores} para los actores controlados por la máquina).\\

Es importante recordar que este método \emph{cargarActores()} se debe llamar desde el constructor de la clase derivada de \emph{Nivel} con la intención de que la creación de los actores se realice en el momento de cargar el escenario. Además, es recomendable vaciar la estructura temporal cuando se finalice el proceso.\\

El siguiente paso en la generación de esta clase derivada es implementar los métodos de actualización del nivel, que se deberían llamar en cada fotograma del programa. A continuación se indican cuáles son, y qué funcionalidad se espera que tengan.\\

En el método \emph{actualizarPj()} se debe actualizar el estado de un único actor jugador, atendiendo tanto al mando concreto que tenga asociado en la estructura \emph{\_jugadores}, como a la situación del escenario. Un ejemplo sencillo podría ser:

\begin{lstlisting}[style=C++]
void actualizarPj(const std::string& jugador, const Mando& m) {

  // Estado NORMAL: puede pasar a MOVER
  if(_jugadores[jugador]->estado() == "normal") {
    if(m.pressed(Mando::BOTON_ARRIBA) or m.pressed(Mando::BOTON_ABAJO))
      _jugadores[jugador]->setEstado("mover");
  }

  // Estado MOVER: puede pasar a NORMAL
  if(_jugadores[jugador]->estado() == "mover") {
    if(m.pressed(Mando::BOTON_ARRIBA)) {
      _jugadores[jugador]->invertirDibujo(true);
      s16 vel_x = _jugadores[jugador]->velX();
      if(vel_x > 0)
        vel_x *= -1;
      _jugadores[jugador]->setVelX(vel_x);
    } else if(m.pressed(Mando::BOTON_ABAJO)) {
      _jugadores[jugador]->invertirDibujo(false);
      s16 vel_x = _jugadores[jugador]->velX();
      if(vel_x < 0)
        vel_x *= -1;
      _jugadores[jugador]->setVelX(vel_x);
    } else
      _jugadores[jugador]->setEstado("normal");
  }

  // Actualizar el actor en base a su nuevo estado actual
  _jugadores[jugador]->actualizar();
}
\end{lstlisting}

Por otro lado, el método \emph{actualizarNpj()} debe recorrer la estructura en la que se almacenan los actores controlados por la máquina y actualizar los que se consideren oportunos (aquí se deja en manos del programador el actualizar todos los actores, sólo los que están en pantalla, o los que cumplan un determinado criterio). Como ejemplo, se muestra la siguiente función que actualizaría el estado de todos los actores no jugadores:

\begin{lstlisting}[style=C++]
void actualizarNpj(void) {
  for(Actores::iterator i = _actores.begin() ; i != _actores.end() ; ++i) {
    // Estado NORMAL: puede pasar a CAER
    if((*i)->estado() == "normal")
      if(not colision((*i)))
        (*i)->setEstado("caer");

    // Estado CAER: puede pasar a NORMAL
    if((*i)->estado() == "caer")
      if(colisionVertical((*i)))
        (*i)->setEstado("normal");

    // Actualizacion del actor
    (*i)->actualizar();
  }
};
\end{lstlisting}

El último método a implementar es \emph{actualizarEscenario()}, en el que se espera que se implementen todos los demás detalles relativos al escenario que necesiten ser actualizados a cada fotograma del juego. El siguiente ejemplo muestra una implementación que únicamente actualiza el \emph{scroll} de la pantalla sobre el escenario, según la posición horizontal del jugador cuyo código identificador es \emph{pj1}:

\begin{lstlisting}[style=C++]
void actualizarEscenario(void) {

  if(_jugadores["pj1"]->x() - _scroll_x >= screen->ancho() / 2)
    moverScroll(_scroll_x + abs(_jugadores["pj1"]->velX()), _scroll_y);
  else if(_jugadores["pj1"]->x() - _scroll_x <= screen->ancho() / 4)
    moverScroll(_scroll_x - abs(_jugadores["pj1"]->velX()), _scroll_y);

  if(_jugadores["pj1"]->y() - _scroll_y >= screen->alto() / 2)
    moverScroll(_scroll_x, _scroll_y + abs(_jugadores["pj1"]->velY()));
  else if(_jugadores["pj1"]->y() - _scroll_y <= screen->alto() / 4)
    moverScroll(_scroll_x, _scroll_y - abs(_jugadores["pj1"]->velY()));
};
\end{lstlisting}

Por supuesto, quiero remarcar que los métodos de ejemplo son precisamente eso, ejemplos muy sencillos cuya finalidad es que sirvan de guía para comprender cómo se trabaja con la plantilla de niveles de \emph{LibWiiEsp}, y a partir de los cuales poder desarrollar los métodos de actualización necesarios (al derivar la clase abstracta \emph{Nivel} se pueden añadir los métodos que se consideren necesarios).\\



\subsection{Juego}
% Este archivo es parte de libWiiEsp. Copyright (C) 2011 Ezequiel Vázquez de la Calle
% Licencia GFDL. Encontrará una copia de la licencia en el archivo fdl-1.3.tex

La clase abstracta \emph{Juego} es la tercera y última plantilla que \emph{LibWiiEsp} ofrece para facilitar el desarrollo de videojuegos para Nintendo Wii. Es muy sencilla, y consiste en dos partes principales. El constructor se encarga de inicializar la consola a partir de la información que se introduzca en el archivo de configuración de la aplicación, y el método \emph{run()} ejecuta el bucle principal del programa. A continuación se aportan todos los detalles relativos a esta plantilla para construir la clase principal de nuestro videojuego.

\subsubsection{Inicialización de la consola}

Como ya se ha comentado, la inicialización de todos los sistemas de la consola Nintendo Wii se realiza en el constructor de la clase \emph{Juego}, que recibe como parámetro la ruta absoluta en la tarjeta SD de un archivo XML de configuración. Esta inicialización consiste en montar la primera partición de la tarjeta SD de la consola (debe tener un sistema de ficheros FAT), establecer el sistema de \emph{logging}, leer el archivo de configuración y, a partir de éste, iniciar todos los aspectos de la consola que vamos a utilizar.\\

El proceso de inicialización, llegados a este punto, es el siguiente:

\begin{enumerate}
\item Inicializar la pantalla, el sistema de mandos, el sonido y las fuentes de textos (en este orden).
\item Establecer el color transparente, y los fotogramas por segundo que tendrá la aplicación.
\item Cargar los identificadores de los jugadores y asociar un mando con cada uno de ellos.
\item Cargar en memoria todos los recursos multimedia que se indiquen en el archivo XML de la galería.
\item Cargar en memoria las etiquetas de texto del soporte de idiomas.
\end{enumerate}

Cuando creamos una clase derivada de \emph{Juego} hay que llamar al constructor de la clase base, pasándole como parámetro la ruta absoluta en la tarjeta SD del archivo de configuración. Por otro lado, el destructor de la clase base se encarga de liberar la memoria ocupada por la estructura que almacena los objetos de la clase \emph{Mando}, y de llamar a la función \emph{exit()}, hecho obligatorio para que la pila de la función \emph{atexit()} se ejecute al salir del programa (esto es muy importante, ya que en caso contrario nos encontraremos con una pantalla de error por no haber apagado los sistemas de la consola).\\

Un ejemplo del archivo XML de configuración es el siguiente:

\begin{lstlisting}[style=XML]
<?xml version="1.0" encoding="UTF-8"?>
<conf>
  <log valor="/apps/wiipang/info.log" nivel="3" />
  <alpha valor="0xFF00FFFF" />
  <fps valor="25" />
  <galeria valor="/apps/wiipang/xml/galeria.xml" />
  <lang valor="/apps/wiipang/xml/lang.xml" defecto="english" />
  <jugadores pj1="pj1" pj2="pj2" pj3="" pj4="" />
</conf>
\end{lstlisting}

En este archivo de configuración se establece que el sistema de \emph{logging} registrará todos los eventos que sucedan en el sistema, el color transparente será el magenta, se correrá la aplicación a 25 fotogramas por segundo, se indican los archivos XML de la galería y el sistema de idiomas, se establece el inglés como idioma por defecto, y se prepara la consola para trabajar con dos mandos, asociados respectivamente a un jugador identificado por el código \emph{pj1} y otro identificado por \emph{pj2}.\\

\subsubsection{El bucle principal}

La clase base \emph{Juego} proporciona, además, un método que ejecuta un bucle principal sencillo. Este método es virtual, de tal manera que si el programador necesita otra forma de gestionar su aplicación, se le permite redefinirlo en su clase derivada.\\

El método \emph{run()} es el que se encarga de controlar este bucle principal. En primer lugar, llama al método \emph{cargar()}, que es virtual puro y debe ser definido en la clase derivada de \emph{Juego}. En esta función debe ejecutarse todo lo que se necesite \textbf{antes} de que se entre en el bucle. Después, se inicializa la bandera de salida con un valor falso y se establece el contador de ciclos del procesador a cero (este contador se utiliza para mantener constante el valor de los fotogramas por segundo), tras lo cual se entra en el bucle principal.\\

El bucle principal actualiza, al principio de cada fotograma, el estado de todos los mandos conectados a la consola, ejecuta el método virtual puro \emph{frame()}, y después finaliza el fotograma y controla la tasa de FPS. En este método \emph{frame()} se incluirán todos los detalles de la ejecución de cada fotograma, y devolverá un valor booleano falso si la ejecución debe continuar, siendo el valor de retorno verdadero en el caso de que el programa deba terminar.\\

Un detalle más, en el caso de que ocurriera una excepción en el transcurso de la ejecución del programa, ésta será registrada por el sistema de \emph{logging} (siempre que éste esté activado, al menos, en el nivel \emph{error}, identificador 1), y después se saldrá de la aplicación.\\

Por último, destacar el hecho de que tanto si se necesita una gestión del bucle principal diferente, o un mayor número de funciones, el hecho de tener que derivar de la clase \emph{Juego} implica la posibilidad de crear tantos métodos como sea necesario, y la redefinición opcional del método \emph{run()} nos permite ejecutar estos nuevos métodos de la manera que mejor se adecúe a nuestras necesidades.



\clearpage

