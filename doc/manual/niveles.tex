% Este archivo es parte de libWiiEsp. Copyright (C) 2011 Ezequiel Vázquez de la Calle
% Licencia GFDL. Encontrará una copia de la licencia en el archivo fdl-1.3.tex

Un nivel representa el escenario donde los actores, ya estén controlados por un jugador o por la máquina, interactúan entre sí y con los componentes de dicho nivel. Al igual que ocurre con los actores, \emph{LibWiiEsp} proporciona una clase base para crear niveles de una manera sencilla y rápida, que permite diseñar cada nuevo escenario utilizando el software \emph{Tiled}, editor de mapas de tiles.\\

Hay que distinguir entre los conceptos de escenario y nivel. Para \emph{LibWiiEsp}, un nivel es una clase derivada de la clase abstracta \emph{Nivel}, y que define varios escenarios que se comportan de la misma forma, siendo cada escenario un mapa de tiles generado con \emph{Tiled} en el que se especifica la disposición de los tiles y los actores que participan.\\

Con esto se consiguen varias ventajas. En primer lugar, definiendo una sola vez el comportamiento de un tipo de escenario en una clase derivada de \emph{Nivel}, se pueden generar múltiples escenarios cuya lógica sea la implementada en esta clase. Por otro lado, este sistema permite que, en un mismo videojuego, se puedan intercalar fácilmente escenarios con comportamientos distintos (como ejemplo, se puede pensar en las típicas fases de \emph{bonus} de clásicos como \emph{Street Fighter}, en las que se debe destruir un coche o varios barriles en lugar de luchar contra otro oponente controlado por la máquina).

\subsubsection{Partes de un nivel}

Un nivel se compone de tres partes, que son los distintos tipos de objetos que se cargan en un escenario. En primer lugar, se encuentran las \emph{propiedades} del nivel, que son una serie de cadenas de caracteres que indican códigos de recursos en la galeria de medias del sistema; las propiedades del nivel son la imagen de fondo (que es fija, y permite dibujar un paisaje estático en la última capa de dibujo), la pista de música asociada a un nivel y la imagen del \emph{tileset}. Pueden añadirse más propiedades según el videojuego que estemos desarrollando, pero la lectura y carga de esta información deberá ser programada en el constructor de la clase derivada.

\figura{partesescenario.png}{scale=0.5}{Distintas partes de un escenario}{partesescenario}{H}

Llegados a este punto, es necesario explicar brevemente los conceptos de \emph{tile} y \emph{tileset}. Un tile es una imagen pequeña, generalmente cuadrada o rectangular, que se utiliza para componer un escenario en un videojuego en dos dimensiones. En un mismo escenario, se emplean numerosos tiles que se recogen en una única imagen organizados como una tabla (es decir, en filas y columnas). Esta imagen que almacena todos los tiles utilizados en un mismo escenario es lo que conocemos por tileset.

\figura{tileset.png}{scale=1}{Ejemplo de tileset, formado por 6 tiles de 32x32 pixeles}{tileset}{H}

Continuando con los tipos de objeto que componen un escenario, en segundo lugar tenemos la propia composición de éste, que está formada por dos capas de dibujo en las que se define la composición del escenario en sí a partir de los tiles del tileset asociado al nivel. Internamente, cada capa de dibujo se divide en una rejilla de cuadros (filas y columnas), donde cada celda tiene las mismas dimensiones que un tile, y en ella se coloca un tile concreto. Una de las dos capas del escenario, la llamada \emph{PLATAFORMAS}, se caracteriza en que, cuando \emph{LibWiiEsp} la carga en memoria, asigna a cada tile una figura de colisión de su mismo tamaño, de tal manera que los tiles que componen esta capa de dibujo pueden interactuar con los actores del juego. Por otro lado, la capa denominada \emph{ESCENARIO} está compuesta por tiles que no tienen asociada ninguna figura de colisión, y únicamente se añaden al nivel con el objetivo de mejorar visualmente el escenario.\\

Generalmente, esta distinción entre tiles con figura de colisión asociada y sin ella se utiliza para definir los tiles que pueden ser atravesados por los actores del juego (aquellos que no tienen asociada una figura de colisión), y los que no permiten que los actores los atraviesen, que son los que sí tienen figura de colisión asociada.\\

\figura{ejemplotiles.png}{scale=0.8}{Ejemplo de escenario con tiles atravesables y no atravesables}{ejemplotiles}{H}

Por último, nos encontramos con el último tipo de objetos que necesitamos para definir un escenario, que son los actores que participan en el nivel. Se distinguen los actores controlados por los jugadores, y los que son dirigidos por la máquina.\\

Como ya se ha comentado en la sección de los actores, éstos tendrán definido su comportamiento según el estado en el que se encuentren en un momento determinado. Sin embargo, las transiciones entre distintos estados es conveniente realizarlas en el correspondiente método de actualización de la clase que gestiona el escenario, a pesar de que se pueden hacer desde la clase del actor.

\subsubsection{Creación de un escenario con \emph{Tiled}}

El proceso de creación de un escenario a partir de la herramienta \emph{Tiled} es muy sencillo. En su página oficial, \emph{http://www.mapeditor.org/}, podemos descargar la última versión o, dependiendo de si el sistema en el que nos encontremos (en mi caso al redactar este manual, una Ubuntu 10.10) dispone de esta utilidad en los repositorios, se puede instalar con la siguiente orden:

\begin{lstlisting}[style=consola]
sudo apt-get install tiled
\end{lstlisting}

Una vez instalado, ejecutamos \emph{Tiled} y pulsamos en el botón \emph{Nuevo}. Se abrirá una ventana en la que se nos preguntan algunos parámetros del nuevo mapa de tiles. Seleccionamos proyección ortogonal, el ancho y alto en píxeles de cada tile, y el ancho y alto medido en número de tiles que tendrá el escenario. Las medidas de un tile deben ser múltiplos de 8 píxeles, debido a los requisitos para la carga de texturas con \emph{LibWiiEsp}, siendo un tamaño recomendable es 32x32 píxeles por tile. La resolución que utiliza la Nintendo Wii en un sistema PAL y proporción 4:3 es de 640 píxeles de ancho por 528 de alto, por lo que el tamaño mínimo del escenario debería ser (si utilizamos la medida 32x32 píxeles por tile) de 20 tiles de ancho, por 16 tiles de alto. Una vez establecidos estos parámetros, pulsamos en aceptar.\\

A continuación, hay que cargar en la herramienta la imagen que contiene el tileset. Para ello, pulsamos en \emph{Mapa}, y después en \emph{Nuevo conjunto de patrones}. Se abrirá un diálogo en el que debemos introducir el nombre que le daremos a la imagen (esto es un dato interno de \emph{Tiled}, y el valor introducido aquí no es relevante), la propia imagen, y las medidas que tendrá un tile (ancho y alto en píxeles). Un detalle a tener en cuenta, en nuestra imagen no debe haber separación entre los tiles, ni tampoco margen. Al introducir estos datos, hacemos clic en \emph{Aceptar} y ya tendremos el tileset listo para dibujar el mapa.\\

Como siguiente punto, hay que introducir las tres propiedades del escenario. Nos vamos a \emph{Mapa} y entonces a \emph{Propiedades del Mapa}, e introducimos en la lista estas propiedades:

\begin{itemize}
\item imagen\_fondo: Su valor es el código que tendrá la imagen de fondo del escenario en la galería de recursos del sistema.
\item imagen\_tileset: Su valor es el código que tendrá la imagen del tileset del escenario en la galería de recursos del sistema.
\item musica: Su valor es el código que tendrá la pista de música del escenario en la galería de recursos del sistema.
\end{itemize}

Tras introducir las propiedades, pulsamos en \emph{Aceptar}, y procedemos a preparar las tres capas de dibujo de nuestro escenario, que deben llamarse obligatoriamente como se describe en la lista:

\begin{itemize}
\item escenario: Capa de patrones donde dibujaremos los tiles atrabvesables.
\item plataformas: Capa de patrones donde dibujaremos los tiles no atrabvesables, es decir, los que tendrán asociada una figura de colisión.
\item actores: Capa de objetos donde situaremos todos los actores que participen en el nivel.
\end{itemize}

En este punto, ya estamos en disposición de comenzar a dibujar nuestro escenario. Es recomendable comenzar por la capa de tiles no atravesables, continuar decorando con los tiles atravesables, y por último, establecer la posición de los actores. Para situar un actor en el escenario, insertamos un objeto en el lugar que deseemos, y lo redimensionamos adecuadamente para que se vea correctamente cómo quedaría su posición inicial. Seguidamente, hacemos clic con el botón derecho en el actor, y pulsamos en \emph{Propiedades del objeto}. Se mostrará un diálogo en el que debemos indicar, en el campo \emph{Tipo}, el tipo del actor que comenzará aquí (dato necesario a la hora de distinguir qué clase derivada de \emph{Actor} hay que instanciar), y además la propiedad \emph{xml}, cuyo valor debe ser la ruta absoluta hasta el archivo XML que contiene la información de este tipo de actor en la tarjeta SD de la Nintendo Wii. Además, si el actor es controlado por un jugador, debemos añadir también la propiedad \emph{jugador} con el código identificativo del jugador que jugará con este actor como valor.\\

A la hora de definir los distintos actores, se puede utilizar el campo \emph{Nombre} del diálogo de propiedades de un actor para identificarlo en tiempo de diseño del escenario.\\

\subsubsection{Implementando una clase que controle escenarios}

Una vez tenemos creados uno o varios escenarios cuyo comportamiento (la gestión de transiciones entre los estados de los distintos actores que participan en él, el movimiento o no del \emph{scroll} de la pantalla sobre el escenario, etc.) es común, el último paso para poder disfrutar de ellos es definir una clase derivada de la abstracta \emph{Nivel} que controle todos los detalles del nivel. Cada clase derivada definirá una gestión distinta de un grupo de varios escenarios.\\

El primer paso para crear una clase derivada de la abstracta \emph{Nivel} es implementar el método virtual \emph{cargarActores()}. Como se detalla en la documentación de la clase, en el constructor se cargan todos los tiles del escenario, se toma la imagen de fondo y del tileset desde la galería, y se leen los datos de inicialización de cada actor participante, almacenándose en una estructura temporal. La definición de este método debe recorrer esta estructura temporal de datos de actores, creando, para cada ocurrencia, un actor de la clase derivada de \emph{Actor} correspondiente, tal y como se aprecia en el siguiente ejemplo:

\begin{lstlisting}[style=C++]
void cargarActores(void) {
  for(Temporal::iterator i = _temporal.begin(); i != _temporal.end(); ++i)
  {
    if(i->tipo_actor == "jugador") {
      Personaje* p = new Personaje(i->xml, this);
      p->mover(i->x, i->y);
      _jugadores.insert(std::make_pair(i->jugador, p));
    } else if(i->tipo_actor == "bicho") {
      Bicho* p = new Bicho(i->xml, this);
      p->mover(i->x, i->y);
      _actores.push_back(p);
    }
  }
  _temporal.clear();
};
\end{lstlisting}

En el ejemplo, el programador ha definido dos clases derivadas de \emph{Actor}, denominadas \emph{Personaje} y \emph{Bicho}. En el método implementado, se recorre la estructura temporal de datos de actores del escenario, y se crea un actor a partir de la clase correspondiente (según el tipo de actor que se haya indicado desde \emph{Tiled}), se mueve el actor hasta su posición en el escenario, y por último se inserta en la estructura adecuada (\emph{\_jugadores} en el caso de los actores controlados por un jugador, en la que hay que indicar también el código identificador del jugador concreto; o \emph{\_actores} para los actores controlados por la máquina).\\

Es importante recordar que este método \emph{cargarActores()} se debe llamar desde el constructor de la clase derivada de \emph{Nivel} con la intención de que la creación de los actores se realice en el momento de cargar el escenario. Además, es recomendable vaciar la estructura temporal cuando se finalice el proceso.\\

El siguiente paso en la generación de esta clase derivada es implementar los métodos de actualización del nivel, que se deberían llamar en cada fotograma del programa. A continuación se indican cuáles son, y qué funcionalidad se espera que tengan.\\

En el método \emph{actualizarPj()} se debe actualizar el estado de un único actor jugador, atendiendo tanto al mando concreto que tenga asociado en la estructura \emph{\_jugadores}, como a la situación del escenario. Un ejemplo sencillo podría ser:

\begin{lstlisting}[style=C++]
void actualizarPj(const std::string& jugador, const Mando& m) {

  // Estado NORMAL: puede pasar a MOVER
  if(_jugadores[jugador]->estado() == "normal") {
    if(m.pressed(Mando::BOTON_ARRIBA) or m.pressed(Mando::BOTON_ABAJO))
      _jugadores[jugador]->setEstado("mover");
  }

  // Estado MOVER: puede pasar a NORMAL
  if(_jugadores[jugador]->estado() == "mover") {
    if(m.pressed(Mando::BOTON_ARRIBA)) {
      _jugadores[jugador]->invertirDibujo(true);
      s16 vel_x = _jugadores[jugador]->velX();
      if(vel_x > 0)
        vel_x *= -1;
      _jugadores[jugador]->setVelX(vel_x);
    } else if(m.pressed(Mando::BOTON_ABAJO)) {
      _jugadores[jugador]->invertirDibujo(false);
      s16 vel_x = _jugadores[jugador]->velX();
      if(vel_x < 0)
        vel_x *= -1;
      _jugadores[jugador]->setVelX(vel_x);
    } else
      _jugadores[jugador]->setEstado("normal");
  }

  // Actualizar el actor en base a su nuevo estado actual
  _jugadores[jugador]->actualizar();
}
\end{lstlisting}

Por otro lado, el método \emph{actualizarNpj()} debe recorrer la estructura en la que se almacenan los actores controlados por la máquina y actualizar los que se consideren oportunos (aquí se deja en manos del programador el actualizar todos los actores, sólo los que están en pantalla, o los que cumplan un determinado criterio). Como ejemplo, se muestra la siguiente función que actualizaría el estado de todos los actores no jugadores:

\begin{lstlisting}[style=C++]
void actualizarNpj(void) {
  for(Actores::iterator i = _actores.begin() ; i != _actores.end() ; ++i) {
    // Estado NORMAL: puede pasar a CAER
    if((*i)->estado() == "normal")
      if(not colision((*i)))
        (*i)->setEstado("caer");

    // Estado CAER: puede pasar a NORMAL
    if((*i)->estado() == "caer")
      if(colisionVertical((*i)))
        (*i)->setEstado("normal");

    // Actualizacion del actor
    (*i)->actualizar();
  }
};
\end{lstlisting}

El último método a implementar es \emph{actualizarEscenario()}, en el que se espera que se implementen todos los demás detalles relativos al escenario que necesiten ser actualizados a cada fotograma del juego. El siguiente ejemplo muestra una implementación que únicamente actualiza el \emph{scroll} de la pantalla sobre el escenario, según la posición horizontal del jugador cuyo código identificador es \emph{pj1}:

\begin{lstlisting}[style=C++]
void actualizarEscenario(void) {

  if(_jugadores["pj1"]->x() - _scroll_x >= screen->ancho() / 2)
    moverScroll(_scroll_x + abs(_jugadores["pj1"]->velX()), _scroll_y);
  else if(_jugadores["pj1"]->x() - _scroll_x <= screen->ancho() / 4)
    moverScroll(_scroll_x - abs(_jugadores["pj1"]->velX()), _scroll_y);

  if(_jugadores["pj1"]->y() - _scroll_y >= screen->alto() / 2)
    moverScroll(_scroll_x, _scroll_y + abs(_jugadores["pj1"]->velY()));
  else if(_jugadores["pj1"]->y() - _scroll_y <= screen->alto() / 4)
    moverScroll(_scroll_x, _scroll_y - abs(_jugadores["pj1"]->velY()));
};
\end{lstlisting}

Por supuesto, quiero remarcar que los métodos de ejemplo son precisamente eso, ejemplos muy sencillos cuya finalidad es que sirvan de guía para comprender cómo se trabaja con la plantilla de niveles de \emph{LibWiiEsp}, y a partir de los cuales poder desarrollar los métodos de actualización necesarios (al derivar la clase abstracta \emph{Nivel} se pueden añadir los métodos que se consideren necesarios).\\

