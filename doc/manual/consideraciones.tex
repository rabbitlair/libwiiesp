% Este archivo es parte de libWiiEsp. Copyright (C) 2011 Ezequiel Vázquez de la Calle
% Licencia GFDL. Encontrará una copia de la licencia en el archivo fdl-1.3.tex

A la hora de programar para una plataforma cerrada como es Nintendo Wii, hay que tener una serie de aspectos en cuenta, debido a la arquitectura concreta que tiene el hardware de la consola. Es muy importante tener en cuenta estos detalles, ya que en caso contrario se pueden producir errores o comportamientos inesperados, como texturas que se pisan entre sí en la memoria principal, o excepciones propias del sistema. \\

A continuación se describen todos esos pequeños detalles a tener en cuenta, que si bien sólo suponen un cambio en algunos hábitos a la hora de programar, nos aseguran que todo irá bien (al menos, en lo que al funcionamiento del hardware se refiere).

\subsection{Big Endian}

Lo primero que hay que tener en cuenta es que la representación de los datos en la consola Nintendo Wii se realiza con Big Endian, al contrario que las plataformas Intel (la arquitectura, no la marca del procesador), que utilizan Little Endian para almacenar la información. Cuando un dato ocupa más de un byte, se puede organizar de mayor a menor peso (esto sería Big Endian), o bien de menor a mayor peso (que es la organización en Little Endian). \\

Un ejemplo muy claro es la representación del \emph{número mágico} que emplean los archivos de imágenes formadas por mapas de bits (también conocidos como \emph{bitmaps}). Este número es, en su representación decimal, 19778. Si esta cifra la representamos en sistema hexadecimal con Little Endian el resultado sería 0x4D42; sin embargo, en un sistema que utilice Big Endian, esta cifra se representaría como el hexadecimal 0x424D en la memoria principal. \\

Así pues, como consecuencia de esto, siempre que queramos cargar un archivo binario (una imagen, una pista de música, etc.) en la consola Nintendo Wii, tenemos dos opciones: o bien modificamos la representación del archivo en la plataforma de origen (que normalmente será un ordenador con arquitectura Intel) para que el recurso esté ya representado como Big Endian, o podemos utilizar las funciones del espacio de nombres \emph{endian} que se proporcionan en el archivo de cabecera \emph{util.h} de \emph{LibWiiEsp}. Este espacio de nombre aporta las siguientes dos funciones, que se encargan de transformar variables de 16 o 32 bits entre Little Endian y Big Endian (soportan ambas transformaciones):

\begin{lstlisting}[style=C++]
u16 inline swap16(u16 a)
{
  return ((a<<8) | (a>>8));
}

u32 inline swap32(u32 a)
{
  return ((a)<<24|(((a)<<8) & 0x00FF0000)|(((a)>>8) & 0x0000FF00)|(a)>>24);
}
\end{lstlisting}

Así pues, únicamente utilizando estas dos funciones con cada dos o cuatro bytes cuando queramos cargar un recurso desde la tarjeta SD podremos evitar los problemas derivados de los distintos sistemas de representación. Para un ejemplo práctico sobre el uso de estas dos funciones, ver el código fuente de la clase Imagen de \emph{LibWiiEsp}.

\subsection{Los tipos de datos}

A la hora de programar para Nintendo Wii se utilizan siempre estos tipos de datos:

\begin{table}[H]
  \label{tiposdatos}
  \begin{center}
  \begin{tabular}{| c | c | c |}
    \hline
    Tipo de dato & Descripción & Rango \\ \hline
    u8 & Entero de 8 bits sin signo & 0 a 255 \\ \hline
    s8 & Entero de 8 bits con signo & -127 a 128 \\ \hline
    u16 & Entero de 16 bits sin signo & rango 0 a 65535 \\ \hline
    s16 & Entero de 16 bits con signo & -32768 a 32767 \\ \hline
    u32 & Entero de 32 bits sin signo & 0 a 0xffffffff \\ \hline
    s32 & Entero de 32 bits con signo & -0x80000000 a 0x7fffffff \\ \hline
    u64 & Entero de 64 bits sin signo & 0 a 0xffffffffffffffff \\ \hline
    s64 & Entero de 64 bits con signo & -0x8000000000000000 a 0x7fffffffffffffff  \\ \hline
    f32 & Flotante de 32 bits & - \\ \hline
    f64 & Flotante de 64 bits & - \\ \hline
  \end{tabular}
  \end{center}
  \caption{Tipos de datos que se deben emplear al programar para Wii}
\end{table}

Todos los tipos de datos disponibles se encuentran en \emph{/opt/devkitpro/libogc/include/gctypes.h}.

\subsection{La alineación de los datos}

Otro detalle importante que se debe contemplar a la hora de programar para Nintendo Wii es que el procesador de la consola necesita alinear los datos conforme a su tamaño. Es decir, si vamos a trabajar con un entero de 32 bits (4 bytes), sólo se pueden almacenar en posiciones de memoria múltiplos de cuatro: 0, 4, 8, 12\ldots Lo mismo ocurre con variables de 16 o 64 bits. \\

Para ilustrar el comportamiento del procesador respecto a la alineación de los datos, podemos considerar el siguiente ejemplo:

\begin{lstlisting}[style=C++]
struct ejemplo {
  u8 entero;
  u32 otro;
}
\end{lstlisting}

La estructura \emph{ejemplo} no se representará igual compilando en una plataforma Intel que en la Wii, ya que en la primera, \emph{entero} se situará en la posición 0 de la memoria, y \emph{otro} ocupará de la posición 1 hasta la 5. Sin embargo, en nuestra consola \emph{entero} ocupará el mismo lugar, la posición 0, pero la variable \emph{otro} necesitará estar alineada a su tamaño, es decir, ocupará las posiciones 4 a 7 de la memoria, y las posiciones entre \emph{entero} y \emph{otro} se rellenarán con ceros para asegurar que el entero de 32 bits está alineado. \\

Esto revierte tanto en la cantidad de memoria ocupada, como en el hecho de que si el procesador encontrara un dato desalineado e intentara leerlo, se produciría un error en el sistema. \\

Para solventar este inconveniente, se debe jugar inteligentemente con las declaraciones de las variables, de tal manera que se organicen los datos de forma alineada.

\subsection{Alineación de datos que usan DMA}

Este es otro caso en el que influye la necesidad de alinear los datos con los que se trabaja. El procesador de la Nintendo Wii trabaja con datos cacheados, pero no así los periféricos (la tarjeta SD, los dispositivos USB o el lector DVD). Además, estos periféricos trabajan con una alineación fija de 32 bytes, y si no se contempla este detalle, se pueden producir errores de machacamiento de datos al realizar dos lecturas consecutivas desde periféricos. \\

Además, cuando leemos un dato desde un periférico, se corre el riesgo de machacar el contenido de la caché del procesador, por lo que es imprescindible volcar explícitamente los datos leídos en la memoria, es decir, asegurar de que la lectura se ha realizado completamente antes de realizar cualquier otra acción. \\

Es muy sencillo evitar esta situación, y es preparando la memoria en la que se almacenarán los datos leídos desde el periférico, de tal manera que esté alineada a 32 bytes y su tamaño sea múltiplo de 32. Para ilustrar cómo hacer esto, se muestra un ejemplo a continuación:

\begin{lstlisting}[style=C++]
  // Abrimos el archivo mediante un flujo
  ifstream archivo;
  archivo.open("SD:/apps/wiipang/media/sonido.pcm", ios::binary);

  // Obtener el tamano del sonido
  archivo.seekg(0, ios::end);
  u16 size = archivo.tellg();
  archivo.seekg(0, ios::beg);

  // Calcular el relleno que hay que aplicar a la memoria reservada
  // para que su tamano sea multiplo de 32
  u8 relleno = (size * sizeof(s16)) % 32;

  // Reservar memoria alineada: utilizamos memalign en lugar de malloc
  s16* sonido = (s16*)memalign(32, size * sizeof(s16) + relleno);

  // Realizar la lectura de datos desde el periferico
  // Utilizar char* como tipo de lectura es por compatibilidad con libFat
  archivo.read((char*)sonido, size);

  // Fijar los datos leidos en la memoria, evita machacamiento en la cache
  DCFlushRange(sonido, size * sizeof(s16) + relleno);
\end{lstlisting}

Como puede verse en el ejemplo, se calcula en primer lugar el relleno necesario para que la memoria que ocupa el archivo binario a cargar sea múltiplo de 32 bytes, a continuación se reserva la memoria alineada a 32 bytes utilizando la función \emph{memalign} en lugar de \emph{malloc}, y el último paso, tras leer la información desde el archivo, consiste en realizar el volcado explícito de información desde la caché de lectura a la memoria, utilizando la función \emph{DCFlushRange} (esta función reibe la dirección de memoria a la que se quiere realizar el volcado, y el tamaño de ésta).

\subsection{Depuración con \emph{LibWiiEsp}}

Con toda la información anterior descrita en este manual y la referencia completa de la biblioteca, se puede comenzar a desarrollar un videojuego sencillo. Como en todo proceso de desarrollo de software, ocurrirán errores, y aquí es donde se hace patente la falta de medios disponibles para depurar el código.\\

Existen dos tipos de errores que podremos encontrarnos a la hora de programar para Nintendo Wii, que corresponden con las fases de compilación y ejecución. A continuación, vamos a plantear cómo solucionar los posibles errores que pueden surgir en cada uno de estos momentos:

\subsubsection{Errores de compilación}

En la fase de compilación suelen ocurrir, sobretodo, errores de sintaxis, aunque también es posible que ocurran errores de enlazado si las rutas hasta los archivos de cabeceras o las bibliotecas de enlace estático no son correctas. El compilador nos avisará de cualquier tipo de error que ocurra, tanto en el preprocesado, como en la compilación propiamente dicha y en el enlazado. Así pues, prestando atención a los mensajes que pueda proporcionarnos el compilador sobre los errores o avisos que se den, podremos depurar el código fuente de nuestra aplicación.\\

Sobre los mensajes de enlazado, el enlazador nos proporcionará información suficiente para saber qué ha fallado durante esta operación, pero si se sigue al pie de la letra este manual (especialmente, la instalación del entorno y la creación del \emph{makefile}) no debería haber ningún problema.

\subsubsection{Errores de ejecución}

En la fase de ejecución es cuando tenemos verdaderos problemas para depurar nuestra aplicación, y es que apenas hay herramientas que puedan facilitarnos el conocer el estado de los objetos del sistema, el contenido de las variables, etc.\\

\emph{LibWiiEsp} proporciona un sistema de registro de eventos del sistema, la clase \emph{Logger}, que permite que escribamos un log de errores, avisos e información variada. La forma de trabajar con esta clase es muy sencilla, y todos los detalles pueden consultarse en su documentación (ver referencia completa de la biblioteca). Pero hay ocasiones en que los errores en tiempo de ejecución requieren más precisión que una serie de mensajes que aportemos desde nuestro propio código, ya que el uso de la clase \emph{Logger} está recomendado en casos de comportamiento inesperado del programa, pero no de errores como tal.\\

\figura{excepcion.png}{scale=0.5}{Ejemplo de pantalla de error en tiempo de ejecución}{excepcion}{H}

Cuando ocurre un error de ejecución en la Nintendo Wii, ésta nos presentará una pantalla de error parecida a la imagen de la figura 1. En esta pantalla de error pueden apreciarse tres partes principales. A la hora de localizar en qué parte de nuestro código se ha provocado este error, necesitamos fijarnos en la sección \emph{STACK DUMP}, es decir, la segunda. Esta parte del mensaje de error nos detalla la traza de llamadas a función que se han realizado hasta llegar a la llamada que ha producido el error, estando cada elemento de la traza representado por una dirección de memoria en hexadecimal.\\

Esta informacion es muy útil, ya que podemos localizar a qué línea de nuestro código fuente corresponde cada llamada con una utilidad incluida en \emph{DevKitPPC}, y esta es la utilidad \emph{powerpc-eabi-addr2line}. Se puede localizar en la carpeta \emph{/opt/devkitpro/devkitPPC/bin} si se ha seguido este manual para instalar \emph{LibWiiEsp}. Lo más cómodo es crear un enlace simbólico a esta utilidad en el directorio principal de nuestro proyecto, aunque también se puede añadir el directorio \emph{devkitPPC/bin} a la ruta donde el sistema busca ejecutables.\\

Siguiendo el ejemplo de la figura 1, si queremos saber a qué archivo y qué línea de código pertenece la dirección \emph{0x80011f54}, basta con ejecutar la siguiente línea de código en el directorio principal del proyecto:

\begin{lstlisting}[style=consola]
  ./powerpc-eabi-addr2line -e boot.elf -f 0x80011f54
\end{lstlisting}

Para que esta utilidad funcione, debemos tener en el directorio donde la ejecutamos una copia del ejecutable \emph{.elf} generado por nuestro proyecto (el mismo que hemos ejecutado en la consola y que nos ha dado el mensaje de error de la Figura 1). El parámetro \emph{-e} recibe el nombre de este ejecutable \emph{.elf}, y el parámetro \emph{-f} es la dirección de memoria (en hexadecimal) que ha producido el error.\\

Un ejemplo del resultado de la ejecución de la utilidad \emph{addr2line} puede ser el siguiente:

\begin{lstlisting}[style=consola]
  Actor
  /home/rabbit/Escritorio/libwiiesp/src/actor.cpp:27
\end{lstlisting}

Lo cual nos indica que el error se ha producido en la línea 27 del fichero fuente actor.cpp de \emph{LibWiiEsp}.\\

Un detalle a comentar es que, para salir de la pantalla de error que se muestra en la Figura 1, basta con pulsar el botón \emph{Reset} de la consola, lo que nos devolverá al \emph{HomeBrew Channel}. \\

En resumen, con la herramienta \emph{addr2line} y la clase \emph{Logger} podemos ir realizando una decente depuración de nuestra aplicación, que aunque hay que reconocer que no es muy cómodo, es mejor que ir dando palos de ciego.

\clearpage

