% Este archivo es parte de libWiiEsp. Copyright (C) 2011 Ezequiel Vázquez de la Calle
% Licencia GFDL. Encontrará una copia de la licencia en el archivo fdl-1.3.tex

\section*{Licencia}

Este documento ha sido liberado bajo Licencia GFDL 1.3 (GNU Free
Documentation License). Se incluyen los términos de la licencia en
inglés al final del mismo.\\

Copyright (c) 2011 Ezequiel Vázquez de la Calle.\\

Permission is granted to copy, distribute and/or modify this document under the
terms of the GNU Free Documentation License, Version 1.3 or any later version
published by the Free Software Foundation; with no Invariant Sections, no
Front-Cover Texts, and no Back-Cover Texts. A copy of the license is included in
the section entitled "GNU Free Documentation License".\\

\clearpage

\tableofcontents
\clearpage
\listoffigures
\listoftables
\clearpage

\begin{abstract}
Debido a las diferencias existentes entre el desarrollo para un PC ordinario y una videoconsola, en este caso, para Nintendo Wii, se hace patente la necesidad de recoger todos esos detalles que, siendo sencillos de solventar, pueden suponer más de un quebradero de cabeza a un programador sin ningún tipo de experiencia en la programación para Wii. Además, la propia construcción de \emph{LibWiiEsp} implica una serie de consideraciones a tener en cuenta a la hora de sacarle el máximo rendimiento. El objetivo de este documento es recoger todos los pormenores que un programador debe tener en cuenta a la hora de crear un videojuego para la consola, utilizando \emph{LibWiiEsp} como herramienta.
\end{abstract}

