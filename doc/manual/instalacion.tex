% Este archivo es parte de libWiiEsp. Copyright (C) 2011 Ezequiel Vázquez de la Calle
% Licencia GFDL. Encontrará una copia de la licencia en el archivo fdl-1.3.tex

Para comenzar el desarrollo de un videojuego para Wii utilizando como herramienta \emph{LibWiiEsp}, es necesario instalar en el sistema todas las dependencias que la biblioteca necesita. \\

Esta sección del manual detalla paso a paso la instalación de estas dependencias en un sistema \emph{GNU/Linux} de 32 bits, siendo el proceso prácticamente el mismo para sistemas de 64 bits (únicamente cambia la versión de las herramientas base, que deberá ser en este caso la adecuada para 64 bits). Para sistemas Windows también es posible desarrollar con \emph{LibWiiEsp}, pero este manual no cubre este tipo de arquitecturas. \\

En primer lugar, hay que crear una carpeta accesible para todos los usuarios del sistema, donde irán emplazadas las herramientas y dependencias de \emph{LibWiiEsp}. Lo ideal es crear un directorio dentro de \texttt{/opt} y otorgarle a éste los permisos necesarios, pero es posible realizar la instalación en nuestro directorio \texttt{/home}, de tal manera que sólo nuestro usuario pueda acceder a estas herramientas. \\

Suponiendo que se escoge la primera opción, se crea el directorio en \texttt{/opt} y se le asignan permisos para que todos los usuarios del sistema puedan hacer uso de lo que en él se almacene:

\begin{lstlisting}[style=consola]
  sudo mkdir /opt/devkitpro
  sudo chmod 777 /opt/devkitpro
\end{lstlisting}

A continuación, hay que descargar las herramientas que sirven de base para \emph{LibWiiEsp}, que son el conjunto de compiladores, bibliotecas y binarios \emph{DevKitPPC}, y la biblioteca de bajo nivel \emph{libOgc}, en su versión 1.8.4, junto con la modificación de \emph{libFat} 1.0.7 compatible con ella. Ambos recursos se encuentran disponibles en la forja del proyecto, en el paquete \emph{Dependencias}. Se deben descargar en el directorio que acabamos de crear:

\begin{lstlisting}[style=consola]
  cd /opt/devkitpro
  wget http://forja.rediris.es/frs/download.php/2316/devkitPPC_r21-i686-linux.tar.bz2
  wget http://forja.rediris.es/frs/download.php/2315/libogc-1.8.4-and-libfat-1.0.7.tar.gz
\end{lstlisting}

El siguiente paso es descomprimir ambos ficheros y, si no queremos tener ocupado espacio innecesariamente, eliminarlos. Las instrucciones para ejecutar estas acciones son:

\begin{lstlisting}[style=consola]
  tar -xvjf devkitPPC_r21-i686-linux.tar.bz2
  tar -xvzf libogc-1.8.4-and-libfat-1.0.7.tar.gz
  rm devkitPPC_r21-i686-linux.tar.bz2 libogc-1.8.4-and-libfat-1.0.7.tar.gz
\end{lstlisting}

Después de esto, es necesario establecer algunas variables de entorno para que el sistema sepa dónde localizar estas herramientas que acabamos de instalar. Las dos primeras consisten en la ruta hasta el directorio base \emph{devkitpro}, y hasta el directorio donde se encuentran las herramientas como tal, concretamente \emph{devkitpro/devkitPPC}. La tercera variable de entorno indica la dirección IP de nuestra Nintendo Wii dentro de la red local, dato necesario para ejecutar correctamente las aplicaciones que se desarrollen sin que haga falta copiar el ejecutable en la tarjeta SD de la consola cada vez que se genere uno nuevo. \\

Para establecer estas variables de entorno, basta con editar el archivo de configuración \emph{~/.bashrc} del usuario con el que vayamos a desarrollar utilizando \emph{LibWiiEsp}. Si por alguna causa fuera necesario que todos los usuarios del sistema tengan que usar estas herramientas, el archivo de configuración a editar sería \emph{/etc/bashrc}, o su equivalente en el sistema (por ejemplo, para una distribución \emph{Ubuntu 10.10} de 32 bits, el archivo es \emph{/etc/bash.bashrc}).

Suponiendo que sólo nuestro usuario va a desarrollar utilizando \emph{LibWiiEsp}, las instrucciones para establecer estas variables de entorno serían:

\begin{lstlisting}[style=consola]
  echo export DEVKITPRO=/opt/devkitpro >> ~/.bashrc
  echo export DEVKITPPC=$DEVKITPRO/devkitPPC >> ~/.bashrc
  echo export WIILOAD=tcp:192.168.X.X >> ~/.bashrc
\end{lstlisting}

Un detalle, en estas instrucciones, \emph{192.168.X.X} se debe sustituir por la dirección IP que tiene asignada la consola Nintendo Wii en la red local.

A continuación, podemos recargar el fichero de configuración \emph{~/.bashrc} para tener listas las varibles de entorno mediante la orden:

\begin{lstlisting}[style=consola]
  source ~/.bashrc
\end{lstlisting}

Llegados a este punto, el último paso para que el entorno de desarrollo funcione correctamente con \emph{LibWiiEsp} es instalar la biblioteca propiamente dicha. Existen dos maneras de realizar esto, o bien podemos descargar la versión estable publicada en la forja, o también compilando el código disponible desde ésta. Para el primer caso, basta con descargar el paquete comprimido desde la página principal de la Forja del proyecto (http://forja.rediris.es/projects/libwiiesp), copiarlo al directorio \emph{/opt/devkitpro}, y descomprimirlo allí. \\

Para compilar la biblioteca a partir de los fuentes, debemos ejecutar las siguientes líneas, siempre que se hayan instalado previamente las dependencias:

\begin{lstlisting}[style=consola]
  svn co https://forja.rediris.es/svn/libwiiesp/trunk libwiiesp
  cd libwiiesp
  make install
\end{lstlisting}

Y después de la instalación de la biblioteca, ya tenemos preparado nuestro sistema para comenzar con el desarrollo de videojuegos en dos dimensiones para Nintendo Wii.

\clearpage

