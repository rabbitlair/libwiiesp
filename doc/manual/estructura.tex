% Este archivo es parte de libWiiEsp. Copyright (C) 2011 Ezequiel Vázquez de la Calle
% Licencia GFDL. Encontrará una copia de la licencia en el archivo fdl-1.3.tex

Una vez instalado correctamente el entorno de desarrollo para utilizar \emph{LibWiiEsp}, el siguiente paso es crear la estructura de archivos y directorios necesarios para trabajar con un nuevo proyecto para Nintendo Wii. Por supuesto, queda en manos del programador la decisión final sobre esta estructura de directorios, pero, debido a todo lo que hay que tener en cuenta a la hora de desarrollar con \emph{LibWiiEsp} (sobretodo, en lo referente a la compilación y enlazado de los archivos que componen un proyecto), voy a presentar en esta sección un ejemplo de proyecto vacío que quedaría listo para comenzar el desarrollo. Además de tratar la estructura de directorios, también mostraré y explicaré el código de un archivo \emph{makefile} que pueda trabajar con esta organizción para el proyecto.

\subsection{Estructura de directorios}

En primer lugar, hay que crear un directorio base donde almacenar todo lo relativo a nuestro proyecto. La localización de este directorio en el sistema es indiferente, de tal manera que, para el ejemplo, lo haremos en nuestro directorio \emph{/home} con las instrucciones:

\begin{lstlisting}[style=consola]
  mkdir ~/juego
  cd ~/juego
\end{lstlisting}

Una vez dentro, la siguiente estructura de directorios sería más que suficiente para albergar todos los componentes del proyecto:

\begin{itemize}
\item \emph{doc}: Documentación del proyecto.
\item \emph{lib}: Bibliotecas externas que se vayan a utilizar en el proyecto. Aquí se debe guardar cada biblioteca en un directorio separado. En cada directorio debe existir un archivo \emph{makefile} el cual compile la biblioteca, y genere un archivo de enlazado estático \emph{.a}, además los archivos de cabecera de la biblioteca externa tienen que estar justo bajo este directorio principal de la biblioteca externa.
\item \emph{media}: En este directorio se colocarán todos los archivos que contengan recursos multimedia que vayan a ser empleados en el proyecto. De momento, sólo están soportados imágenes \emph{bitmap} de 24 bits de color directo, fuentes de texto soportadas por \emph{FreeType2}, efectos de sonido en formato \emph{PCM}, y pistas de música \emph{mp3}.
\item \emph{src}: Aquí van los archivos fuente del proyecto.
\item \emph{xml}: Archivos de datos del proyecto. Como mínimo, aquí se encontrarán el archivo que describe los recursos multimedia que se cargarán en la galería de medias, el archivo del soporte de idiomas, y el archivo de configuración del programa.
\end{itemize}

\subsection{El archivo \emph{makefile}}

La estructura de directorios, y los detalles que la acompañan (como las restricciones de estructuración a la hora de utilizar bibliotecas externas en el proyecto), se han definido así para trabajar con un archivo de compilación \emph{makefile} concreto. \\

Este archivo de compilación implica una serie de modificaciones considerables en comparación con uno que genere un ejecutable para PC en entornos \emph{GNU/Linux}, por lo que va a detallarse su funcionamiento sección a sección. El objetivo de este \emph{makefile} es generar un ejecutable con extensión \emph{.dol}, que puede lanzarse en una videoconsola Nintendo Wii. A continuación, se muestra un sencillo ejemplo de \emph{makefile} compatible con la estructura de directorios planteada en el punto anterior:

\lstinputlisting[style=makefile]{make_ejemplo}

En el primer bloque que aparece en el ejemplo, se definen una serie de variables que indican los directorios que forman parte del proyecto. \emph{LOCALLIBS} debe recoger los nombres, separados por un espacio, de los directorios principales de cada biblioteca externa que se vaya a emplear en la compilación. La variable \emph{PROJECT} indica el nombre completo del proyecto, y \emph{TARGET} el nombre, sin extensión, del ejecutable que se generará. Para que el programa pueda ser ejecutado con \emph{HomeBrew Channel}, se recomienda que se nombre \emph{boot}. Por último, \emph{BUILD} indica el directorio que se creará cuando se ordene la compilación del proyecto para almacenar todos los ficheros objeto del proyecto, \emph{SOURCE} es el directorio donde están todos los archivos fuente, y \emph{DEPSDIR} es donde se generarán los archivos que indican las dependencias entre módulos del sistema, que debe ser el mismo directorio que \emph{BUILD}. \\

A continuación, se deben importar las reglas de compilación para Nintendo Wii, tanto las necesarias para utilizar las herramientas de \emph{DevKitPPC}, como las propias de \emph{LibWiiEsp}. Para ello, antes hay que limpiar las reglas implícitas existentes, lo cual se consigue con la instrucción \emph{.SUFFIXES:}. \\

El bloque de tres instrucciones que viene a continuación se encarga de almacenar en variables una lista de todos los archivos \emph{.cpp} que se encuentran en el directorio de fuentes, y otra lista con la ruta absoluta de los ficheros objeto que se generarán al compilar (es decir, archivos con extensión \emph{.o} en la carpeta indicada por la variable \emph{BUILD}). \\

El cuarto bloque de instrucciones recoge los directorios donde se encuentran los archivos de cabeceras externas (los almacena en la variable \emph{INCLUDE}), los directorios en los cuales hay que buscar las bibliotecas de enlazado estático (guardados en la variable \emph{LIBPATHS}), establece el \emph{VPATH}, crea la lista de bibliotecas a enlazar estáticamente (tomando la variable \emph{LIBWIIESP\_LIBS} para contar con todo lo que necesita un ejecutable generado con \emph{LibWiiEsp}, pero se le debe añadir además lo necesario para enlazar también las bibliotecas externas), y por último, indica la ruta absoluta hasta el ejecutable sin extensión y define el enlazador que se utilizará. \\

El bloque siguiente establece los flags de compilación necesarios para generar un ejecutable de extensión \emph{.elf}, que posteriormente se transformará a otro con extensión \emph{.dol}. Y justo después comienzan los objetivos del \emph{makefile}, que se describen a continuación:

\begin{itemize}
\item \emph{all}: Crea un ejecutable \emph{.dol} a partir del código fuente que se encuentre en el directorio indicado en la variable \emph{SOURCE}. Es el objetivo por defecto al ejecutar \emph{make}.
\item \emph{\$(BUILD)}: Crea el directorio donde se crearán todos los ficheros objetos del proyecto.
\item \emph{libs} y \emph{\$(LOCALLIBS)}: Se encargan de compilar las bibliotecas externas, de empaquetarlas en ficheros de enlace estático y mover éstos al directorio de los ficheros objeto.
\item \emph{\$(CURDIR)/\$(BUILD)/\%.o}: Genera un fichero objeto en el directorio \emph{\$(BUILD)} por cada módulo que se encuentre en el directorio de los ficheros fuentes.
\item \emph{listo}: Objetivo que limpia un archivo de extensión \emph{.elf.map}.
\item \emph{run}: Lanza la utilidad \emph{wiiload} para enviar el ejecutable a la Nintendo Wii. Ésta debe tener abierto el \emph{HomeBrew Channel} y estar conectada a la red local con la misma dirección IP que se indicó en la variable de entorno \emph{WIILOAD}, de otro modo no ejecutará el programa.
\item \emph{clean}: Objetivo que limpia de archivos temporales el proyecto. Elimina la carpeta donde se almacenan los ficheros objeto, y ejecuta también los objetivos \emph{clean} de cada biblioteca externa.
\end{itemize}

La última parte de este ejemplo se encarga de comprobar que se cumplan las dependencias de los módulos a compilar, de generar un ejecutable \emph{.elf} a partir de los módulos objeto del proyecto, y en último lugar, de crear el ejecutable definitivo \emph{.dol} a partir del archivo \emph{.elf}.

\subsection{Ejecución del programa}

Una vez generado nuestro programa, existen dos maneras de ejecutarlo en la videoconsola. La primera, ya descrita, es lanzarlo a través del objetivo \emph{make run}, para lo cual necesitamos tener correctamente conectada la Nintendo Wii a la red local, y \emph{wiiload} bien configurado con la IP de la consola. \\

La otra manera de hacerlo, que es la necesaria para poder distribuir el programa, es guardar el ejecutable en un directorio de la tarjeta SD de la consola, cuya ruta debe ser \emph{/apps/XXX/boot.dol}, donde XXX es el nombre unix de nuestra aplicación (importante que no contenga espacios). Nótese que tanto el nombre \emph{boot.dol} como el hecho de que el directorio que lo contiene esté dentro del directorio \emph{/apps} es obligatorio. \\

Debido a las características de \emph{HomeBrew Channel}, podemos acompañar nuestro ejecutable con una imagen que servirá de icono para la aplicación (debe tener formato PNG, llamarse \emph{icon.png}, y tener un tamaño de 128 píxeles de ancho y 48 de alto), y un fichero XML con la información que deseemos aportar sobre el programa (debe ser un XML con un formato concreto, llamado \emph{meta.xml}). Ambos ficheros, la imagen y el archivo de datos, deben ir en el mismo directorio que el ejecutable. \\

Una referencia sobre los nodos del fichero de datos \emph{meta.xml} que acompaña a una aplicación se puede encontrar a continuación:

\begin{lstlisting}[style=XML]
<?xml version="1.0" encoding="UTF-8" standalone="yes"?>
<app version="1">
  <name>El nombre de la aplicacion</name>
  <coder>Autor o autores del codigo de la aplicacion.</coder>
  <version>La version de la aplicacion.</version>
  <release_date>Dia de lanzamiento. Formato: AAAAMMDD</release_date>
  <short_description>Comentario que se muestra en el menu principal. No se recomienda que sea mayor de 30 caracteres</short_description>
  <long_description>Descripcion detallada del programa</long_description>
</app>
\end{lstlisting}

\clearpage

